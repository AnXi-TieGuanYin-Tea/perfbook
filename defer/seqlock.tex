% defer/seqlock.tex

\section{Sequence Locks}
\label{sec:defer:Sequence Locks}

Sequence locks are used in the Linux kernel for read-mostly data that
must be seen in a consistent state by readers.
However, unlike reader-writer locking, readers do not exclude writers.
Instead, sequence-lock readers \emph{retry} an operation if they detect
activity from a concurrent writer.

\QuickQuiz{}
	Why isn't this sequence-lock discussion in Chapter~\ref{chp:Locking},
	you know, the one on \emph{locking}?
\QuickQuizAnswer{
	The sequence-lock mechanism is really a combination of two
	separate synchronization mechanisms, sequence counts and
	locking.
	In fact, the sequence-count mechanism is available separately
	in the Linux kernel via the
	\co{write_seqcount_begin()} and \co{write_seqcount_end()}
	primitives.

	However, the combined \co{write_seqlock()} and
	\co{write_sequnlock()} primitives are used much more heavily
	in the Linux kernel.
	More importantly, many more people will understand what you
	mean if you san ``sequence lock'' than if you say
	``sequence count''.

	So this section is entitled ``Sequence Locks'' so that people
	will understand what it is about just from the title, and
	it appears in the ``Deferred Processing'' because (1) of the
	emphasis on the ``sequence count'' aspect of ``sequence locks''
	and (2) because a ``sequence lock'' is much more than merely
	a lock.
} \QuickQuizEnd
