% formal/axiomatic.tex

\section{Axiomatic Approaches}
\label{sec:formal:Axiomatic Approaches}
\OriginallyPublished{Section}{sec:formal:Axiomatic Approaches}{Axiomatic Approaches}{Linux Weekly News}{PaulEMcKenney2014weakaxiom}

\begin{figure*}[tb]
{ \scriptsize
\begin{verbbox}
 1 PPC IRIW.litmus
 2 ""
 3 (* Traditional IRIW. *)
 4 {
 5 0:r1=1; 0:r2=x;
 6 1:r1=1;         1:r4=y;
 7         2:r2=x; 2:r4=y;
 8         3:r2=x; 3:r4=y;
 9 }
10 P0           | P1           | P2           | P3           ;
11 stw r1,0(r2) | stw r1,0(r4) | lwz r3,0(r2) | lwz r3,0(r4) ;
12              |              | sync         | sync         ;
13              |              | lwz r5,0(r4) | lwz r5,0(r2) ;
14 
15 exists
16 (2:r3=1 /\ 2:r5=0 /\ 3:r3=1 /\ 3:r5=0)
\end{verbbox}
}
\centering
\theverbbox
\caption{IRIW Litmus Test}
\label{fig:formal:IRIW Litmus Test}
\end{figure*}

Although the PPCMEM tool can solve the famous ``independent reads of
independent writes'' (IRIW) litmus test shown in
Figure~\ref{fig:formal:IRIW Litmus Test}, doing so requires no less than
fourteen CPU hours and generates no less than ten gigabytes of state space.
That said, this situation is a great improvement over that before the advent
of PPCMEM, where solving this problem required perusing volumes of
reference manuals, attempting proofs, discussing with experts, and
being unsure of the final answer.
Although fourteen hours can seem like a long time, it is much shorter
than weeks or even months.

However, the time required is a bit surprising given the simplicity
of the litmus test, which has two threads storing to two separate variables
and two other threads loading from these two variables in opposite
orders.
The assertion triggers if the two loading threads disagree on the order
of the two stores.
This litmus test is simple, even by the standards of memory-order litmus
tests.

One reason for the amount of time and space consumed is that PPCMEM does
a trace-based full-state-space search, which means that it must generate
and evaluate all possible orders and combinations of events at the
architectural level.
At this level, both loads and stores correspond to ornate sequences
of events and actions, resulting in a very large state space that must
be completely searched, in turn resulting in large memory and CPU
consumption.

Of course, many of the traces are quite similar to one another, which
suggests that an approach that treated similar traces as one might
improve performace.
One such approach is the axiomatic approach of
Alglave et al.~\cite{Alglave:2014:HCM:2594291.2594347},
which creates a set of axioms to represent the memory model and then
converts litmus tests to theorems that might be proven or disproven
over this set of axioms.
The resulting tool, called ``herd'',  conveniently takes as input the
same litmus tests as PPCMEM, including the IRIW litmus test shown in
Figure~\ref{fig:formal:IRIW Litmus Test}.

\begin{figure*}[tb]
{ \scriptsize
\begin{verbbox}
 1 PPC IRIW5.litmus
 2 ""
 3 (* Traditional IRIW, but with five stores instead of just one. *)
 4 {
 5 0:r1=1; 0:r2=x;
 6 1:r1=1;         1:r4=y;
 7         2:r2=x; 2:r4=y;
 8         3:r2=x; 3:r4=y;
 9 }
10 P0           | P1           | P2           | P3           ;
11 stw r1,0(r2) | stw r1,0(r4) | lwz r3,0(r2) | lwz r3,0(r4) ;
12 addi r1,r1,1 | addi r1,r1,1 | sync         | sync         ;
13 stw r1,0(r2) | stw r1,0(r4) | lwz r5,0(r4) | lwz r5,0(r2) ;
14 addi r1,r1,1 | addi r1,r1,1 |              |              ;
15 stw r1,0(r2) | stw r1,0(r4) |              |              ;
16 addi r1,r1,1 | addi r1,r1,1 |              |              ;
17 stw r1,0(r2) | stw r1,0(r4) |              |              ;
18 addi r1,r1,1 | addi r1,r1,1 |              |              ;
19 stw r1,0(r2) | stw r1,0(r4) |              |              ;
20 
21 exists
22 (2:r3=1 /\ 2:r5=0 /\ 3:r3=1 /\ 3:r5=0)
\end{verbbox}
}
\centering
\theverbbox
\caption{Expanded IRIW Litmus Test}
\label{fig:formal:Expanded IRIW Litmus Test}
\end{figure*}

However, where PPCMEM requires 14 CPU hours to solve IRIW, herd does so
in 17 milliseconds, which represents a speedup of more than six orders
of magnitude.
That said, the problem is exponential in nature, so we should expect
herd to exhibit exponential slowdowns for larger problems.
And this is exactly what happens, for example, if we add four more writes
per writing CPU as shown in
Figure~\ref{fig:formal:Expanded IRIW Litmus Test},
herd slows down by a factor of more than 50,000, requiring more than
15 \emph{minutes} of CPU time.
Adding threads also results in exponential
slowdowns~\cite{PaulEMcKenney2014weakaxiom}.

Despite their exponential nature, both PPCMEM and herd have proven quite
useful for checking key parallel algorithms, including the queued-lock
handoff on x86 systems.
The weaknesses of the herd tool are similar to those of PPCMEM, which
were described in
Section~\ref{sec:formal:PPCMEM Discussion}.
There are some obscure (but very real) cases for which the PPCMEM and
herd tools disagree, and as of late 2014 resolving these disagreements
was ongoing.

Longer term, the hope is that the axiomatic approaches incorporate
axioms describing higher-level software artifacts.
This could potentially allow axiomatic verification of much larger
software systems.
Another alternative is to press the axioms of boolean logic into service,
as described in the next section.
