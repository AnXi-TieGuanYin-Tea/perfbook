% count/count.tex

\chapter{Counting}
\label{chp:Counting}

\QuickQuizChapter{chp:Counting}

Counting may seem like the easiest and most natural arithmetic operation
possible.
However, doing so in an efficient and scalable manner on a large
shared-memory multiprocessor can be quite challenging.
This chapter covers a number of special cases for which there are simple,
fast, and scalable counting algorithms.
But first, let us find out how much you already know about concurrent
counting.

\QuickQuiz{}
	Why on earth should efficient and scalable counting be hard???
\QuickQuizAnswer{
	Because the straightforward counting algorithms, for example,
	atomic operations on a shared counter, are slow and scale
	very badly, as will be seen in Section @@@.
} \QuickQuizEnd

\QuickQuiz{}
	Suppose that you need to collect statistics on the number
	of networking packets (or total number of bytes) transmitted
	and/or received.
	Packets might be transmitted or received by any CPU on
	the system.
	Suppose further that this large machine is capable of
	handling a million packets per second, and that there
	is a systems-monitoring package that reads out the count
	every five seconds.
	How would you implement this statistical counter?
\QuickQuizAnswer{
	Hint: the act of updating the counter must be blazingly
	fast, but because the counter is read out only about once
	in five million updates, the act of reading out the counter can be
	quite slow.
	In addition, the value read out normally need not be all that
	accurate---after all, since the counter is updated a thousand
	times per millisecond, we should be able to work with a value
	that is within a few thousand counts of the ``true value'',
	whatever ``true value'' might mean in this context.
	However, the value read out should maintain roughly the same
	absolute error over time.
	For example, while a 1\% error might be OK when the count
	is on the order of a million or so, it is absolutely unacceptable
	once the count reaches a trillion.
	See Section @@@.
} \QuickQuizEnd

\QuickQuiz{}
	Suppose that you need to maintain a reference count on a
	heavily used removable mass-storage device, so that you
	can tell the user when it is safe to removed the device.
	This device follows the usual removal procedure where
	the user indicates a desire to remove the device, and
	the system tells the user when it is safe to do so.
\QuickQuizAnswer{
	Hint: the act of updating the counter must be blazingly
	fast, but because the counter is read out only when the
	user wishes to remove the device, the counter read-out
	operation can be extremely slow.
	Furthermore, there is no need to be able to read out
	the counter at all unless the user has already indicated
	a desire to remove the device.
	In addition, the value read out need not be accurate
	\emph{except} that it absolutely must distinguish perfectly
	between non-zero and zero values.
	See Section @@@.
} \QuickQuizEnd

\QuickQuiz{}
	Suppose that you need to maintain a count of the number of
	structures allocated in order to fail any allocations
	once the number of structures in use exceeds a limit
	(say, 10,000).
	Suppose further that these structures are short-lived,
	and that the limit is rarely exceeded.
\QuickQuizAnswer{
	Hint: the act of updating the counter must be blazingly
	fast, but the counter is read out each time that the
	counter is increased.
	However, the value read out need not be accurate
	\emph{except} that it absolutely must distinguish perfectly
	between values below the limit and values greater than or
	equal to the limit.
	See Section @@@.
} \QuickQuizEnd
