% preface.tex

\chapter*[Preface]{Preface}

Parallel programming has earned a reputation as one of the most
difficult areas a hacker can tackle.
Papers and textbooks warn of the perils of deadlock, livelock,
race conditions, non-determinism, Amdahl's-Law limits to scaling,
and excessive realtime latencies.

And these perils are quite real; we authors have accumulated uncounted
% 2006:
%	16 for Paul E. McKenney
years of experience dealing with them, and all of the emotional scars,
grey hairs, and hair loss that go with such an experience.
However, these perils are commonly exaggerated.
Parallel programming {\em is} difficult, but so is sequential programming.
One important, but too-frequently overlooked, reason that parallel
programming has such a nasty reputation is not that parallel programming
is inhumanly difficult, but rather that parallel systems have until
quite recently been exquisitely rare and hideously expensive.
Until quite recently, only a very few lucky hackers enjoyed the rare and
exciting experience of parallel programming on a day-to-day basis.

With the advent of inexpensive multithreaded and multicore CPUs,
{\em everyone} can easily own a parallel system.
Papers calling out the advantages of multicore CPUs were published
as early as 1996~\cite{Olukotun96}, IBM introduced simultaneous multi-threading
into its high-end POWER family in 2000, and multi-core in 2001.
Intel introduced hyperthreading into its commodity Pentium line in
November 2000, and both AMD and Intel introduced
dual-core CPUs in 2005.
Sun followed with the multi-core/multi-threaded Niagara in late 2005.

This availability of low-cost parallel hardware means that the formerly
rare experience of parallel programming will become commonplace.
With experience comes expertise; with practice, what was once hard
becomes easy.
Therefore, over the first few decades of this millenium, we can expect
the formerly forbidding practice of parallel programming to become
much more familiar and friendly.

If parallel programming will soon become easy, why bother writing a
book about it?
This book is written in the hope that summarizing experience in
parallel programming will free a new generation of parallel hackers
from the need to slowly and painstakingly reinvent old wheels, instead
focusing their energy and creativity on new frontiers.
We hope that this book will be useful to you, and that the experience
of parallel programming will bring you as much fun, excitement, and
challenge as it has provided us over the years.
