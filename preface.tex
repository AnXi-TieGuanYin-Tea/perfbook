% preface.tex

\chapter*{Preface}

Parallel programming has earned a reputation as one of the most
difficult areas a hacker can tackle.
Papers and textbooks warn of the perils of deadlock, livelock,
race conditions, non-determinism, Amdahl's-Law limits to scaling,
and excessive realtime latencies.
And these perils are quite real; we authors have accumulated uncounted
% 2008:
%	18 for Paul E. McKenney
years of experience dealing with them, and all of the emotional scars,
grey hairs, and hair loss that go with such an experience.

However, new technologies have always been difficult to use, but have
over time become easier.
For example, there was a time when the ability to drive a car was a rare
skill, but in many developed countries, this skill is now commonplace.
This dramatic change came about for two basic reasons: (1) cars became
cheaper and more readily available, so that more people had the
opportunity to learn to drive, and (2) cars became simpler to operate,
due to automatic transmissions, automatic chokes, automatic starters,
improved reliability,
and a host of other technological improvements.

The same is true of a host of other technologies, including computers
themselves.
It is no longer necessary to operate a keypunch in order to program.
Spreadsheets allow most non-programmers to get results from their computers
that would have required a team of specialists a few decades ago.
Perhaps the most compelling example is surfing the web and creating
web content, which since the early 2000s has been easily done by
untrained, uneducated people using wikis and other social-networking
tools.
As recently as 1968 this was a far-out research
project~\cite{DouglasEngelbart1968}, described at
the time as
``like a UFO landing on the White House lawn''\cite{ScottGriffen2000}.
% http://www.ibiblio.org/pioneers/englebart.html
% http://www.histech.rwth-aachen.de/www/quellen/engelbart/ahi62index.html

Therefore, if you wish to argue that parallel programming will remain
as difficult as it is currently perceived by many to be, you must shoulder
the burden of proof, keeping in mind the centuries of experience to
the contrary in a variety of fields of endeavor.

As indicated by its title, this book takes a different approach.
Rather than complain about the difficulty of parallel programming,
it instead examines the reasons why parallel programming is
difficult, and then works to help the reader to overcome these
difficulties.
As will be seen, these difficulties have fallen into several categories,
including:

\begin{enumerate}
\item	The historic high cost and relative rarity of parallel systems.
\item	The typical researcher's and practitioner's lack of experience
	with parallel systems.
\item	The paucity of publicly accessible parallel code.
\item	The lack of a widely understood engineering discipline of
	parallel programming.
\item	The high cost of communication relative to that of processing,
	even in tightly coupled shared-memory computers.
\end{enumerate}

Many of these historic difficulties are in the process of being overcome,
however, others have thus far remained in full force.
First, over the past few decades, the cost of parallel systems
has decreased from many multiples of that of a house to a fraction of 
that of a used car, thanks to the advent of multicore systems.
Papers calling out the advantages of multicore CPUs were published
as early as 1996~\cite{Olukotun96}, IBM introduced simultaneous multi-threading
into its high-end POWER family in 2000, and multicore in 2001.
Intel introduced hyperthreading into its commodity Pentium line in
November 2000, and both AMD and Intel introduced
dual-core CPUs in 2005.
Sun followed with the multicore/multi-threaded Niagara in late 2005.
In fact, in 2008, it is becoming difficult
to find a single-CPU desktop system.

Second, the advent of low-cost and readily available multicore system
means that the once-rare experience of parallel programming is
now available to almost all researchers and practitioners.
In fact, parallel systems are now well within the budget of students
and hobbyists.
We can therefore expect greatly increased levels of invention and
innovation surrounding parallel systems, and that increased familiarity
will over time make once-forbidding field of parallel programming
much more friendly and commonplace.

Third, where in the 20\textsuperscript{th} century, large systems of
highly parallel software were almost always closely guarded proprietary
secrets, the 21\textsuperscript{st} century has seen numerous
open-source (and thus publicly available) parallel software projects,
including the Linux kernel~\cite{Torvalds2.6kernel},
database systems~\cite{PostgreSQL2008,MySQL2008},
and message-passing systems~\cite{OpenMPI2008}.
This book will draw primarily from the Linux kernel, but will
use numerous examples suitable for user-level applications.

Fourth, even though the large-scale parallel-programming projects of
the 1980s and 1990s were almost all proprietary projects, these
projects have seeded the community with a cadre of developers who
understand the engineering discipline required to develop production-quality
parallel code.
A major purpose of this book is to present this engineering discipline.

Unfortunately, the fifth difficulty, the high cost of communication
relative to that of processing, remains largely in force.
Although this difficulty has been receiving increasing attention during
the new millenium, according to Stephen Hawkings,
the finite speed of light and the atomic
nature of matter is likely to limit progress in this area.
Fortunately, this difficulty has been in force since the late 1980s,
so that the aforementioned engineering discipline has evolved practical
and effective strategies for handling it.

In presenting this engineering discipline, this book will examine
the specific development tasks peculiar to parallel programming,
and describe how they may be dealt with and, in some surprisingly
common special cases, automated.

This book is written in the hope that presenting the engineering
discipline underlying successful
parallel-programming projects will free a new generation of parallel hackers
from the need to slowly and painstakingly reinvent old wheels, instead
focusing their energy and creativity on new frontiers.
We expect that 
We hope that this book will be useful to you, and that the experience
of parallel programming will bring you as much fun, excitement, and
challenge as it has provided us over the years.
