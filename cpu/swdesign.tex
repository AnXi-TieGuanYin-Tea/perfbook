% cpu/swdesign.tex

\section{Software Design Implications}
\label{sec:cpu:Software Design Implications}

The values of the ratios in
Table~\ref{tab:cpu:Performance of Synchronization Mechanisms on 4-CPU 1.8GHz AMD Opteron 844 System}
are critically important, as they limit the
efficiency of a given parallel application.
To see this, suppose that the parallel application uses CAS
operations to communicate among threads.
These CAS operations will typically involve a cache miss, that is, assuming
that the threads are communicating primarily with each other rather than
with themselves.
Suppose further that the unit of work corresponding to each CAS communication
operation takes 300ns, which is sufficient time to compute several
floating-point transcendental functions.
Then about half of the execution time will be consumed by the CAS
communication operations!
This in turn means that a two-CPU system running such a parallel program
would run no faster than one a sequential implementation running on a
single CPU.

The situation is even worse in the distributed-system case, where the
latency of a single communications operation might take as long as
thousands or even millions of floating-point operations.
This illustrates how important it is for communications operations to
be extremely infrequent and to enable very large quantities of processing.

\QuickQuiz{}
	Given that distributed-systems communication is so horribly
	expensive, why does anyone bother with them?
\QuickQuizAnswer{
	There are a number of reasons:

	\begin{enumerate}
	\item	Shared memory multiprocessor have strict size limits,
		If you need more than a few thousand CPUs, you have no
		choice but to use a distributed system.
	\item	Extremely large shared-memory systems tend to be
		quite expensive and to have even longer cache-miss
		latencies than does the small four-CPU system
		shown in
		Table~\ref{tab:cpu:Performance of Synchronization Mechanisms on 4-CPU 1.8GHz AMD Opteron 844 System}.
	\item	The distributed-systems communications latencies do
		not necessarily consume the CPU, which can often allow
		computation to proceed in parallel with message transfer.
	\item	Many important problems are ``embarrassingly parallel'',
		so that extremely large quantities of processing may
		be enabled by a very small number of messages.
		SETI@HOME~\cite{SETIatHOME2008}
		is but one example of such an application.
		These sorts of applications can make good use of networks
		of computers despite extremely long communications
		latencies.
	\end{enumerate}

	It is likely that continued work on parallel applications will
	increase the number of embarrassingly parallel applications that
	can run well on machines and/or clusters having long communications
	latencies.
	That said, greatly reduced hardware latencies would be an
	extremely welcome development.
} \QuickQuizEnd

The lesson should be quite clear:
parallel algorithms must be explicitly designed
to run nearly independent threads.
The less frequently the threads communicate, whether by atomic operations,
locks, or explicit messages, the better the application's performance
and scalability will be.
In short, achieving excellent parallel performance and scalability means
striving for embarrassingly parallel algorithms and implementations,
whether by careful choice of data structures and algorithms, use of
existing parallel applications and environments, or transforming the
problem into one for which an embarrassingly parallel solution exists.

Chapter~\ref{cha:Partitioning and Synchronization Design}
will discuss design disciplines that promote performance and scalability.
