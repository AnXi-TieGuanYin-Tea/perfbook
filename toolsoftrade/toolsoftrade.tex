% toolsoftrade/toolsoftrade.tex

\chapter{Tools of the Trade}
\label{chp:Tools of the Trade}

\QuickQuizChapter{chp:Tools of the Trade}

This chapter provides a brief introduction to some basic tools of the
parallel-programming trade, focusing mainly on those available to
user applications running on operating systems similar to Linux.
Section~\ref{sec:toolsoftrade:Scripting Languages} begins with
scripting languages,
Section~\ref{sec:toolsoftrade:POSIX Multiprocessing}
describes the multi-process parallelism supported by the POSIX API,
Section~\ref{sec:toolsoftrade:POSIX Multiprocessing} touches on POSIX threads,
and finally,
Section~\ref{sec:toolsoftrade:Atomic Operations}
describes atomic operations.

Please note that this chapter provides but a brief introduction.
More detail is available from the references cited, and more information
on how best to use these tools will be provided in later chapters.

\section{Scripting Languages}
\label{sec:toolsoftrade:Scripting Languages}

The Linux shell scripting languages provide simple but effective ways
of managing parallelism.
For example, suppose that you had a program \url{compute_it}
that you needed to run twice with two different sets of arguments.
This can be accomplished as follows:

\vspace{5pt}
\begin{minipage}[t]{\columnwidth}
\begin{verbatim}
  1 compute_it 1 > compute_it.1.out &
  2 compute_it 2 > compute_it.2.out &
  3 wait
  4 cat < compute_it.1.out
  5 cat < compute_it.2.out
\end{verbatim}
\end{minipage}
\vspace{5pt}

Lines~1 and 2 launch two instances of this program, redirecting their
output to two separate files, with the \url{&} character directing the
shell to run the two instances of the program in the background.
Line~3 waits for both instances to complete, and lines~4 and 5
display their output.
% @@@ Maui scheduler, load balancing, BOINC, and so on.

\QuickQuiz{}
	But this silly shell script isn't a \emph{real} parallel program!!!
	Why bother with such trivia???
\QuickQuizAnswer{
	Because you should \emph{never} forget the simple stuff!!!

	Please keep in mind that the title of this book is
	``Is Parallel Programming Hard, And, If So, What Can You Do About It?''.
	One of the most effective things you can do about it is to
	avoid forgetting the simple stuff!
	After all, if you choose to do parallel programming the hard
	way, you have no one but yourself to blame for it being hard.
} \QuickQuizEnd

\QuickQuiz{}
	Is there a simpler way to create a parallel shell script?
	If so, how?  If not, why not?
\QuickQuizAnswer{
	One straightforward approach is the shell pipeline:
\vspace{5pt}
\begin{minipage}[t]{\columnwidth}
\small
\begin{verbatim}
grep $pattern1 | sed -e 's/a/b/' | sort
\end{verbatim}
\end{minipage}
\vspace{5pt}
	For a sufficiently large input file,
	\url{grep} will pattern-match in parallel with \url{sed}
	editing and with the input processing of \url{sort}.
	See the file \url{parallel.sh} for a demonstration of
	shell-script parallelism and pipelining.
} \QuickQuizEnd

For another example, the \url{make} software-build scripting language
provides a \url{-j} option that specifies how much parallelism should be
introduced into the build process.
For example, typing \url{make}~\url{-j4} when building a Linux kernel
specifies that up to four parallel compiles be carried out concurrently.

It is hoped that these simple examples convince you that parallel
programming need not always be complex or difficult.

\QuickQuiz{}
	But if script-based parallel programming is so easy, why
	bother with anything else?
\QuickQuizAnswer{
	In fact, it is quite likely that a very large fraction of
	parallel programs in use today are script-based.
	However, script-based parallelism does have its limitations:
	\begin{enumerate}
	\item	Creation of new processes is usually quite heavyweight,
		involving the expensive \url{fork()} and \url{exec()}
		system calls.
	\item	Sharing of data, including pipelining, typically involves
		expensive file I/O.
	\item	The reliable synchronization primitives available to
		scripts also typically involve expensive file I/O.
	\end{enumerate}
	These limitations require that script-based parallelism use
	coarse-grained parallelism, with each unit of work having
	execution time of at least tens of milliseconds, and preferably
	much longer.

	Those requiring finer-grained parallelism are well advised to
	think hard about their problem to see if it can be expressed
	in a coarse-grained form.
	If not, they should consider using other parallel-programming
	environments, such as those discussed in
	Section~\ref{sec:toolsoftrade:POSIX Multiprocessing}.
} \QuickQuizEnd

\section{POSIX Multiprocessing}
\label{sec:toolsoftrade:POSIX Multiprocessing}

This section scratches the surface of the
POSIX environment, including pthreads~\cite{OpenGroup1997pthreads},
as this environment is readily available and widely implemented.
Section~\ref{sec:toolsoftrade:POSIX Process Creation and Destruction}
provides a glimpse of the POSIX \url{fork()} and related primitives,
Section~\ref{sec:toolsoftrade:POSIX Thread Creation and Destruction}
touches on thread creation and distruction,
Section~\ref{sec:toolsoftrade:POSIX Locking} gives a brief overview
of POSIX locking, and, finally,
Section~\ref{sec:toolsoftrade:Linux-Kernel Equivalents to POSIX Operations}
presents the analogous operations within the Linux kernel.

\subsection{POSIX Process Creation and Destruction}
\label{sec:toolsoftrade:POSIX Process Creation and Destruction}

Processes are created using the \url{fork()} primitive, they may
be destroyed using the \url{kill()} primitive, they may destroy
themselves using the \url{exit()} primitive.
A process executing a \url{fork()} primitive is said to be the ``parent''
of the newly created process.
A parent may wait on its children using the \url{wait()} primitive.

Please note that the examples in this section are quite simple.
Real-world applications using these primitives might need to manipulate
signals, file descriptors, shared memory segments, and any number of
other resources.
In addition, some applications need to take specific actions if a given
child terminates, and might also need to be concerned with the reason
that the child terminated.
These concerns can of course add substantial complexity to the code.
For more information, see any of a number of textbooks on the
subject~\cite{WRichardStevens1992}.

\begin{figure}[tbp]
{ \scriptsize
\begin{verbatim}
  1 pid = fork();
  2 if (pid == 0) {
  3   /* child */
  4 } else if (pid < 0) {
  5   /* parent, upon error */
  6   perror("fork");
  7   exit(-1);
  8 } else {
  9   /* parent, pid == child ID */
 10 }
\end{verbatim}
}
\caption{Using the fork() Primitive}
\label{fig:toolsoftrade:Using the fork() Primitive}
\end{figure}

If \url{fork()} succeeds, it returns twice, once for the parent
and again for the child.
The value returned from \url{fork()} allows the caller to tell
the difference, as shown in
Figure~\ref{fig:toolsoftrade:Using the fork() Primitive}.
Line~1 executes the \url{fork()} primitive, and saves its return value
in local variable \url{pid}.
Line~2 checks to see if \url{pid} is zero, in which case, this is the
child, which continues on to execute line~3.
As noted earlier, the child may terminate via the \url{exit()} primitive.
Otherwise, this is the parent, which checks for an error return from
the \url{fork()} primitive on line~4, and prints an error and exits
on lines~5-7 if so.
Otherwise, the \url{fork()} has executed successfully, and the parent
therefore executes line 9 with the variable \url{pid} containing the
process ID of the child.

\begin{figure}[tbp]
{ \scriptsize
\begin{verbatim}
  1 void waitall(void)
  2 {
  3   int pid;
  4   int status;
  5 
  6   for (;;) {
  7     pid = wait(&status);
  8     if (pid == -1) {
  9       if (errno == ECHILD)
 10         break;
 11       perror("wait");
 12       exit(-1);
 13     }
 14   }
 15 }
\end{verbatim}
}
\caption{Using the wait() Primitive}
\label{fig:toolsoftrade:Using the wait() Primitive}
\end{figure}

The parent process may use the \url{wait()} primitive to wait for its children
to complete.
However, use of this primitive is a bit more complicated than its shell-script
counterpart, as each invocation of \url{wait()} waits for but one child
process.
It is therefore customary to wrap \url{wait()} into a function similar
to the \url{waitall()} function shown in
Figure~\ref{fig:toolsoftrade:Using the wait() Primitive},
this \url{waitall()} function having semantics similar to the
shell-script \url{wait} command.
Each pass through the loop spanning lines~6-15 waits on one child process.
Line~7 invokes the \url{wait()} primitive, which blocks until a child process
exits, and returns that child's process ID.
If the process ID is instead -1, this indicates that the \url{wait()}
primitive was unable to wait on a child.
If so, line~9 checks for the \url{ECHILD} errno, which indicates that there
are no more child processes, so that line~10 exits the loop.
Otherwise, lines~11 and 12 print an error and exit.

\QuickQuiz{}
	Why does this \url{wait()} primitive need to be so complicated?
	Why not just make it work like the shell-script \url{wait} does?
\QuickQuizAnswer{
	Some parallel applications need to take special action when
	specific children exit, and therefore need to wait for each
	child individually.
	In addition, some parallel applications need to detect the
	reason that the child died.
	As we saw in Figure~\ref{fig:toolsoftrade:Using the wait() Primitive},
	it is not hard to build a \url{waitall()} function out of
	the \url{wait()} function, but it would be impossible to
	do the reverse.
	Once the information about a specific child is lost, it is lost.
} \QuickQuizEnd

\begin{figure}[tbp]
{ \scriptsize
\begin{verbatim}
  1 int x = 0;
  2 int pid;
  3 
  4 pid = fork();
  5 if (pid == 0) { /* child */
  6   x = 1;
  7   printf("Child process set x=1\n");
  8   exit(0);
  9 }
 10 if (pid < 0) { /* parent, upon error */
 11   perror("fork");
 12   exit(-1);
 13 }
 14 waitall();
 15 printf("Parent process sees x=%d\n", x);
\end{verbatim}
}
\caption{Processes Created Via fork() Do Not Share Memory}
\label{fig:toolsoftrade:Processes Created Via fork() Do Not Share Memory}
\end{figure}

It is critically important to note that the parent and child do \url{not}
share memory.
This is illustrated by the program shown in
Figure~\ref{fig:toolsoftrade:Processes Created Via fork() Do Not Share Memory},
in which the child sets a global variable \url{x} to 1 on line~6,
prints a message on line~7, and exits on line~8.
The parent continues at line~14, where it waits on the child,
and on line~15 finds that its copy of the variable \url{x} is still zero.

\QuickQuiz{}
	Isn't there a lot more to \url{fork()} and \url{wait()}
	than discussed here?
\QuickQuizAnswer{
	Indeed there is!
	There are any number of textbooks that cover these primitives
	in great detail,
	and the truly motivated can read manpages, existing parallel
	applications using these primitives, as well as the
	Linux-kernel implementations.
} \QuickQuizEnd

The finest-grained parallelism of course requires shared memory, and
this is covered in
Section~\ref{sec:toolsoftrade:POSIX Thread Creation and Destruction}.
That said, shared-memory parallelism can be significantly more complex
than fork-join parallelism.

\subsection{POSIX Thread Creation and Destruction}
\label{sec:toolsoftrade:POSIX Thread Creation and Destruction}

\url{pthread_create()}

\url{pthread_exit()}

\url{pthread_join()}

\subsection{POSIX Locking}
\label{sec:toolsoftrade:POSIX Locking}

\url{pthread_mutex_lock()}

\url{pthread_mutex_unlock()}

\subsection{Linux-Kernel Equivalents to POSIX Operations}
\label{sec:toolsoftrade:Linux-Kernel Equivalents to POSIX Operations}

\section{Atomic Operations}
\label{sec:toolsoftrade:Atomic Operations}
