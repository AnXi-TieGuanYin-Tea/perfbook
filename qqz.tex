\QuickQAC{chp:Introduction}
\QuickQ{Come on now!!!
	Parallel programming has been known to be exceedingly
	hard for many decades.
	So just why are you trying to tell me differently now?}

	If you really believe that parallel programming is exceedingly
	hard, then you should have a ready answer to the question
	``Why is parallel programming hard?''
	One could list any number of reasons, ranging from deadlocks to
	race conditions to testing coverage, but the real answer is that
	{\em it is not really all that hard}.
	After all, if parallel programming was really so horribly difficult,
	how could a large number of open-source projects, ranging from Apache
	to MySQL to the Linux kernel, have managed to master it?

	A better question might be: ''Why is parallel programming {\em
	perceived} to be so difficult?''
	To see the answer, let's go back to the year 1991.
	Paul McKenney was walking across the parking lot to Sequent's
	benchmarking center carrying six dual-80486 Sequent Symmetry CPU
	boards, when he suddenly realized that he was carrying several
	times the price of the house he had just purchased.\footnote{
		Yes, this sudden realization {\em did} cause him to walk quite
		a bit more carefully.
		Why do you ask?}
	This high cost of parallel systems meant that
	parallel programming was restricted to a privileged few who
	worked for an employer who either manufactured or could afford to
	purchase machines costing upwards of \$100,000 --- in 1991 dollars US.

	In contrast, in 2006, Paul finds himself typing these words on a
	dual-core x86 laptop.
	Unlike the dual-80486 CPU boards, this laptop also contains
	2GB of main memory, a 60GB disk drive, a display, Ethernet,
	USB ports, wireless, and Bluetooth.
	And the laptop is more than an order of magnitude cheaper than
	even one of those dual-80486 CPU boards, even before taking inflation
	into account.

	Parallel systems have truly arrived.
	They are no longer the sole domain of a privileged few, but something
	available to almost everyone.

	The earlier restricted availability of parallel hardware is
	the \emph{real} reason that parallel programming is considered
	so difficult.
	After all, it is quite difficult to learn to program even the simplest
	machine if you have no access to it.
	Since the age of rare and expensive parallel machines is for the most
	part behind us, the age during which
	parallel programming is perceived to be mind-crushingly difficult is
	coming to a close.\footnote{
		Parallel programming is in some ways more difficult than
		sequential programming, for example, parallel validation
		is more difficult.
		But no longer mind-crushingly difficult.}

\QuickQ{How could parallel programming \emph{ever} be as easy
	   as sequential programming???}

	   It depends on the programming environment.
	   SQL~\cite{DIS9075SQL92} is an underappreciated success
	   story, as it permits programmers who know nothing about parallelism
	   to keep a large parallel system productively busy.
	   We can expect more variations on this theme as parallel
	   computers continue to become cheaper and more readily available.
	   For example, one possible contender in the scientific and
	   technical computing arena is Matlab*p @@@ cite @@@,
	   which is an attempt to automatically parallelize comon
	   matrix operations.

\QuickQ{Where are the answers to the Quick Quizzes found?}

	In Appendix~\ref{chp:Answers to Quick Quizzes} starting on
	page~\pageref{chp:Answers to Quick Quizzes}.

	Hey, I thought I owed you an easy one!!!

\QuickQAC{chp:Hardware and its Habits}
\QuickQ{What types of machines would allow atomic operations on
	multiple data elements?}

	One answer to this question is that it is often possible to
	pack multiple elements of data into a single machine word,
	which can then be manipulated atomically.

	A more trendy answer would be machines supporting transactional
	memory~\cite{DBLomet1977SIGSOFT}.
	However, such machines are still (as of 2008) research
	curiosities.
	The jury is still out on the applicability of transactional
	memory~\cite{McKenney2007PLOSTM,DonaldEPorter2007TRANSACT,
	ChistopherJRossbach2007a}.

\QuickQAC{cha:SMP Synchronization Design}
\QuickQ{In what situation would hierarchical locking work well?}

	If the comparison on line~31 of
	Figure~\ref{fig:SMPdesign:Hierarchical-Locking Hash Table Search}
	were replaced by a much heavier-weight operation,
	then releasing {\tt bp->bucket\_lock} \emph{might} reduce lock
	contention enough to outweigh the overhead of the extra
	acquisition and release of {\tt cur->node\_lock}.

\QuickQ{In Figure~\ref{fig:SMPdesign:Allocator Cache Performance},
	there is a pattern of performance rising with increasing run
	length in groups of three samples, for example, for run lengths
	10, 11, and 12.
	Why?}

	This is due to the per-CPU target value being three.
	A run length of 12 must acquire the global-pool lock twice,
	while a run length of 13 must acquire the global-pool lock
	three times.

\QuickQ{Allocation failures were observed in the two-thread
	tests at run lengths of 19 and greater.
	Given the global-pool size of 40 and the per-CPU target
	pool size of three, what is the smallest allocation run
	length at which failures can occur?}

	The exact solution to this problem is left as an exercise to
	the reader.
	The first solution received will be credited to its submitter.
	As a rough rule of thumb, the global pool size should be at least
	$m+2sn$, where
	``m'' is the maximum number of elements allocated at a given time,
	``s'' is the per-CPU pool size,
	and ``n'' is the number of CPUs.

\QuickQAC{chp:defer:Deferred Processing}
\QuickQ{Why isn't it necessary to guard against cases where
	   one CPU acquires a reference just after another
	   CPU releases the last reference?}

	  Because a CPU must already hold a reference in order
	  to legally acquire another reference.
	  Therefore, if one CPU releases the last reference,
	  there cannot possibly be any CPU that is permitted
	  to acquire a new reference.
	  This same fact allows the non-atomic check in line~22
	  of Figure~\ref{fig:defer:Linux Kernel kref API}.

\QuickQ{If the check on line~22 of
	   Figure~\ref{fig:defer:Linux Kernel kref API} fails, how
	   could the check on line~23 possibly succeed?}

	  Suppose that {\tt kref\_put()} is protected by RCU, so
	  that two CPUs might be executing line~22 concurrently.
	  Both might see the value ``2'', causing both to then
	  execute line~23.
	  One of the two instances of {\tt atomic\_dec\_and\_test()}
	  will decrement the value to zero and thus return 1.

\QuickQ{How can it possibly be safe to non-atomically check
	   for equality with ``1'' on line~22 of
	   Figure~\ref{fig:defer:Linux Kernel kref API}?}

	  Remember that it is not legal to call either {\tt kref\_get()}
	  or {\tt kref\_put()} unless you hold a reference.
	  If the reference count is equal to ``1'', then there
	  can't possibly be another CPU authorized to change the
	  value of the reference count.

\QuickQ{Why can't the check for a zero reference count be
	   made in a simple ``if'' statement with an atomic
	   increment in its ``then'' clause?}

	  Suppose that the ``if'' condition completed, finding
	  the reference counter value equal to one.
	  Suppose that a release operation executes, decrementing
	  the reference counter to zero and therefore starting
	  cleanup operations.
	  But now the ``then'' clause can increment the counter
	  back to a value of one, allowing the object to be
	  used after it has been cleaned up.

\QuickQ{But doesn't seqlock also permit readers and updaters to get
work done concurrently?}

Yes and no.
Although seqlock readers can run concurrently with
seqlock writers, whenever this happens, the {\tt read\_seqretry()}
primitive will force the reader to retry.
This means that any work done by a seqlock reader running concurrently
with a seqlock updater will be discarded and redone.
So seqlock readers can \emph{run} concurrently with updaters,
but they cannot actually get any work done in this case.

In contrast, RCU readers can perform useful work even in presence
of concurrent RCU updaters.

\QuickQ{What prevents the {\tt list\_for\_each\_entry\_rcu()} from
getting a segfault if it happens to execute at exactly the same
time as the {\tt list\_add\_rcu()}?}

On all systems running Linux, loads from and stores
to pointers are atomic, that is, if a store to a pointer occurs at
the same time as a load from that same pointer, the load will return
either the initial value or the value stored, never some bitwise mashup
of the two.
In addition, the {\tt list\_for\_each\_entry\_rcu()} always proceeds
forward through the list, never looking back.
Therefore, the {\tt list\_for\_each\_entry\_rcu()} will either see
the element being added by {\tt list\_add\_rcu()} or it will not,
but either way, it will see a valid well-formed list.

\QuickQ{Why do we need to pass two pointers into
{\tt hlist\_for\_each\_entry\_rcu()}
when only one is needed for {\tt list\_for\_each\_entry\_rcu()}?}

Because in an hlist it is necessary to check for
NULL rather than for encountering the head.
(Try coding up a single-pointer {\tt hlist\_for\_each\_entry\_rcu()}
If you come up with a nice solution, it would be a very good thing!)

\QuickQ{How would you modify the deletion example to permit more than two
versions of the list to be active?}

One way of accomplishing this is as shown in
Figure~\ref{fig:defer:Concurrent RCU Deletion}.

\begin{figure}[htbp]
{ \centering
\begin{verbatim}
  1 spin_lock(&mylock);
  2 p = search(head, key);
  3 if (p == NULL)
  4   spin_unlock(&mylock);
  5 else {
  6   list_del_rcu(&p->list);
  7   spin_unlock(&mylock);
  8   synchronize_rcu();
  9   kfree(p);
 10 }
\end{verbatim}
}
\caption{Concurrent RCU Deletion}
\label{fig:defer:Concurrent RCU Deletion}
\end{figure}

Note that this means that multiple concurrent deletions might be
waiting in {\tt synchronize\_rcu()}.

\QuickQ{How many RCU versions of a given list can be
active at any given time?}

That depends on the synchronization design.
If a semaphore protecting the update is held across the grace period,
then there can be at most two versions, the old and the new.

However, if only the search, the update, and the
{\tt list\_replace\_rcu()} were protected by a lock, then
there could be an arbitrary number of versions active, limited only
by memory and by how many updates could be completed within a
grace period.
But please note that data structures that are updated so frequently
probably are not good candidates for RCU.
That said, RCU can handle high update rates when necessary.

\QuickQ{How can RCU updaters possibly delay RCU readers, given that the
{\tt rcu\_read\_lock()} and {\tt rcu\_read\_unlock()}
primitives neither spin nor block?}

The modifications undertaken by a given RCU updater will cause the
corresponding CPU to invalidate cache lines containing the data,
forcing the CPUs running concurrent RCU readers to incur expensive
cache misses.
(Can you design an algorithm that changes a data structure \emph{without}
inflicting expensive cache misses on concurrent readers?
On subsequent readers?)

\QuickQ{
WTF???
How the heck do you expect me to believe that RCU has a
100-femtosecond overhead when the clock period at 3GHz is more than
300 \emph{picoseconds}?}

First, consider that the inner loop used to
take this measurement is as follows:

\vspace{5pt}
\begin{minipage}[t]{\columnwidth}
\small
\begin{verbatim}
  1 for (i = 0; i < CSCOUNT_SCALE; i++) {
  2   rcu_read_lock();
  3   rcu_read_unlock();
  4 }
\end{verbatim}
\end{minipage}
\vspace{5pt}

Next, consider the effective definitions of \url{rcu_read_lock()}
and \url{rcu_read_unlock()}:

\vspace{5pt}
\begin{minipage}[t]{\columnwidth}
\small
\begin{verbatim}
  1 #define rcu_read_lock()   do { } while (0)
  2 #define rcu_read_unlock() do { } while (0)
\end{verbatim}
\end{minipage}
\vspace{5pt}

Consider also that the compiler does simple optimizations,
allowing it to replace the loop with:

\vspace{5pt}
\begin{minipage}[t]{\columnwidth}
\small
\begin{verbatim}
i = CSCOUNT_SCALE;
\end{verbatim}
\end{minipage}
\vspace{5pt}

So the "measurement" of 100 femtoseconds is simply the fixed
overhead of the timing measurements divided by the number of
passes through the inner loop containing the calls
to \url{rcu_read_lock()} and \url{rcu_read_unlock()}.
And therefore, this measurement really is in error, in fact,
in error by an arbitrary number of orders of magnitude.
As you can see by the definition of \url{rcu_read_lock()}
and \url{rcu_read_unlock()} above, the actual overhead
is precisely zero.

It certainly is not every day that a timing measurement of
100 femtoseconds turns out to be an overestimate!

\QuickQ{
Why does both the variability and overhead of rwlock decrease as the
critical-section overhead increases?}

Because the contention on the underlying
\url{rwlock_t} decreases as the critical-section overhead
increases.
However, the rwlock overhead will not quite drop to that on a single
CPU because of cache-thrashing overhead.

\QuickQ{
Is there an exception to this deadlock immunity, and if so,
what sequence of events could lead to deadlock?}

One way to cause a deadlock cycle involving
RCU read-side primitives is via the following (illegal) sequence
of statements:

\vspace{5pt}
\begin{minipage}[t]{\columnwidth}
\small
\begin{verbatim}
idx = srcu_read_lock(&srcucb);
synchronize_srcu(&srcucb);
srcu_read_unlock(&srcucb, idx);
\end{verbatim}
\end{minipage}
\vspace{5pt}

The \url{synchronize_rcu()} cannot return until all
pre-existing SRCU read-side critical sections complete, but
is enclosed in an SRCU read-side critical section that cannot
complete until the \url{synchronize_srcu()} returns.
The result is a classic self-deadlock--you get the same
effect when attempting to write-acquire a reader-writer lock
while read-holding it.

Note that this self-deadlock scenario does not apply to
RCU Classic, because the context switch performed by the
\url{synchronize_rcu()} would act as a quiescent state
for this CPU, allowing a grace period to complete.
However, this is if anything even worse, because data used
by the RCU read-side critical section might be freed as a
result of the grace period completing.

In short, do not invoke synchronous RCU update-side primitives
from within an RCU read-side critical section.

\QuickQ{
But wait!
This is exactly the same code that might be used when thinking
of RCU as a replacement for reader-writer locking!
What gives?}

This is an effect of the Law of Toy Examples:
beyond a certain point, the code fragments look the same.
The only difference is in how we think about the code.
However, this difference can be extremely important.
For but one example of the importance, consider that if we think
of RCU as a restricted reference counting scheme, we would never
be fooled into thinking that the updates would exclude the RCU
read-side critical sections.
\\ ~ \\
It nevertheless is often useful to think of RCU as a replacement
for reader-writer locking, for example, when you are replacing reader-writer
locking with RCU.

\QuickQ{
Why the dip in refcnt overhead near 6 CPUs?}

Most likely NUMA effects.
However, there is substantial variance in the values measured for the
refcnt line, as can be seen by the error bars.
In fact, standard deviations range in excess of 10% of measured
values in some cases.
The dip in overhead therefore might well be a statistical aberration.

\QuickQ{
	But what if there is an arbitrarily long series of RCU
	read-side critical sections in multiple threads, so that at
	any point in time there is at least one thread in the system
	executing in an RCU read-side critical section?
	Wouldn't that prevent any data from a \url{SLAB_DESTROY_BY_RCU}
	slab ever being returned to the system, possibly resulting
	in OOM events?}

	There could certainly be an arbitrarily long period of time
	during which at least one thread is always in an RCU read-side
	critical section.
	However, the key words in the description in
	Section~\ref{sec:deferRCU is a Way of Providing Type-Safe Memory}
	are ``in-use'' and ``pre-existing''.
	Keep in mind that a given RCU read-side critical section is
	conceptually only permitted to gain references to data elements
	that were in use at the beginning of that critical section.
	Furthermore, remember that a slab cannot be returned to the
	system until all of its data elements have been freed, in fact,
	the RCU grace period cannot start until after they have all been
	freed.

	Therefore, the slab cache need only wait for those RCU read-side
	critical sections that started before the freeing of the last element
	of the slab.
	This in turn means that any RCU grace period that begins after
	the freeing of the last element will do---the slab may be returned
	to the system after that grace period ends.

\QuickQ{
Suppose that the \url{nmi_profile()} function was preemptible.
What would need to change to make this example work correctly?}

One approach would be to use
\url{rcu_read_lock()} and \url{rcu_read_unlock()}
in \url{nmi_profile()}, and to replace the
\url{synchronize_sched()} with \url{synchronize_rcu()},
perhaps as shown in
Figure~\ref{fig:defer:Using RCU to Wait for Mythical Preemptable NMIs to Finish}.
\\ ~ \\
\begin{figure}[tbp]
{ \tt \scriptsize
\begin{verbatim}
  1 struct profile_buffer {
  2   long size;
  3   atomic_t entry[0];
  4 };
  5 static struct profile_buffer *buf = NULL;
  6 
  7 void nmi_profile(unsigned long pcvalue)
  8 {
  9   struct profile_buffer *p;
 10 
 11   rcu_read_lock();
 12   p = rcu_dereference(buf);
 13   if (p == NULL) {
 14     rcu_read_unlock();
 15     return;
 16   }
 17   if (pcvalue >= p->size) {
 18     rcu_read_unlock();
 19     return;
 20   }
 21   atomic_inc(&p->entry[pcvalue]);
 22   rcu_read_unlock();
 23 }
 24 
 25 void nmi_stop(void)
 26 {
 27   struct profile_buffer *p = buf;
 28 
 29   if (p == NULL)
 30     return;
 31   rcu_assign_pointer(buf, NULL);
 32   synchronize_rcu();
 33   kfree(p);
 34 }
\end{verbatim}
}
\caption{Using RCU to Wait for Mythical Preemptable NMIs to Finish}
\label{fig:defer:Using RCU to Wait for Mythical Preemptable NMIs to Finish}
\end{figure}


\QuickQ{Why do some of the cells in
Table~\ref{tab:defer:RCU Wait-to-Finish APIs}
have exclamation marks (``!'')?}

The API members with exclamation marks (\url{rcu_read_lock()},
\url{rcu_read_unlock()}, and \url{call_rcu()}) were the
only members of the Linux RCU API that Paul E. McKenney was aware of back
in the mid-90s.
During this timeframe, he was under the mistaken impression that
he knew all that there is to know about RCU.

\QuickQ{What happens if you mix and match?
For example, suppose you use \url{rcu_read_lock()} and
\url{rcu_read_unlock()} to delimit RCU read-side critical
sections, but then use \url{call_rcu_bh()} to post an
RCU callback?}

If there happened to be no RCU read-side critical
sections delimited by \url{rcu_read_lock_bh()} and
\url{rcu_read_unlock_bh()} at the time \url{call_rcu_bh()}
was invoked, RCU would be within its rights to invoke the callback
immediately, possibly freeing a data structure still being used by
the RCU read-side critical section!
This is not merely a theoretical possibility: a long-running RCU
read-side critical section delimited by \url{rcu_read_lock()}
and \url{rcu_read_unlock()} is vulnerable to this failure mode.

This vulnerability disappears in -rt kernels, where
RCU Classic and RCU BH both map onto a common implementation.

\QuickQ{What happens if you mix and match RCU Classic and RCU Sched?}

In a non-\url{PREEMPT} or a \url{PREEMPT} kernel, mixing these
two works "by accident" because in those kernel builds, RCU Classic and RCU
Sched map to the same implementation.
However, this mixture is fatal in \url{PREEMPT_RT} builds using the -rt
patchset, due to the fact that Realtime RCU's read-side critical
sections can be preempted, which would permit
\url{synchronize_sched()} to return before the
RCU read-side critical section reached its \url{rcu_read_unlock()}
call.
This could in turn result in a data structure being freed before the
read-side critical section was finished with it,
which could in turn greatly increase the actuarial risk experienced
by your kernel.

In fact, the split between RCU Classic and RCU Sched was inspired
by the need for preemptible RCU read-side critical sections.

\QuickQ{Why do both SRCU and QRCU lack asynchronous \url{call_srcu()}
or \url{call_qrcu()} interfaces?}

Given an asynchronous interface, a single task
could register an arbitrarily large number of SRCU or QRCU callbacks,
thereby consuming an arbitrarily large quantity of memory.
In contrast, given the current synchronous
\url{synchronize_srcu()} and \url{synchronize_qrcu()}
interfaces, a given task must finish waiting for a given grace period
before it can start waiting for the next one.

\QuickQ{Under what conditions can \url{synchronize_srcu()} be safely
used within an SRCU read-side critical section?}

In principle, you can use
\url{synchronize_srcu()} with a given \url{srcu_struct}
within an SRCU read-side critical section that uses some other
\url{srcu_struct}.
In practice, however, doing this is almost certainly a bad idea.
In particular, the code shown in
Figure~\ref{fig:defer:Multistage SRCU Deadlocks}
could still result in deadlock.

\begin{figure}[htbp]
{ \centering
\begin{verbatim}
  1 idx = srcu_read_lock(&ssa);
  2 synchronize_srcu(&ssb);
  3 srcu_read_unlock(&ssa, idx);
  4 
  5 /* . . . */
  6 
  7 idx = srcu_read_lock(&ssb);
  8 synchronize_srcu(&ssa);
  9 srcu_read_unlock(&ssb, idx);
\end{verbatim}
}
\caption{Multistage SRCU Deadlocks}
\label{fig:defer:Multistage SRCU Deadlocks}
\end{figure}


\QuickQ{Why doesn't \url{list_del_rcu()} poison both the \url{next}
and \url{prev} pointers?}

Poisoning the \url{next} pointer would interfere
with concurrent RCU readers, who must use this pointer.
However, RCU readers are forbidden from using the \url{prev}
pointer, so it may safely be poisoned.

\QuickQ{Normally, any pointer subject to \url{rcu_dereference()} \emph{must}
always be updated using \url{rcu_assign_pointer()}.
What is an exception to this rule?}

One such exception is when a multi-element linked
data structure is initialized as a unit while inaccessible to other
CPUs, and then a single \url{rcu_assign_pointer()} is used
to plant a global pointer to this data structure.
The initialization-time pointer assignments need not use
\url{rcu_assign_pointer()}, though any such assignments that
happen after the structure is globally visible \url{must} use
\url{rcu_assign_pointer()}.
\\
However, unless this initialization code is on an impressively hot
code-path, it is probably wise to use \url{rcu_assign_pointer()}
anyway, even though it is in theory unnecessary.
It is all too easy for a "minor" change to invalidate your cherished
assumptions about the initialization happening privately.

\QuickQ{Are there any downsides to the fact that these traversal and update
primitives can be used with any of the RCU API family members?}

It can sometimes be difficult for automated
code checkers such as ``sparse'' (or indeed for human beings) to
work out which type of RCU read-side critical section a given
RCU traversal primitive corresponds to.
For example, consider the code shown in
Figure~\ref{fig:defer:Diverse RCU Read-Side Nesting}.

\begin{figure}[htbp]
{ \centering
\begin{verbatim}
  1 rcu_read_lock();
  2 preempt_disable();
  3 p = rcu_dereference(global_pointer);
  4 
  5 /* . . . */
  6 
  7 preempt_enable();
  8 rcu_read_unlock();
\end{verbatim}
}
\caption{Diverse RCU Read-Side Nesting}
\label{fig:defer:Diverse RCU Read-Side Nesting}
\end{figure}

Is the \url{rcu_dereference()} primitive in an RCU Classic
or an RCU Sched critical section?
What would you have to do to figure this out?

\QuickQAC{sec:advsync:Advanced Synchronization}
\QuickQ{How on earth could the assertion on line~21 of the code in
	Figure~\ref{fig:advsync:Parallel Hardware is Non-Causal} on
	page~\pageref{fig:advsync:Parallel Hardware is Non-Causal}
	\emph{possibly} fail???}

	The key point is that the intuitive analysis missed is that
	there is nothing preventing the assignment to C from overtaking
	the assignment to A as both race to reach {\tt thread2()}.
	This is explained in the remainder of this section.

\QuickQ{Great...  So how do I fix it?}

	The easiest fix is to replace the \url{barrier()} on
	line~12 with an \url{smp_mb()}.

\QuickQ{What assumption is the code fragment
	   in Figure~\ref{fig:advsync:Software Logic Analyzer}
	   making that might not be valid on real hardware?}

	   The code assumes that as soon as a given CPU stops
	   seeing its own value, it will immediately see the
	   final agreed-upon value.
	   On real hardware, some of the CPUs might well see several
	   intermediate results before converging on the final value.

\QuickQ{How could CPUs possibly have different views of the
	   value of a single variable \emph{at the same time?}}

	   Many CPUs have write buffers that record the values of
	   recent writes, which are applied once the corresponding
	   cache line makes its way to the CPU.
	   Therefore, it is quite possible for each CPU to see a
	   different value for a given variable at a single point
	   in time --- and for main memory to hold yet another value.
	   One of the reasons that memory barriers were invented was
	   to allow software to deal gracefully with situations like
	   this one.

\QuickQ{Why do CPUs~2 and 3 come to agreement so quickly, when it
	   takes so long for CPUs~1 and 4 to come to the party?}

	   CPUs~2 and 3 are a pair of hardware threads on the same
	   core, sharing the same cache hierarchy, and therefore have
	   very low communications latencies.
	   This is a NUMA, or, more accurately, a NUCA effect.

	   This leads to the question of why CPUs~2 and 3 ever disagree
	   at all.
	   One possible reason is that they each might have a small amount
	   of private cache in addition to a larger shared cache.
	   Another possible reason is instruction reordering, given the
	   short 10-nanosecond duration of the disagreement and the
	   total lack of memory barriers in the code fragment.

\QuickQ{But if the memory barriers do not unconditionally force
	ordering, how the heck can a device driver reliably execute
	sequences of loads and stores to MMIO registers???}

	MMIO registers are special cases: because they appear
	in uncached regions of physical memory.
	Memory barriers \emph{do} unconditionally force ordering
	of loads and stores to uncached memory.
	See Section~@@@ for more information on memory barriers
	and MMIO regions.

\QuickQ{How could the assertion {\tt b==2} on
	page~\pageref{codesample:advsync:What Can You Count On? 1}
	possibly fail?}

	If the CPU is not required to see all of its loads and
	stores in order, then the {\tt b=1+a} might well see an
	old version of the variable ``a''.
	
	This is why it is so very important that each CPU or thread
	see all of its own loads and stores in program order.

\QuickQ{How could the code on 
	page~\pageref{codesample:advsync:What Can You Count On? 2}
	possibly leak memory?}

	Only the first execution of the critical section should
	see {\tt p==NULL}.
	However, if there is no global ordering of critical sections for
	{\tt mylock}, then how can you say that a particular one was
	first?
	If several different executions of that critical section thought
	that they were first, they would all see {\tt p==NULL}, and
	they would all allocate memory.
	All but one of those allocations would be leaked.
	
	This is why it is so very important that all the critical sections
	for a given exclusive lock appear to execute in some well-defined
	order.

\QuickQ{How could the code on 
	page~\pageref{codesample:advsync:What Can You Count On? 2}
	possibly count backwards?}

	Suppose that the counter started out with the value zero,
	and that three executions of the critical section had therefore
	brought its value to three.
	If the fourth execution of the critical section is not constrained
	to see the most recent store to this variable, it might well see
	the original value of zero, and therefore set the counter to
	one, which would be going backwards.
	
	This is why it is so very important that loads from a given variable
	in a given critical
	section see the last store from the last prior critical section to
	store to that variable.

\QuickQ{What effect does the following sequence have on the
	order of stores to variables ``a'' and ``b''? \\
	{\tt ~~~~a = 1;} \\
	{\tt ~~~~b = 1;} \\
	{\tt ~~~~<write barrier>}}

	Absolutely none.  This barrier {\em would} ensure that the
	assignments to ``a'' and ``b'' happened before any subsequent
	assignments, but it does nothing to enforce any order of
	assignments to ``a'' and ``b'' themselves.

\QuickQ{What sequence of LOCK-UNLOCK operations \emph{would}
	act as a full memory barrier?}

	A series of two back-to-back LOCK-UNLOCK operations, or, somewhat
	less conventionally, an UNLOCK operations followed by a LOCK
	operation.

\QuickQ{What (if any) CPUs have memory-barrier instructions
	from which these semi-permiable locking primitives might
	be constructed?}

	Itanium is one example.
	The identification of any others is left as an
	exercise for the reader.

\QuickQ{Given that operations grouped in curly braces are executed
	concurrently, which of the rows of
	Table~\ref{tab:advsync:Lock-Based Critical Sections}
	are legitimate reorderings of the assignments to variables
	``A'' through ``F'' and the LOCK/UNLOCK operations?
	(The order in the code is A, B, LOCK, C, D, UNLOCK, E, F.)
	Why or why not?}

	\begin{enumerate}
	\item	Legitimate, executed in order.
	\item	Legitimate, the lock acquisition was executed concurrently
		with the last assignment preceding the critical section.
	\item	Illegitimate, the assignment to ``F'' must follow the LOCK
		operation.
	\item	Illegitimate, the LOCK must complete before any operation in
		the critical section.  However, the UNLOCK may legitimately
		be executed concurrently with subsequent operations.
	\item	Legitimate, the assignment to ``A'' precedes the UNLOCK,
		as required, and all other operations are in order.
	\item	Illegitimate, the assignment to ``C'' must follow the LOCK.
	\item	Illegitimate, the assignment to ``D'' must precede the UNLOCK.
	\item	Legitimate, all assignments are ordered with respect to the
		LOCK and UNLOCK operations.
	\item	Illegitimate, the assignment to ``A'' must precede the UNLOCK.
	\end{enumerate}

\QuickQ{What are the constraints for 
	Table~\ref{tab:advsync:Lock-Based Critical Sections}?}

	They are as follows:
	\begin{enumerate}
	\item	LOCK M must precede B, C, and D.
	\item	UNLOCK M must follow A, B, and C.
	\item	LOCK Q must precede F, G, and H.
	\item	UNLOCK Q must follow E, F, and G.
	\end{enumerate}

\QuickQAC{chp:Ease of Use}
\QuickQ{Can a similar algorithm be used when deleting elements?}

	Yes.
	However, since each thread must hold the locks of three
	consecutive elements to delete the middle one, if there
	are $N$ threads, there must be $2N+1$ elements (rather than
	just $N+1$ in order to avoid deadlock.

\QuickQ{Yetch!!!
	What ever possessed someone to come up with an algorithm
	that deserves to be shaved as much as this one does???}

	That would be Paul.

	He was considering the \emph{Dining Philosopher's Problem}, which
	involves a rather unsanitary spaghetti dinner attended by
	five philosphers.
	Given that there are five plates and but five forks on the table, and
	given that each philosopher requires two forks at a time to eat,
	one is supposed to come up with a fork-allocation algorithm that
	avoids deadlock.
	Paul's response was ``Sheesh!!!  Just get five more forks!!!''.

	This in itself was OK, but Paul then applied this same solution to
	circular linked lists.

	This would not have been so bad either, but he had to go and tell
	someone about it!!!

\QuickQ{Give an exception to this rule.}

	One exception would be a difficult and complex algorithm that
	was the only one known to work in a given situation.
	Another exception would be a difficult and complex algorithm
	that was nonetheless the simplest of the set known to work in
	a given situation.
	However, even in these cases, it may be very worthwhile to spend
	a little time trying to come up with a simpler algorithm!
	After all, if you managed to invent the first algorithm
	to do some task, it shouldn't be that hard to go on to
	invent a simpler one.

\QuickQAC{cha:app:Important Questions}
\QuickQ{What SMP coding errors can you see in these examples?
See @@@ for full code.}

(1)	Missing barrier() or volatile on tight loops.
(2)	Missing Memory barriers on update side.
(3)	Lack of synchronization between producer and consumer.

\QuickQ{How could there be such a large gap between successive
consumer reads?
See @@@ for full code.}

(1)	The consumer might be preempted for long time periods.
(2)	A long-running interrupt might delay the consumer.
(3)	The producer might also be running on a faster CPU than is the
	consumer (for example, one of the CPUs might have had to decrease its
	clock frequency due to heat-dissipation or power-consumption
	constraints).

\QuickQAC{app:primitives:Synchronization Primitives}
\QuickQ{Give an example of a parallel program that could be written
	   without synchronization primitives.}

	   There are many examples.
	   One of the simplest would be a parametric study using a
	   single independent variable.
	   If the program {\tt run\_study} took a single argument,
	   then we could use the following bash script to run two
	   instances in parallel, as might be appropriate on a
	   two-CPU system:

	   { \scriptsize \tt run\_study 1 > 1.out\& run\_study 2 > 2.out; wait}

	   One could of course argue that the bash ampersand operator and
	   the ``wait'' primitive are in fact synchronization primitives.
	   If so, then consider that 
	   this script could be run manually in two separate
	   command windows, so that the only synchronization would be
	   supplied by the user himself or herself.

\QuickQ{What problems could occur if the variable {\tt counter} were
	incremented without the protection of {\tt mutex}?}

	On CPUs with load-store architectures, incrementing {\tt counter}
	might compile into something like the following:

\vspace{5pt}
\begin{minipage}[t]{\columnwidth}
\small 
\begin{verbatim}
LOAD counter,r0
INC r0
STORE r0,counter
\end{verbatim}
\end{minipage} 
\vspace{5pt}

	On such machines, two threads might simultaneously load the
	value of {\tt counter}, each increment it, and each store the
	result.
	The new value of {\tt counter} will then only be one greater
	than before, despite two threads each incrementing it.

\QuickQ{How could you work around the lack of a per-thread-variable
	API on systems that do not provide it?}

	One approach would be to create an array indexed by
	{\tt smp\_thread\_id()}, and another would be to use a hash
	table to map from {\tt smp\_thread\_id()} to an array
	index --- which is in fact what this
	set of APIs does in pthread environments.

	Another approach would be for the parent to allocate a structure
	containing fields for each desired per-thread variable, then
	pass this to the child during thread creation.
	However, this approach can impose large software-engineering
	costs in large systems.
	To see this, imagine if all global variables in a large system
	had to be declared in a single file, regardless of whether or
	not they were C static variables!

\QuickQAC{chp:app:whymb:Why Memory Barriers?}
\QuickQ{What happens if two CPUs attempt to invalidate the
same cache line concurrently?}
One of the CPUs gains access
to the shared bus first,
and that CPU ``wins''.  The other CPU must invalidate its copy of the
cache line and transmit an ``invalidate acknowledge'' message
to the other CPU. \\
Of course, the losing CPU can be expected to immediately issue a
``read invalidate'' transaction, so the winning CPU's victory will
be quite ephemeral.

\QuickQ{When an ``invalidate'' message appears in a
	   large multiprocessor, every CPU must give an ``invalidate
	   acknowledge'' response.  Wouldn't the resulting ``storm''
	   of ``invalidate acknowledge'' responses totally saturate the
	   system bus?}

	It might, if large-scale multiprocessors were in fact implemented
	that way.  Larger multiprocessors, particularly NUMA machines,
	tend to use so-called ``directory-based'' cache-coherence
	protocols to avoid this and other problems.

\QuickQ{If SMP machines are really using message passing
	   anyway, why bother with SMP at all?}

	There has been quite a bit of controversy on this topic over
	the past few decades.  One answer is that the cache-coherence
	protocols are quite simple, and therefore can be implemented
	directly in hardware, gaining bandwidths and latencies
	unattainable by software message passing.  Another answer is that
	the real truth is to be found in economics due to the relative
	prices of large SMP machines and that of clusters of smaller
	SMP machines.  A third answer is that the SMP programming
	model is easier to use than that of distributed systems, but
	a rebuttal might note the appearance of HPC clusters and MPI.
	And so the argument continues.

\QuickQ{How does the hardware handle the delayed transitions
	   described above?}

	Usually by adding additional states, though these additional
	states need not be actually stored with the cache line, due to
	the fact that only a few lines at a time will be transitioning.
	The need to delay transitions is but one issue that results in
	real-world cache coherence protocols being much more complex than
	the over-simplified MESI protocol described in this appendix.
	Hennessy and Patterson's classic introduction to computer
	architecture~\cite{Hennessy95a} covers many of these issues.

\QuickQ{What sequence of operations would put the CPUs' caches
	   all back into the ``invalid'' state?}

	There is no such sequence, at least in absence of special
	``flush my cache'' instructions in the CPU's instruction set.
	Most CPUs do have such instructions.

\QuickQ{Does the guarantee that each CPU sees its own memory accesses
	   in order also guarantee that each user-level thread will see
	   its own memory accesses in order?  Why or why not?}

	No.  Consider the case where a thread migrates from one CPU to
	another, and where the destination CPU perceives the source
	CPU's recent memory operations out of order.  To preserve
	user-mode sanity, kernel hackers must use memory barriers in
	the context-switch path.  However, the locking already required
	to safely do a context switch should automatically provide
	the memory barriers needed to cause the user-level task to see
	its own accesses in order.  That said, if you are designing a
	super-optimized scheduler, either in the kernel or at user level,
	please keep this scenario in mind!

\QuickQ{Could this code be fixed by inserting a memory barrier
between CPU~1's ``while'' and assignment to ``c''?  Why or why not?}

No.  Such a memory barrier would only force ordering local to CPU~1.
It would have no effect on the relative ordering of CPU~0's and
CPU~1's accesses, so the assertion could still fire.

\QuickQ{Suppose that lines~3-5 for CPUs~1 and 2 are in an interrupt
handler, and that the CPU~2's line~9 is run at process level.
What changes, if any, are required to enable the code to work
correctly, in other words, to prevent the assertion from firing?}

The assertion will need to coded so as to ensure that the load of
``e'' precedes that of ``a''.
In the Linux kernel, the barrier() primitive may be used to accomplish
this in much the same way that the memory barrier was used in the
assertions in the previous examples.

\QuickQAC{app:rcuimpl:Read-Copy Update Implementations}
\QuickQ{What are some of the shortcomings of the lock-based
	implementation shown in
	Figure~\ref{fig:app:rcuimpl:Lock-Based RCU Implementation}?}

	There are a number of serious shortcomings:
	\begin{enumerate}
	\item	The lock operations in \url{rcu_read_lock()} and
		\url{rcu_read_unlock()} are extremely heavyweight.
	\item	These same lock operations permit \url{rcu_read_lock()}
		to participate in deadlock cycles.
	\item	Only a single thread can be in an RCU read-side
		critical section at a time, which negates RCU's
		scalability advantages.
		This could be overcome by using reader-writer
		locking~\cite{PaulMcKenney2005e}, but only for
		unusually long RCU read-side critical sections.
	\item	RCU read-side critical sections cannot be nested.
	\item	Although concurrent RCU updates could in principle be
		satisfied by a common grace period, this implementation
		serializes grace periods, preventing grace-period
		sharing.
	\end{enumerate}
	These shortcomings are problematic for all the reasons
	discussed in Chapter~\ref{chp:Introduction}.
	It is hard to imagine this implementation being useful
	in a production setting, though it does have the virtue
	of being implementable in user-level applications.

\QuickQ{What are some of the shortcomings of the counter-based
	implementation shown in
	Figure~\ref{fig:app:rcuimpl:RCU Implementation Using Single Global Reference Counter}?}

	There are still some serious shortcomings:
	\begin{enumerate}
	\item	The atomic operations in \url{rcu_read_lock()} and
		\url{rcu_read_unlock()} are still quite  heavyweight.
		This means that the RCU read-side critical sections
		have to be quite long in order to get any real
		read-side parallelism.
	\item	If there are many concurrent \url{rcu_read_lock()}
		and \url{rcu_read_unlock()} operations, there will
		be extreme memory contention on \url{rcu_refcnt},
		resulting in expensive cache misses.
		This further extends the RCU read-side critical-section
		duration required to provide parallel read-side access.
	\item	A large number of RCU readers with long read-side
		critical sections could prevent \url{synchronize_rcu()}
		from ever completing, as the global counter might
		never reach zero.
		This could result in starvation of RCU updates, which
		is unacceptable in production settings.
	\item	Although concurrent RCU updates could in principle be
		satisfied by a common grace period, this implementation
		serializes grace periods, preventing grace-period
		sharing.
	\end{enumerate}
	It is still hard to imagine this implementation being useful
	in a production setting, though it has a bit more potential
	than the lock-based mechanism.

\QuickQ{Why the memory barrier on line~5 of \url{synchronize_rcu()} in
	Figure~\ref{fig:app:rcuimpl:RCU Update Using Global Reference-Count Pair}
	given that there is a spin-lock acquisition immediately after?}

	The spin-lock acquisition only guarantees that the spin-lock's
	critical section will not ``bleed out'' to precede the
	acquisition.
	It in no way guarantees that code preceding the spin-lock
	acquisitoin won't be reordered into the critical section.
	Such reordering could cause a removal from an RCU-protected
	list to be reordered to follow the complementing of
	\url{rcu_idx}, which could allow a newly starting RCU
	read-side critical section to see the recently removed
	data element.

	Exercise for the reader: use a tool such as Promela/spin
	to determine which (if any) of the memory barriers in
	Figure~\ref{fig:app:rcuimpl:RCU Update Using Global Reference-Count Pair}
	are really needed.
	See Section~\ref{app:formal:Formal Verification}
	for information on using these tools.
	The first correct and complete response will be credited.

\QuickQ{Why is the counter flipped twice in
	Figure~\ref{fig:app:rcuimpl:RCU Update Using Global Reference-Count Pair}?
	Shouldn't a single flip-and-wait cycle be sufficient?}

	Both flips are absolutely required.
	To see this, consider the following sequence of events:
	\begin{enumerate}
	\item	Line~8 of \url{rcu_read_lock()} in
		Figure~\ref{fig:app:rcuimpl:RCU Read-Side Using Global Reference-Count Pair}
		picks up \url{rcu_idx}, finding its value to be zero.
	\item	Line~8 of \url{synchronize_rcu()} in
		Figure~\ref{fig:app:rcuimpl:RCU Update Using Global Reference-Count Pair}
		complements the value of \url{rcu_idx}, setting its
		value to one.
	\item	Lines~10-13 of \url{synchronize_rcu()} find that the
		value of \url{rcu_refcnt[0]} is zero, and thus
		returns.
		(Recall that the question is asking what happens if
		lines~14-20 are omitted.)
	\item	Lines~9 and 10 of \url{rcu_read_lock()} store the
		value zero to this thread's instance of \url{rcu_read_idx}
		and increments \url{rcu_refcnt[0]}, respectively.
		Execution then proceeds into the RCU read-side critical
		section.
		\label{app:rcuimpl:rcu_rcgp:RCU Read Side Start}
	\item	Another instance of \url{synchronize_rcu()} again complements
		\url{rcu_idx}, this time setting its value to zero.
		Because \url{rcu_refcnt[1]} is zero, \url{synchronize_rcu()}
		returns immediately.
		(Recall that \url{rcu_read_lock()} incremented
		\url{rcu_refcnt[0]}, not \url{rcu_refcnt[1]}!)
		\label{app:rcuimpl:rcu_rcgp:RCU Grace Period Start}
	\item	The grace period that started in
		step~\ref{app:rcuimpl:rcu_rcgp:RCU Grace Period Start}
		has been allowed to end, despite
		the fact that the RCU read-side critical section
		that started beforehand in
		step~\ref{app:rcuimpl:rcu_rcgp:RCU Read Side Start}
		has not completed.
		This violates RCU semantics, and could allow the update
		to free a data element that the RCU read-side critical
		section was still referencing.
	\end{enumerate}

	Exercise for the reader: What happens if \url{rcu_read_lock()}
	is preempted for a very long time (hours!) just after
	line~8?
	Does this implementation operate correctly in that case?
	Why or why not?
	The first correct and complete response will be credited.

\QuickQ{What are some of the remaining shortcomings of the
	counter-pair-based implementation shown in
	Figures~\ref{fig:app:rcuimpl:RCU Read-Side Using Global Reference-Count Pair}
	and \ref{fig:app:rcuimpl:RCU Update Using Global Reference-Count Pair}?}

	There are still some serious shortcomings:
	\begin{enumerate}
	\item	The atomic operations in \url{rcu_read_lock()} and
		\url{rcu_read_unlock()} are still quite heavyweight,
		in fact, they are more complex than those of the
		single-counter variant shown in
		Figure~\ref{fig:app:rcuimpl:RCU Implementation Using Single Global Reference Counter}.
		This means that the RCU read-side critical sections
		have to be quite long in order to get any real
		read-side parallelism.
	\item	If there are many concurrent \url{rcu_read_lock()}
		and \url{rcu_read_unlock()} operations, there will
		be extreme memory contention on the \url{rcu_refcnt}
		elements, resulting in expensive cache misses.
		This further extends the RCU read-side critical-section
		duration required to provide parallel read-side access.
	\item	The need to flip \url{rcu_idx} twice imposes substantial
		overhead on updates, especially if there are large
		numbers of threads.
	\item	Although concurrent RCU updates could in principle be
		satisfied by a common grace period, this implementation
		serializes grace periods, preventing grace-period
		sharing.
	\end{enumerate}
	Despite these shortcomings, one could imagine this variant
	of RCU being used on small tightly coupled multiprocessors.
	However, it would not not likely scale well beyond a few CPUs.

\QuickQ{Come off it!
	We can see the \url{atomic_read()} primitive in
	\url{rcu_read_lock()}!!!
	So why are you trying to pretend that \url{rcu_read_lock()}
	contains no atomic operations???}

	The \url{atomic_read()} primitives does not actually execute
	atomic machine instructions, but rather does a normal load
	from an \url{atomic_t}.

\QuickQ{Great, if we have $N$ threads, we can have $2N$ ten-millisecond
	waits (one set per \url{flip_counter_and_wait()} invocation,
	and even that assumes that we wait only once for each thread.
	Don't we need the grace period to complete \emph{much} more quickly?}

	Keep in mind that we only wait for a given thread if that thread
	is still in a pre-existing RCU read-side critical section,
	and that waiting for one hold-out thread gives all the other
	threads a chance to complete any pre-existing RCU read-side
	critical sections that they might still be executing.
	So the only way that we would wait for $2N$ intervals
	would be if the last thread still remained in a pre-existing
	RCU read-side critical section despite all the waiting for
	all the prior threads.
	In short, this implementation will not wait unnecessarily.

	However, if you are stress-testing code that uses RCU, you
	might want to comment out the \url{poll()} statement in
	order to better catch bugs that incorrectly retain a reference
	to an RCU-protected data element outside of an RCU
	read-side critical section.

\QuickQ{What are some of the remaining shortcomings of the
	counter-pair-based implementation shown in
	Figures~\ref{fig:app:rcuimpl:RCU Read-Side Using Per-Thread Reference-Count Pair}
	and
	\ref{fig:app:rcuimpl:RCU Update Using Per-Thread Reference-Count Pair}?}

	There are still a few serious shortcomings:
	\begin{enumerate}
	\item	The need to flip \url{rcu_idx} twice imposes substantial
		overhead on updates, especially if there are large
		numbers of threads.
	\item	Although concurrent RCU updates could in principle be
		satisfied by a common grace period, this implementation
		serializes grace periods, preventing grace-period
		sharing.
	\end{enumerate}
	Despite these shortcomings, one could imagine this variant
	of RCU being used on small tightly coupled multiprocessors.
	However, it has a serious update-side bottleneck.

\QuickQ{What are some of the remaining shortcomings of the
	parallel-update counter-pair-based implementation shown in
	Figure~\ref{fig:app:rcuimpl:RCU Read-Side Using Per-Thread Reference-Count Pair and Shared Update}
	and
	\ref{fig:app:rcuimpl:RCU Shared Update Using Per-Thread Reference-Count Pair}?}

	There are still a few shortcomings:
	\begin{enumerate}
	\item	The need to flip \url{rcu_idx} twice imposes substantial
		overhead on updates, especially if there are large
		numbers of threads.
		% Point to user-level RCU implementation after LCA2009.
	\item	Each updater still acquires \url{rcu_gp_lock}, even if there
		is no work to be done.
		This can result in a severe scalability limitation
		if there are large numbers of concurrent updates.
		Appendix~\ref{app:rcuimpl:Preemptable RCU} shows
		one way to avoid this in a production-quality real-time
		implementation of RCU for the Linux kernel.
	\item	On 32-bit machines, a given update thread might be
		preempted long enough for the \url{rcu_idx}
		counter to overflow.
		This could cause such a thread to force an unnecessary
		pair of counter flips.
		However, even if each grace period took only one
		microsecond, the offending thread would need to be
		preempted for more than an hour, in which case an
		extra pair of counter flips is likely the least of
		your worries.
	\end{enumerate}
	Despite these shortcomings, one could imagine this variant
	of RCU being used in real life.

\QuickQ{Why is sleeping prohibited within Classic RCU read-side
critical sections?}

Because sleeping implies a context switch, which in Classic RCU is
a quiescent state, and RCU's grace-period detection requires that
quiescent states never appear in RCU read-side critical sections.

\QuickQ{Why not permit sleeping in Classic RCU read-side critical sections
by eliminating context switch as a quiescent state, leaving user-mode
execution and idle loop as the remaining quiescent states?}

This would mean that a system undergoing heavy kernel-mode execution load
(e.g., due to kernel threads) might never complete a grace period, which
would cause it to exhaust memory sooner or later.

\QuickQ{Why is it OK to assume that updates separated by
	{\tt synchronize\_sched()} will be performed in order?}

	Because this property is required for the {\tt synchronize\_sched()}
	aspect of RCU to work at all.
	For example, consider a code sequence that removes an object
	from a list, invokes {\tt synchronize\_sched()}, then frees
	the object.
	If this property did not hold, then that object might appear
	to be freed before it was
	removed from the list, which is precisely the situation that
	{\tt synchronize\_sched()} is supposed to prevent!

\QuickQ{Why must line~17 in {\tt synchronize\_srcu()}
	(Figure~\ref{fig:app:rcuimpl:Update-Side Implementation})
	precede the release of the mutex on line~18?
	What would have to change to permit these two lines to be
	interchanged?
	Would such a change be worthwhile?
	Why or why not?}

	Suppose that the order was reversed, and that CPU~0
	has just reached line~13 of
	{\tt synchronize\_srcu()}, while both CPU~1 and CPU~2 start executing
	another {\tt synchronize\_srcu()} each, and CPU~3 starts executing a
	{\tt srcu\_read\_lock()}.
	Suppose that CPU~1 reaches line~6 of {\tt synchronize\_srcu()}
	just before CPU~0 increments the counter on line~13.
	Most importantly, suppose that
	CPU~3 executes {\tt srcu\_read\_lock()}
	out of order with the following SRCU read-side critical section,
	so that it acquires a reference to some SRCU-protected data
	structure \emph{before} CPU~0 increments {\tt sp->completed}, but
	executes the {\tt srcu\_read\_lock()} \emph{after} CPU~0 does
	this increment.
	
	Then CPU~0 will \emph{not} wait for CPU~3 to complete its
	SRCU read-side critical section before exiting the ``while''
	loop on lines~15-16 and releasing the mutex (remember, the
	CPU could be reordering the code).
	
	Now suppose that CPU~2 acquires the mutex next,
	and again increments {\tt sp->completed}.
	This CPU will then have to wait for CPU~3 to exit its SRCU
	read-side critical section before exiting the loop on
	lines~15-16 and releasing the mutex.
	But suppose that CPU~3 again executes out of order,
	completing the {\tt srcu\_read\_unlock()} prior to
	executing a final reference to the pointer it obtained
	when entering the SRCU read-side critical section.

	CPU~1 will then acquire the mutex, but see that the
	{\tt sp->completed} counter has incremented twice, and
	therefore take the early exit.
	The caller might well free up the element that CPU~3 is
	still referencing (due to CPU~3's out-of-order execution).

	To prevent this perhaps improbable, but entirely possible,
	scenario, the final {\tt synchronize\_sched()} must precede
	the mutex release in {\tt synchronize\_srcu()}.

	Another approach would be to change to comparison on
	line~7 of {\tt synchronize\_srcu()} to check for at
	least three increments of the counter.
	However, such a change would increase the latency of a
	``bulk update'' scenario, where a hash table is being updated
	or unloaded using multiple threads.
	In the current code, the latency of the resulting concurrent
	{\tt synchronize\_srcu()} calls would take at most two SRCU
	grace periods, while with this change, three would be required.

	More experience will be required to determine which approach
	is really better.
	For one thing, there must first be some use of SRCU with
	multiple concurrent updaters.

\QuickQ{Why is it important that blocking primitives
	called from within a preemptible-RCU read-side critical section be
	subject to priority inheritance?}

	Because blocked readers stall RCU grace periods,
	which can result in OOM.
	For example, if a reader did a \url{wait_event()} within
	an RCU read-side critical section, and that event never occurred,
	then RCU grace periods would stall indefinitely, guaranteeing that
	the system would OOM sooner or later.
	There must therefore be some way to cause these readers to progress
	through their read-side critical sections in order to avoid such OOMs.
	Priority boosting is one way to force such progress, but only if
	readers are restricted to blocking such that they can be awakened via
	priority boosting.

	Of course, there are other methods besides priority inheritance
	that handle the priority inversion problem, including priority ceiling,
	preemption disabling, and so on.
	However, there are good reasons why priority inheritance is the approach
	used in the Linux kernel, so this is what is used for RCU.

\QuickQ{Could the prohibition against using primitives
	that would block in a non-\url{CONFIG_PREEMPT} kernel be lifted,
	and if so, under what conditions?}

	If testing and benchmarking demonstrated that the
	preemptible RCU worked well enough that classic RCU could be dispensed
	with entirely, and if priority inheritance was implemented for blocking
	synchronization primitives
	such as \url{semaphore}s, then those primitives could be
	used in RCU read-side critical sections.

\QuickQ{How is it possible for lines~38-43 of
	\url{__rcu_advance_callbacks()} to be executed when
	lines~7-37 have not?
	Won't they both be executed just after a counter flip, and
	never at any other time?}

Consider the following sequence of events:
\begin{enumerate}
\item	CPU 0 executes lines~5-12 of
	\url{rcu_try_flip_idle()}.
\item	CPU 1 executes \url{__rcu_advance_callbacks()}.
	Because \url{rcu_ctrlblk.completed} has been
	incremented, lines~7-37 execute.
	However, none of the \url{rcu_flip_flag} variables
	have been set, so lines~38-43 do \emph{not} execute.
\item	CPU 0 executes lines~13-15 of
	\url{rcu_try_flip_idle()}.
\item	Later, CPU 1 again executes \url{__rcu_advance_callbacks()}.
	The counter has not been incremented since the earlier
	execution, but the \url{rcu_flip_flag} variables have
	all been set, so only lines~38-43 are executed.
\end{enumerate}

\QuickQ{What problems could arise if the lines containing
	\url{ACCESS_ONCE()} in \url{rcu_read_unlock()}
	were reordered by the compiler?}

\begin{enumerate}
\item	If the \url{ACCESS_ONCE()} were omitted from the
	fetch of \url{rcu_flipctr_idx} (line~14), then the compiler
	would be within its rights to eliminate \url{idx}.
	It would also be free to compile the \url{rcu_flipctr}
	decrement as a fetch-increment-store sequence, separately fetching
	\url{rcu_flipctr_idx} for both the fetch and the store.
	If an NMI were to occur between the fetch and the store, and
	if the NMI handler contained an \url{rcu_read_lock()},
	then the value of \url{rcu_flipctr_idx} would change
	in the meantime, resulting in corruption of the
	\url{rcu_flipctr} values, destroying the ability
	to correctly identify grace periods.
\item	Another failure that could result from omitting the
	\url{ACCESS_ONCE()} from line~14 is due to
	the compiler reordering this statement to follow the
	decrement of \url{rcu_read_lock_nesting}
	(line~16).
	In this case, if an NMI were to occur between these two
	statements, then any \url{rcu_read_lock()} in the
	NMI handler could corrupt \url{rcu_flipctr_idx},
	causing the wrong \url{rcu_flipctr} to be
	decremented.
	As with the analogous situation in \url{rcu_read_lock()},
	this could result in premature grace-period termination,
	an indefinite grace period, or even both.
\item	If \url{ACCESS_ONCE()} macros were omitted such that
	the update of \url{rcu_read_lock_nesting} could be
	interchanged by the compiler with the decrement of
	\url{rcu_flipctr}, and if an NMI occurred in between,
	any \url{rcu_read_lock()} in the NMI handler would
	incorrectly conclude that it was protected by an enclosing
	\url{rcu_read_lock()}, and fail to increment the
	\url{rcu_flipctr} variables.
\end{enumerate}

It is not clear that the \url{ACCESS_ONCE()} on the
fetch of \url{rcu_read_lock_nesting} (line~7) is required.

\QuickQ{What problems could arise if the lines containing
	\url{ACCESS_ONCE()} in \url{rcu_read_unlock()}
	were reordered by the CPU?}

	Absolutely none!!!  The code in \url{rcu_read_unlock()}
	interacts with the scheduling-clock interrupt handler
	running on the same CPU, and is thus insensitive to reorderings
	because CPUs always see their own accesses as if they occurred
	in program order.
	Other CPUs do access the \url{rcu_flipctr}, but because these
	other CPUs don't access any of the other variables, ordering is
	irrelevant.

\QuickQ{What problems could arise in
	\url{rcu_read_unlock()} if irqs were not disabled?}

\begin{enumerate}
\item	Disabling irqs has the side effect of disabling preemption.
	Suppose that this code were to be preempted in the midst
	of line~17 between selecting the current CPU's copy
	of the \url{rcu_flipctr} array and the decrement of
	the element indicated by \url{rcu_flipctr_idx}.
	Execution might well resume on some other CPU.
	If this resumption happened concurrently with an
	\url{rcu_read_lock()} or \url{rcu_read_unlock()}
	running on the original CPU,
	an increment or decrement might be lost, resulting in either
	premature termination of a grace period, indefinite extension
	of a grace period, or even both.
\item	Failing to disable preemption can also defeat RCU priority
	boosting, which relies on \url{rcu_read_lock_nesting}
	to determine which tasks to boost.
	If preemption occurred between the update of
	\url{rcu_read_lock_nesting} (line~16) and of
	\url{rcu_flipctr} (line~17), then a grace
	period might be stalled until this task resumed.
	But because the RCU priority booster has no way of knowing
	that this particular task is stalling grace periods, needed
	boosting will never occur.
	Therefore, if there are CPU-bound realtime tasks running,
	the preempted task might never resume, stalling grace periods
	indefinitely, and eventually resulting in OOM.
\end{enumerate}

Of course, both of these situations could be handled by disabling
preemption rather than disabling irqs.
(The CPUs I have access to do not show much difference between these
two alternatives, but others might.)

\QuickQ{Suppose that the irq disabling in
	\url{rcu_read_lock()} was replaced by preemption disabling.
	What effect would that have on \url{GP_STAGES}?}

No finite value of \url{GP_STAGES} suffices.
The following scenario, courtesy of Oleg Nesterov, demonstrates this:

Suppose that low-priority Task~A has executed
\url{rcu_read_lock()} on CPU 0,
and thus has incremented \url{per_cpu(rcu_flipctr, 0)[0]},
which thus has a value of one.
Suppose further that Task~A is now preempted indefinitely.

Given this situation, consider the following sequence of events:
\begin{enumerate}
\item	Task~B starts executing \url{rcu_read_lock()}, also on
	CPU 0, picking up the low-order bit of
	\url{rcu_ctrlblk.completed}, which is still equal to zero.
\item	Task~B is interrupted by a sufficient number of scheduling-clock
	interrupts to allow the current grace-period stage to complete,
	and also be sufficient long-running interrupts to allow the
	RCU grace-period state machine to advance the
	\url{rcu_ctrlblk.complete} counter so that its bottom bit
	is now equal to one and all CPUs have acknowledged this increment
	operation.
\item	CPU 1 starts summing the index==0 counters, starting with
	\url{per_cpu(rcu_flipctr, 0)[0]}, which is equal to one
	due to Task~A's increment.
	CPU 1's local variable \url{sum} is therefore equal to one.
\item	Task~B returns from interrupt, resuming its execution of
	\url{rcu_read_lock()}, incrementing
	\url{per_cpu(rcu_flipctr, 0)[0]}, which now has a value
	of two.
\item	Task~B is migrated to CPU 2.
\item	Task~B completes its RCU read-side critical section, and executes
	\url{rcu_read_unlock()}, which decrements
	\url{per_cpu(rcu_flipctr, 2)[0]}, which is now -1.
\item	CPU 1 now adds \url{per_cpu(rcu_flipctr, 1)[0]} and 
	\url{per_cpu(rcu_flipctr, 2)[0]} to its
	local variable \url{sum}, obtaining the value zero.
\item	CPU 1 then incorrectly concludes that all prior RCU read-side
	critical sections have completed, and advances to the next
	RCU grace-period stage.
	This means that some other task might well free up data structures
	that Task~A is still using!
\end{enumerate}

This sequence of events could repeat indefinitely, so that no finite
value of \url{GP_STAGES} could prevent disrupting Task~A.
This sequence of events demonstrates the importance of the promise
made by CPUs that acknowledge an increment of
\url{rcu_ctrlblk.completed}, as the problem illustrated by the
above sequence of events is caused by Task~B's repeated failure
to honor this promise.

Therefore, more-pervasive changes to the grace-period state will be
required in order for \url{rcu_read_lock()} to be able to safely
dispense with irq disabling.

\QuickQ{Why can't the \url{rcu_dereference()}
	precede the memory barrier?}

	Because the memory barrier is being executed in
	an interrupt handler, and interrupts are exact in the sense that
	a single value of the PC is saved upon interrupt, so that the
	interrupt occurs at a definite place in the code.
	Therefore, if the
	\url{rcu_dereference()} were to precede the memory barrier,
	the interrupt would have had to have occurred after the
	\url{rcu_dereference()}, and therefore
	the interrupt would also have had to have occurred after the
	\url{rcu_read_lock()} that begins the RCU read-side critical
	section.
	This would have forced the \url{rcu_read_lock()} to use
	the earlier value of the grace-period counter, which would in turn
	have meant that the corresponding \url{rcu_read_unlock()}
	would have had to precede the first "Old counters zero [0]" rather
	than the second one.
	This in turn would have meant that the read-side critical section
	would have been much shorter --- which would have been
	counter-productive,
	given that the point of this exercise was to identify the longest
	possible RCU read-side critical section.

\QuickQ{What is a more precise way to say "CPU~0
	might see CPU~1's increment as early as CPU~1's last previous
	memory barrier"?}

	First, it is important to note that the problem with
	the less-precise statement is that it gives the impression that there
	might be a single global timeline, which there is not, at least not for
	popular microprocessors.
	Second, it is important to note that memory barriers are all about
	perceived ordering, not about time.
	Finally, a more precise way of stating above statement would be as
	follows: "If CPU~0 loads the value resulting from CPU~1's
	increment, then any subsequent load by CPU~0 will see the
	values from any relevant stores by CPU~1 if these stores
	preceded CPU~1's last prior memory barrier."

	Even this more-precise version leaves some wiggle room.
	The word "subsequent" must be understood to mean "ordered after",
	either by an explicit memory barrier or by the CPU's underlying
	memory ordering.
	In addition, the memory barriers must be strong enough to order
	the relevant operations.
	For example, CPU~1's last prior memory barrier must order stores
	(for example, \url{smp_wmb()} or \url{smp_mb()}).
	Similarly, if CPU~0 needs an explicit memory barrier to
	ensure that its later load follows the one that saw the increment,
	then this memory barrier needs to be an \url{smp_rmb()}
	or \url{smp_mb()}.

	In general, much care is required when proving parallel algorithms.

\QuickQAC{app:formal:Formal Verification}
\QuickQ{Why is there an unreached statement in
locker?  After all, isn't this a \emph{full} state-space
search???}

The locker process is an infinite loop, so control
never reaches the end of this process.
However, since there are no monotonically increasing variables,
Promela is able to model this infinite loop with a small
number of states.

\QuickQ{What are some Promela code-style issues with this example?}

There are several:
\begin{enumerate}
\item	The declaration of {\tt sum} should be moved to within
	the init block, since it is not used anywhere else.
\item	The assertion code should be moved outside of the
	initialization loop.  The initialization loop can
	then be placed in an atomic block, greatly reducing
	the state space (by how much?).
\item	The atomic block covering the assertion code should
	be extended to include the initialization of {\tt sum}
	and {\tt j}, and also to cover the assertion.
	This also reduces the state space (again, by how
	much?).
\end{enumerate}

\QuickQ{Is there a more straightforward way to code the do-od statement?}

Yes.  Replace it with {\tt if-fi} and remove the two {\tt break} statements.

\QuickQ{Why are there atomic blocks at lines 12-21
and lines 44-56, when the operations within those atomic
blocks have no atomic implementation on any current
production microprocessor?}

Because those operations are for the benefit of the
assertion only.  They are not part of the algorithm itself.
There is therefore no harm in marking them atomic, and
so marking them greatly reduces the state space that must
be searched by the Promela model.

\QuickQ{Is the re-summing of the counters on lines 24-27
\emph{really} necessary???}

Yes.  To see this, delete these lines and run the model.

Alternatively, consider the following sequence of steps:

\begin{enumerate}
\item	One process is within its RCU read-side critical
	section, so that the value of {\tt ctr[0]} is zero and
	the value of {\tt ctr[1]} is two.
\item	An updater starts executing, and sees that the sum of
	the counters is two so that the fastpath cannot be
	executed.  It therefore acquires the lock.
\item	A second updater starts executing, and fetches the value
	of {\tt ctr[0]}, which is zero.
\item	The first updater adds one to {\tt ctr[0]}, flips
	the index (which now becomes zero), then subtracts
	one from {\tt ctr[1]} (which now becomes one).
\item	The second updater fetches the value of {\tt ctr[1]},
	which is now one.
\item	The second updater now incorrectly concludes that it
	is safe to proceed on the fastpath, despite the fact
	that the original reader has not yet completed.
\end{enumerate}

\QuickQ{Yeah, that's great!!!
	Now, just what am I supposed to do if I don't happen to have a
	machine with 40GB of main memory???}

	Relax, there are a number of lawful answers to
	this question:
	\begin{enumerate}
	\item	Further optimize the model, reducing its memory consumption.
	\item	Work out a pencil-and-paper proof, perhaps starting with the
		comments in the code in the Linux kernel.
	\item	Devise careful torture tests, which, though they cannot prove
		the code correct, can find hidden bugs.
	\item	There is some movement towards tools that do model
		checking on clusters of smaller machines.
		However, please note that we have not actually used such
		tools myself, courtesy of some large machines that Paul has
		occasional access to.
	\end{enumerate}

\QuickQ{Why not simply increment \url{rcu_update_flag}, and then only
	increment \url{dynticks_progress_counter} if the old value
	of \url{rcu_update_flag} was zero???}

	This fails in presence of NMIs.
	To see this, suppose an NMI was received just after
	\url{rcu_irq_enter()} incremented \url{rcu_update_flag},
	but before it incremented \url{dynticks_progress_counter}.
	The instance of \url{rcu_irq_enter()} invoked by the NMI
	would see that the original value of \url{rcu_update_flag}
	was non-zero, and would therefore refrain from incrementing
	\url{dynticks_progress_counter}.
	This would leave the RCU grace-period machinery no clue that the
	NMI handler was executing on this CPU, so that any RCU read-side
	critical sections in the NMI handler would lose their RCU protection.

	The possibility of NMI handlers, which, by definition cannot
	be masked, does complicate this code.

\QuickQ{But if line~7 finds that we are the outermost interrupt,
	wouldn't we \emph{always} need to increment
	\url{dynticks_progress_counter}?}

	Not if we interrupted a running task!
	In that case, \url{dynticks_progress_counter} would
	have already been incremented by \url{rcu_exit_nohz()},
	and there would be no need to increment it again.

\QuickQ{Can you spot any bugs in any of the code in this section?}

	Read the next section to see if you were correct.

\QuickQ{Why isn't the memory barrier in \url{rcu_exit_nohz()}
	and \url{rcu_enter_nohz()} modeled in Promela?}

	Promela assumes sequential consistency, so
	it is not necessary to model memory barriers.
	In fact, one must instead explicitly model lack of memory barriers,
	for example, as shown in
	Figure~\ref{fig:analysis:QRCU Unordered Summation} on
	page~\pageref{fig:analysis:QRCU Unordered Summation}.

\QuickQ{Isn't it a bit strange to model \url{rcu_exit_nohz()}
	followed by \url{rcu_enter_nohz()}?
	Wouldn't it be more natural to instead model entry before exit?}

	It probably would be more natural, but we will need
	this particular order for the liveness checks that we will add later.

\QuickQ{Wait a minute!
	In the Linux kernel, both \url{dynticks_progress_counter} and
	\url{rcu_dyntick_snapshot} are per-CPU variables.
	So why are they instead being modeled as single global variables?}

	Because the grace-period code processes each
	CPU's \url{dynticks_progress_counter} and
	\url{rcu_dyntick_snapshot} variables separately,
	we can collapse the state onto a single CPU.
	If the grace-period code were instead to do something special
	given specific values on specific CPUs, then we would indeed need
	to model multiple CPUs.
	But fortunately, we can safely confine ourselves to two CPUs, the
	one running the grace-period processing and the one entering and
	leaving dynticks-idle mode.

\QuickQ{Given there are a pair of back-to-back changes to
	\url{grace_period_state} on lines~25 and 26,
	how can we be sure that line~25's changes won't be lost?}

	Recall that Promela and spin trace out
	every possible sequence of state changes.
	Therefore, timing is irrelevant: Promela/spin will be quite
	happy to jam the entire rest of the model between those two
	statements unless some state variable specifically prohibits
	doing so.

\QuickQ{But what would you do if you needed the statements in a single
	\url{EXECUTE_MAINLINE()} group to execute non-atomically?}

	The easiest thing to do would be to put
	each such statement in its own \url{EXECUTE_MAINLINE()}
	statement.

\QuickQ{But what if the \url{dynticks_nohz()} process had
	``if'' or ``do'' statements with conditions,
	where the statement bodies of these constructs
	needed to execute non-atomically?}

	One approach, as we will see in a later section,
	is to use explicit labels and ``goto'' statements.
	For example, the construct:

	\vspace{5pt}
	\begin{minipage}[t]{\columnwidth}
	\scriptsize
	\begin{verbatim}
		if
		:: i == 0 -> a = -1;
		:: else -> a = -2;
		fi;
	\end{verbatim}
	\end{minipage}
	\vspace{5pt}

	could be modeled as something like:

	\vspace{5pt}
	\begin{minipage}[t]{\columnwidth}
	\scriptsize
	\begin{verbatim}
		EXECUTE_MAINLINE(stmt1,
				 if
				 :: i == 0 -> goto stmt1_then;
				 :: else -> goto stmt1_else;
				 fi)
		stmt1_then: skip;
		EXECUTE_MAINLINE(stmt1_then1, a = -1; goto stmt1_end)
		stmt1_else: skip;
		EXECUTE_MAINLINE(stmt1_then1, a = -2)
		stmt1_end: skip;
	\end{verbatim}
	\end{minipage}
	\vspace{5pt}

	However, it is not clear that the macro is helping much in the case
	of the ``if'' statement, so these sorts of situations will
	be open-coded in the following sections.

\QuickQ{Why are lines~45 and 46 (the \url{in_dyntick_irq = 0;}
	and the \url{i++;}) executed atomically?}

	These lines of code pertain to controlling the
	model, not to the code being modeled, so there is no reason to
	model them non-atomically.
	The motivation for modeling them atomically is to reduce the size
	of the state space.

\QuickQ{What property of interrupts is this \url{dynticks_irq()}
	process unable to model?}

	One such property is nested interrupts,
	which are handled in the following section.

\QuickQ{Does Paul always write his code in this painfully incremental
	manner???}

	Not always, but more and more frequently.
	In this case, Paul started with the smallest slice of code that included
	an interrupt handler, because he was not sure how best to model interrupts
	in Promela.
	Once he got that working, he added other features.
	(But if he was doing it again, he would start with a ``toy'' handler.
	For example, he might have the handler increment a variable twice and
	have the mainline code verify that the value was always even.)

	Why the incremental approach?
	Consider the following, attributed to Brian W. Kernighan:

	\begin{quote}
		Debugging is twice as hard as writing the code in the first
		place. Therefore, if you write the code as cleverly as possible,
		you are, by definition, not smart enough to debug it.
	\end{quote}

	This means that any attempt to optimize the production of code should
	place at least 66\% of its emphasis on optimizing the debugging process,
	even at the expense of increasing the time and effort spent coding.
	Incremental coding and testing is one way to optimize the debugging
	process, at the expense of some increase in coding effort.
	Paul uses this approach because he rarely has the luxury of
	devoting full days (let alone weeks) to coding and debugging.

\QuickQ{But what happens if an NMI handler starts running before
	an irq handler completes, and if that NMI handler continues
	running until a second irq handler starts?}

	This cannot happen within the confines of a single CPU.
	The first irq handler cannot complete until the NMI handler
	returns.
	Therefore, if each of the \url{dynticks} and \url{dynticks_nmi}
	variables have taken on an even value during a given time
	interval, the corresponding CPU really was in a quiescent
	state at some time during that interval.

\QuickQ{This is still pretty complicated.
	Why not just have a \url{cpumask_t} that has a bit set for
	each CPU that is in dyntick-idle mode, clearing the bit
	when entering an irq or NMI handler, and setting it upon
	exit?}

	Although this approach would be functionally correct, it
	would result in excessive irq entry/exit overhead on
	large machines.
	In contrast, the approach laid out in this section allows
	each CPU to touch only per-CPU data on irq and NMI entry/exit,
	resulting in much lower irq entry/exit overhead, especially
	on large machines.

