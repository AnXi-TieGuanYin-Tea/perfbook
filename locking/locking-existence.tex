% locking-existence.tex

\section{Lock-Based Existence Guarantees}
\label{sec:locking:Lock-Based Existence Guarantees}

\begin{figure}[tbp]
{ \scriptsize
\begin{verbatim}
  1 int delete(int key)
  2 {
  3   int b;
  4   struct element *p;
  5
  6   b = hashfunction(key);
  7   p = hashtable[b];
  8   if (p == NULL || p->key != key)
  9     return 0;
 10   spin_lock(&p->lock);
 11   hashtable[b] = NULL;
 12   spin_unlock(&p->lock);
 13   kfree(p);
 14   return 1;
 15 }
\end{verbatim}
}
\caption{Per-Element Locking Without Existence Guarantees}
\label{fig:locking:Per-Element Locking Without Existence Guarantees}
\end{figure}

A key challenge in parallel programming is to provide
\emph{existence guarantees}~\cite{Gamsa99},
so that attempts to delete an object that others are concurrently
attempting to access are correctly resolved.
Existence guarantees are extremely helpful in cases where a data
element is to be protected by a lock or reference count that is
located within the data element in question.
Code failing to provide such guarantees is subject to subtle races,
as shown in
Figure~\ref{fig:locking:Per-Element Locking Without Existence Guarantees}.

\QuickQuiz{}
	What if the element we need to delete is not the first element
	of the list on line~8 of
	Figure~\ref{fig:locking:Per-Element Locking Without Existence Guarantees}?
\QuickQuizAnswer{
	This is a very simple hash table with no chaining, so the only
	element in a given bucket is the first element.
	The reader is invited to adapt this example to a hash table with
	full chaining.
} \QuickQuizEnd

\QuickQuiz{}
	What race condition can occur in
	Figure~\ref{fig:locking:Per-Element Locking Without Existence Guarantees}?
\QuickQuizAnswer{
	Consider the following sequence of events:
	\begin{enumerate}
	\item	Thread~0 invokes \url{delete(0)}, and reaches line~10 of
		the figure, acquiring the lock.
	\item	Thread~1 concurrently invokes \url{delete(0)}, and reaches
		line~10, but spins on the lock because Thread~1 holds it.
	\item	Thread~0 executes lines~11-14, removing the element from
		the hashtable, releasing the lock, and then freeing the
		element.
	\item	Thread~0 continues execution, and allocates memory, getting
		the exact block of memory that it just freed.
	\item	Thread~0 then initializes this block of memory as some
		other type of structure.
	\item	Thread~1's \url{spin_lock()} operation fails due to the
		fact that what it believes to be \url{p->lock} is no longer
		a spinlock.
	\end{enumerate}
	Because there is no existence guarantee, the identity of the
	data element can change while a thread is attempting to acquire
	that element's lock on line~10!
} \QuickQuizEnd

\begin{figure}[tbp]
{ \scriptsize
\begin{verbatim}
  1 int delete(int key)
  2 {
  3   int b;
  4   struct element *p;
  5   spinlock_t *sp;
  6
  7   b = hashfunction(key);
  8   sp = &locktable[b];
  9   spin_lock(sp);
 10   p = hashtable[b];
 11   if (p == NULL || p->key != key) {
 12     spin_unlock(sp);
 13     return 0;
 14   }
 15   hashtable[b] = NULL;
 16   spin_unlock(sp);
 17   kfree(p);
 18   return 1;
 19 }
\end{verbatim}
}
\caption{Per-Element Locking With Lock-Based Existence Guarantees}
\label{fig:locking:Per-Element Locking With Lock-Based Existence Guarantees}
\end{figure}

One way to fix this example is to use a hashed set of global locks, so
that each hash bucket has its own lock, as shown in
Figure~\ref{fig:locking:Per-Element Locking With Lock-Based Existence Guarantees}.
This approach allows acquiring the proper lock (on line~9) before
gaining a pointer to the data element (on line~10).
Although this approach works quite well for elements contained in a
single partitionable data structure such as the hash table shown in the
figure, it can be problematic if a given data element can be a member
of multiple hash tables or given more-complex data structures such
as trees or graphs.
These problems can be solved, in fact, such solutions form the basis
of lock-based software transactional memory
implementations~\cite{Shavit95,DaveDice2006DISC}.
However,
Section~\ref{sec:defer:RCU is a Way of Providing Existence Guarantees}
describes a simpler way of providing existence guarantees for read-mostly
data structures.
