% appendix/primitives/primitives.tex

\chapter{Synchronization Primitives}
\label{app:primitives:Synchronization Primitives}

\QuickQuizChapter{app:primitives:Synchronization Primitives}

All but the simplest parallel programs require synchronization
primitives.
This appendix gives a quick overview of a set of primitives based
loosely on those in the Linux kernel.

Why Linux?
Because it is one of the well-known, largest, and easily obtained
bodies of parallel code available.
We believe that reading code is, if anything, more important to learning 
than is writing code, so by using examples similar to real code in
the Linux kernel, we are enabling you to use Linux to continue your
learning as you progress beyond the confines of this book.

Why based loosely rather than following the Linux kernel API exactly?
First, the Linux API changes with time, so any attempt to track it
exactly would likely end in total frustration for all involved.
Second, many of the members of the Linux kernel API are specialized for
use in a production-quality operating-system kernel.
This specialization introduces complexities that, though absolutely
necessary in the Linux kernel itself, are often more trouble than they are
worth in the ``toy'' programs that we will be using to demonstrate
SMP and realtime design principles and practices.
For example, properly checking for error conditions such as memory
exhaustion is a ``must'' in the Linux kernel, however, in ``toy'' programs
it is perfectly acceptable to simply \url{abort()} the program,
correct the problem, and rerun.

Finally, it should be possible to implement a trivial mapping layer between
this API and most production-level APIs.
A pthreads implementation is available \emph{(@@@ where?)}, and
a Linux-kernel-module API would not be difficult to create.

\QuickQuiz{}
	Give an example of a parallel program that could be written
	without synchronization primitives.
\QuickQuizAnswer{
	There are many examples.
	One of the simplest would be a parametric study using a
	single independent variable.
	If the program {\tt run\_study} took a single argument,
	then we could use the following bash script to run two
	instances in parallel, as might be appropriate on a
	two-CPU system:

	{ \scriptsize \tt run\_study 1 > 1.out\& run\_study 2 > 2.out; wait}

	One could of course argue that the bash ampersand operator and
	the ``wait'' primitive are in fact synchronization primitives.
	If so, then consider that 
	this script could be run manually in two separate
	command windows, so that the only synchronization would be
	supplied by the user himself or herself.
} \QuickQuizEnd

The following sections describe commonly used classes of synchronization
primitives.
@@@ More esoteric primitives will be introduced in later revision.

Section~\ref{app:primitives:Organization and Initialization}
covers organization/initialization primitives;
Section~\ref{app:primitives:Thread Creation, Destruction, and Control}
presents thread creation, destruction, and control primitives;
Section~\ref{app:primitives:Locking}
presents locking primitives;
Section~\ref{app:primitives:Per-Thread Variables}
presents per-thread and per-CPU variable primitives; and
Section~\ref{app:primitives:Performance}
gives an overview of the relative performance of the various primitives.

\section{Organization and Initialization}
\label{app:primitives:Organization and Initialization}

\emph{@@@ currently include ../api.h, and there is only pthreads.
Expand and complete once the CodeSamples structure settles down.}

\subsection{smp\_init():}
You must invoke \url{smp_init()} before invoking any other primitives.

\section{Thread Creation, Destruction, and Control}
\label{app:primitives:Thread Creation, Destruction, and Control}

This API focuses on ``threads'', which are a locus of control.\footnote{
	There are many other names for similar software constructs, including
	``process'', ``task'', ``fiber'', ``event'', and so on.
	Similar design principles apply to all of them.}
Each such thread has an identifier of type \url{thread_id_t},
and no two threads running at a given time will have the same
identifier.
Threads share everything except for per-thread local state,\footnote{
	How is that for a circular definition?}
which includes program counter and stack.

The thread API is shown in
Figure~\ref{fig:intro:Thread API}, and members are described in the
following sections.

\begin{figure*}[htbp] 
{ \scriptsize
\begin{verbatim}
int smp_thread_id(void)
thread_id_t create_thread(void *(*func)(void *), void *arg)
for_each_thread(t)
for_each_running_thread(t)
void *wait_thread(thread_id_t tid)
void wait_all_threads(void)
\end{verbatim}
}
\caption{Thread API}
\label{fig:intro:Thread API}
\end{figure*}

\subsection{create\_thread()}

The \url{create_thread} primitive creates a new thread,
starting the new thread's execution
at the function \url{func} specified by \url{create_thread()}'s
first argument, and passing it the argument specified by
\url{create_thread()}'s second argument.
This newly created thread will terminate when it returns from the
starting function specified by \url{func}.
The \url{create_thread()} primitive returns the \url{thread_id_t}
corresponding to the newly created child thread.

This primitive will abort the program if more than \url{NR_THREADS}
threads are created, counting the one implicitly created by running
the program.
\url{NR_THREADS} is a compile-time constant that may be modified,
though some systems may have an upper bound for the allowable number
of threads.

\subsection{smp\_thread\_id()}

Because the \url{thread_id_t} returned from \url{create_thread()} is
system-dependent, the \url{smp_thread_id()} primitive returns a thread
index corresponding to the thread making the request.
This index is guaranteed to be less than the maximum number of threads
that have been in existence since the program started,
and is therefore useful for bitmasks, array indices, and
the like.

\subsection{for\_each\_thread()}

The \url{for_each_thread()} macro loops through all threads that exist,
including all threads that \emph{would} exist if created.
This macro is useful for handling per-thread variables as will be
seen in Section~\ref{app:primitives:Per-Thread Variables}.

\subsection{for\_each\_running\_thread()}

The \url{for_each_running_thread()}
macro loops through only those threads that currently exist.
It is the caller's responsibility to synchronize with thread
creation and deletion if required.

\subsection{wait\_thread()}

The \url{wait_thread()} primitive waits for completion of the thread
specified by the \url{thread_id_t} passed to it.
This in no way interferes with the execution of the specified thread;
instead, it merely waits for it.
Note that \url{wait_thread()} returns the value that was returned by
the corresponding thread.

\subsection{wait\_all\_threads()}

The \url{wait_all_thread()}
primitive waits for completion of all currently running threads.
It is the caller's responsibility to synchronize with thread creation
and deletion if required.
However, this primitive is normally used to clean up and the end of
a run, so such synchronization is normally not needed.

\subsection{Example Usage}

Figure~\ref{fig:intro:Example Child Thread}
shows an example hello-world-like child thread.
As noted earlier, each thread is allocated its own stack, so
each thread has its own private \url{arg} argument and \url{myarg} variable.
Each child simply prints its argument and its \url{smp_thread_id()}
before exiting.
Note that the \url{return} statement on line~7 terminates the thread,
returning a \url{NULL} to whoever invokes \url{wait_thread()} on this
thread.

\begin{figure}[htbp] 
{ \scriptsize
\begin{verbatim}
  1 void *thread_test(void *arg)
  2 {
  3   int myarg = (int)arg;
  4 
  5   printf("child thread %d: smp_thread_id() = %d\n",
  6          myarg, smp_thread_id());
  7   return NULL;
  8 }
\end{verbatim}
}
\caption{Example Child Thread}
\label{fig:intro:Example Child Thread}
\end{figure}

The parent program is shown in
Figure~\ref{fig:intro:Example Parent Thread}.
It invokes \url{smp_init()} to initialize the threading system on
line~6,
parse arguments on lines~7-14, and announces its presence on line~15.
It creates the specified number of child threads on lines~16-17,
and waits for them to complete on line~18.
Note that \url{wait_all_threads()} discards the threads return values,
as in this case they are all \url{NULL}, which is not very interesting.

\begin{figure}[htbp] 
{ \scriptsize
\begin{verbatim}
  1 int main(int argc, char *argv[])
  2 {
  3   int i;
  4   int nkids = 1;
  5 
  6   smp_init();
  7   if (argc > 1) {
  8     nkids = strtoul(argv[1], NULL, 0);
  9     if (nkids > NR_THREADS) {
 10       fprintf(stderr, "nkids=%d too big, max=%d\n",
 11         nkids, NR_THREADS);
 12       usage(argv[0]);
 13     }
 14   }
 15   printf("Parent spawning %d threads.\n", nkids);
 16   for (i = 0; i < nkids; i++)
 17     create_thread(thread_test, (void *)i);
 18   wait_all_threads();
 19   printf("All threads completed.\n", nkids);
 20   exit(0);
 21 }
\end{verbatim}
}
\caption{Example Parent Thread}
\label{fig:intro:Example Parent Thread}
\end{figure}

\section{Locking}
\label{app:primitives:Locking}

The locking API is shown in
Figure~\ref{fig:intro:Locking API},
each API element being described in the following sections.

\begin{figure}[htbp] 
{ \scriptsize
\begin{verbatim}
void spin_lock_init(spinlock_t *sp);
void spin_lock(spinlock_t *sp);
int spin_trylock(spinlock_t *sp);
void spin_unlock(spinlock_t *sp);
\end{verbatim}
}
\caption{Locking API}
\label{fig:intro:Locking API}
\end{figure}

\subsection{spin\_lock\_init()}

The \url{spin_lock_init()} primitive initializes the specified
\url{spinlock_t} variable, and must be invoked before
this variable is passed to any other spinlock primitive.

\subsection{spin\_lock()}

The \url{spin_lock()} primitive acquires the specified spinlock,
if necessary, waiting until the spinlock becomes available.
In some environments, such as pthreads, this waiting will involve
``spinning'', while
in others, such as the Linux kernel, it will involve blocking.

The key point is that only one thread may hold a spinlock at any
given time.

\subsection{spin\_trylock()}

The \url{spin_trylock()} primitive acquires the specified spinlock,
but only if it is immediately available.
It returns \url{TRUE} if it was able to acquire the spinlock and
\url{FALSE} otherwise.

\subsection{spin\_unlock()}

The \url{spin_unlock()} primitive releases the specified spinlock,
allowing other threads to acquire it.

\emph{@@@ likely need to add reader-writer locking.}

\subsection{Example Usage}

A spinlock named \url{mutex} may be used to protect a variable
\url{counter} as follows:

\vspace{5pt}
\begin{minipage}[t]{\columnwidth}
\small
\begin{verbatim}
spin_lock(&mutex);
counter++;
spin_unlock(&mutex);
\end{verbatim}
\end{minipage}
\vspace{5pt}

\QuickQuiz{}
	What problems could occur if the variable {\tt counter} were
	incremented without the protection of {\tt mutex}?
\QuickQuizAnswer{
	On CPUs with load-store architectures, incrementing {\tt counter}
	might compile into something like the following:

\vspace{5pt}
\begin{minipage}[t]{\columnwidth}
\small 
\begin{verbatim}
LOAD counter,r0
INC r0
STORE r0,counter
\end{verbatim}
\end{minipage} 
\vspace{5pt}

	On such machines, two threads might simultaneously load the
	value of {\tt counter}, each increment it, and each store the
	result.
	The new value of {\tt counter} will then only be one greater
	than before, despite two threads each incrementing it.
} \QuickQuizEnd

However, the \url{spin_lock()} and \url{spin_unlock()} primitives
do have performance consequences, as will be seen in
Section~\ref{app:primitives:Performance}.

\section{Per-Thread Variables}
\label{app:primitives:Per-Thread Variables}

Figure~\ref{fig:intro:Per-Thread-Variable API}
shows the per-thread-variable API.
This API provides the per-thread equivalent of global variables.
Although this API is, strictly speaking, not necessary, it can
greatly simply coding.

\begin{figure}[htbp] 
{ \scriptsize
\begin{verbatim}
DEFINE_PER_THREAD(type, name)
DECLARE_PER_THREAD(type, name)
per_thread(name, thread)
__get_thread_var(name)
init_per_thread(name, v)
\end{verbatim}
}
\caption{Per-Thread-Variable API}
\label{fig:intro:Per-Thread-Variable API}
\end{figure}

\QuickQuiz{}
	How could you work around the lack of a per-thread-variable
	API on systems that do not provide it?
\QuickQuizAnswer{
	One approach would be to create an array indexed by
	{\tt smp\_thread\_id()}, and another would be to use a hash
	table to map from {\tt smp\_thread\_id()} to an array
	index --- which is in fact what this
	set of APIs does in pthread environments.

	Another approach would be for the parent to allocate a structure
	containing fields for each desired per-thread variable, then
	pass this to the child during thread creation.
	However, this approach can impose large software-engineering
	costs in large systems.
	To see this, imagine if all global variables in a large system
	had to be declared in a single file, regardless of whether or
	not they were C static variables!
} \QuickQuizEnd

\subsection{DEFINE\_PER\_THREAD()}

The \url{DEFINE_PER_THREAD()} primitive defines a per-thread variable.
Unfortunately, it is not possible to provide an initializer in the way
permitted by the Linux kernel's \url{DEFINE_PER_THREAD()} primitive,
but there is an \url{init_per_thread()} primitive that permits easy
runtime initialization.

\subsection{DECLARE\_PER\_THREAD()}

The \url{DECLARE_PER_THREAD()} primitive is a declaration in the C sense,
as opposed to a definition.
Thus, a \url{DECLARE_PER_THREAD()} primitive may be used to access
a per-thread variable defined in some other file.

\subsection{per\_thread()}

The \url{per_thread()} primitive accesses the specified thread's variable.

\subsection{\_\_get\_thread\_var()}

The \url{__get_thread_var()} primitive accesses the current thread's variable.

\subsection{init\_per\_thread()}

The \url{init_per_thread()} primitive sets all threads' instances of
the specified variable to the specified value.

\subsection{Usage Example}

Suppose that we have a counter that is incremented very frequently
but read out quite rarely.
As will become clear in
Section~\ref{app:primitives:Performance},
it is helpful to implement such a counter using a per-CPU variable.
Such a variable can be defined as follows:

\vspace{5pt}
\begin{minipage}[t]{\columnwidth}
\small 
\begin{verbatim}
DEFINE_PER_THREAD(int, counter);
\end{verbatim}
\end{minipage} 
\vspace{5pt}

The counter must be initialized as follows:

\vspace{5pt}
\begin{minipage}[t]{\columnwidth}
\small 
\begin{verbatim}
init_per_thread(counter, 0);
\end{verbatim}
\end{minipage} 
\vspace{5pt}

A thread can increment its instance of this counter as follows:

\vspace{5pt}
\begin{minipage}[t]{\columnwidth}
\small 
\begin{verbatim}
__get_thread_var(counter)++;
\end{verbatim}
\end{minipage} 
\vspace{5pt}

The value of the counter is then the sum of its instances.
A snapshot of the value of the counter can thus be collected
as follows:

\vspace{5pt}
\begin{minipage}[t]{\columnwidth}
\small 
\begin{verbatim}
for_each_thread(i)
  sum += per_thread(counter, i);
\end{verbatim}
\end{minipage} 
\vspace{5pt}

Again, it is possible to gain a similar effect using other mechanisms,
but per-thread variables combine convenience and high performance.

\section{Performance}
\label{app:primitives:Performance}

It is instructive to compare the performance of the locked increment
shown in
Section~\ref{app:primitives:Locking}
to that of per-thread variables
(see Section~\ref{app:primitives:Per-Thread Variables}),
as well as to conventional increment (as in ``counter++'').

\emph{@@@ need parable on cache thrashing.}

\emph{@@@ more here using performance results from a modest multiprocessor.}

\emph{@@@ Also work in something about critical-section size? Or put later?}

The difference in performance is quite large, to put it mildly.
The purpose of this book is to help you write SMP programs,
perhaps with realtime response, while avoiding such performance
pitfalls.
The next section starts this process by describing some of the
reasons for this performance shortfall.
