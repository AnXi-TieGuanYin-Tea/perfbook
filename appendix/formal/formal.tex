% appendix/formal/formal.tex

\chapter{Formal Verification}
\label{app:formal:Formal Verification}

\QuickQuizChapter{app:formal:Formal Verification}

Parallel algorithms can be hard to write, and even harder to debug.
Testing, though essential, is insufficient, as fatal race conditions
can have extremely low probabilities of occurrence.
Proofs of correctness can be valuable, but in the end are just as
prone to human error as is the original algorithm.

It would be very helpful to have a tool that could somehow locate
all race conditions.
A number of such tools exist, for example,
the language Promela and its compiler Spin.

% appendix/formal/spinhint.html

\section{What are Promela and Spin?}
\label{app:formal:What are Promela and Spin?}

Promela is a language designed to help verify protocols, but which
can also be used to verify small parallel algorithms.
You recode your algorithm and correctness constraints in the C-like
language Promela, and then use Spin to translate it into a C program
that you can compile and run.
The resulting program conducts a full state-space search of your
algorithm, either verifying or finding counter-examples for
assertions that you can include in your Promela program.

This full-state search can extremely powerful, but can also be a two-edged
sword.
If your algorithm is too complex or your Promela implementation is
careless, there might be more states than fit in memory.
Furthermore, even given sufficient memory, the state-space search might
well run for longer than the expected lifetime of the universe.
Therefore, use this tool for compact but complex parallel algorithms.
Attempts to naively apply it to even moderate-scale algorithms (let alone
the full Linux kernel) will end badly.

Promela and Spin may be downloaded from
\url{http://spinroot.com/spin/whatispin.html}.

The above site also gives links to Gerard Holzmann's excellent
book~\cite{Holzmann03a} on Promela and Spin,
as well as searchable online references starting at:
\url{http://www.spinroot.com/spin/Man/index.html}.

The remainder of this article describes how to use Promela to debug
parallel algorithms, starting with simple examples and progressing to
more complex uses.

\section{Promela Example: Non-Atomic Increment}
\label{app:formal:Promela Example: Non-Atomic Increment}

\begin{figure}[tbp]
{ \scriptsize
\begin{verbatim}
  1 #define NUMPROCS 2
  2
  3 byte counter = 0;
  4 byte progress[NUMPROCS];
  5
  6 proctype incrementer(byte me)
  7 {
  8   int temp;
  9
 10   temp = counter;
 11   counter = temp + 1;
 12   progress[me] = 1;
 13 }
 14
 15 init {
 16   int i = 0;
 17   int sum = 0;
 18
 19   atomic {
 20     i = 0;
 21     do
 22     :: i < NUMPROCS ->
 23       progress[i] = 0;
 24       run incrementer(i);
 25       i++
 26     :: i >= NUMPROCS -> break
 27     od;
 28   }
 29   atomic {
 30     i = 0;
 31     sum = 0;
 32     do
 33     :: i < NUMPROCS ->
 34       sum = sum + progress[i];
 35       i++
 36     :: i >= NUMPROCS -> break
 37     od;
 38     assert(sum < NUMPROCS || counter == NUMPROCS)
 39   }
 40 }
\end{verbatim}
}
\caption{Promela Code for Non-Atomic Increment}
\label{fig:analysis:Promela Code for Non-Atomic Increment}
\end{figure}

Figure~\ref{fig:analysis:Promela Code for Non-Atomic Increment}
demonstrates the textbook race condition
resulting from non-atomic increment.
Line 1 defines the number of processes to run (we will vary this
to see the effect on state space), line 3 defines the counter,
and line 4 is used to implement the assertion that appears on
lines 29-39.

Lines 6-13 define a process that increments the counter non-atomically.
The argument \url{me} is the process number, set by the initialization
block later in the code.
Because simple Promela statements are each assumed atomic, we must
break the increment into the two statements on lines 10-11.
The assignment on line 12 marks the process's completion.
Because the Spin system will fully search the state space, including
all possible sequences of states, there is no need for the loop
that would be used for conventional testing.

Lines 15-40 are the initialization block, which is executed first.
Lines 19-28 actually do the initialization, while lines 29-39
perform the assertion.
Both are atomic blocks in order to avoid unnecessarily increasing
the state space: because they are not part of the algorithm proper,
we loose no verification coverage by making them atomic.

The do-od construct on lines 21-27 implements a Promela loop,
which can be thought of as a C {\tt for (;;)} loop containing a
\url{switch} statement that allows expressions in case labels.
The condition blocks (prefixed by {\tt ::})
are scanned non-deterministically,
though in this case only one of the conditions can possibly hold at a given
time.
The first block of the do-od from lines 22-25 initializes the i-th
incrementer's progress cell, runs the i-th incrementer's process, and
then increments the variable \url{i}.
The second block of the do-od on line 26 exits the loop once
these processes have been started.

The atomic block on lines 29-39 also contains a similar do-od
loop that sums up the progress counters.
The {\tt assert()} statement on line 38 verifies that if all processes
have been completed, then all counts have been correctly recorded.

You can build and run this program as follows:

\vspace{5pt}
\begin{minipage}[t]{\columnwidth}
\begin{verbatim}
spin -a increment.spin		# Translate the model to C
cc -DSAFETY -o pan pan.c	# Compile the model
./pan				# Run the model
\end{verbatim}
\end{minipage}
\vspace{5pt}

\begin{figure}[htbp]
{ \scriptsize
\begin{verbatim}
pan: assertion violated ((sum<2)||(counter==2)) (at depth 20)
pan: wrote increment.spin.trail
(Spin Version 4.2.5 -- 2 April 2005)
Warning: Search not completed
        + Partial Order Reduction

Full statespace search for:
        never claim             - (none specified)
        assertion violations    +
        cycle checks            - (disabled by -DSAFETY)
        invalid end states      +

State-vector 40 byte, depth reached 22, errors: 1
      45 states, stored
      13 states, matched
      58 transitions (= stored+matched)
      51 atomic steps
hash conflicts: 0 (resolved)

2.622  memory usage (Mbyte)
\end{verbatim}
}
\caption{Non-Atomic Increment spin Output}
\label{fig:analysis:Non-Atomic Increment spin Output}
\end{figure}

This will produce output as shown in
Figure~\ref{fig:analysis:Non-Atomic Increment spin Output}.
The first line tells us that our assertion was violated (as expected
given the non-atomic increment!).
The second line that a \url{trail} file was written describing how the
assertion was violated.
The ``Warning'' line reiterates that all was not well with our model.
The second paragraph describes the type of state-search being carried out,
in this case for assertion violations and invalid end states.
The third paragraph gives state-size statistics: this small model had only
45 states.
The final line shows memory usage.

The \url{trail} file may be rendered human-readable as follows:

\vspace{5pt}
\begin{minipage}[t]{\columnwidth}
\begin{verbatim}
spin -t -p increment.spin
\end{verbatim}
\end{minipage}
\vspace{5pt}

\begin{figure*}[htbp]
{ \scriptsize
\begin{verbatim}
Starting :init: with pid 0
 1: proc 0 (:init:) line 20 "increment.spin" (state 1) [i = 0]
 2: proc 0 (:init:) line 22 "increment.spin" (state 2) [((i<2))]
 2: proc 0 (:init:) line 23 "increment.spin" (state 3) [progress[i] = 0]
Starting incrementer with pid 1
 3: proc 0 (:init:) line 24 "increment.spin" (state 4) [(run incrementer(i))]
 3: proc 0 (:init:) line 25 "increment.spin" (state 5) [i = (i+1)]
 4: proc 0 (:init:) line 22 "increment.spin" (state 2) [((i<2))]
 4: proc 0 (:init:) line 23 "increment.spin" (state 3) [progress[i] = 0]
Starting incrementer with pid 2
 5: proc 0 (:init:) line 24 "increment.spin" (state 4) [(run incrementer(i))]
 5: proc 0 (:init:) line 25 "increment.spin" (state 5) [i = (i+1)]
 6: proc 0 (:init:) line 26 "increment.spin" (state 6) [((i>=2))]
 7: proc 0 (:init:) line 21 "increment.spin" (state 10) [break]
 8: proc 2 (incrementer) line 10 "increment.spin" (state 1) [temp = counter]
 9: proc 1 (incrementer) line 10 "increment.spin" (state 1) [temp = counter]
10: proc 2 (incrementer) line 11 "increment.spin" (state 2) [counter = (temp+1)]
11: proc 2 (incrementer) line 12 "increment.spin" (state 3) [progress[me] = 1]
12: proc 2 terminates
13: proc 1 (incrementer) line 11 "increment.spin" (state 2) [counter = (temp+1)]
14: proc 1 (incrementer) line 12 "increment.spin" (state 3) [progress[me] = 1]
15: proc 1 terminates
16: proc 0 (:init:) line 30 "increment.spin" (state 12)	[i = 0]
16: proc 0 (:init:) line 31 "increment.spin" (state 13)	[sum = 0]
17: proc 0 (:init:) line 33 "increment.spin" (state 14)	[((i<2))]
17: proc 0 (:init:) line 34 "increment.spin" (state 15)	[sum = (sum+progress[i])]
17: proc 0 (:init:) line 35 "increment.spin" (state 16)	[i = (i+1)]
18: proc 0 (:init:) line 33 "increment.spin" (state 14)	[((i<2))]
18: proc 0 (:init:) line 34 "increment.spin" (state 15)	[sum = (sum+progress[i])]
18: proc 0 (:init:) line 35 "increment.spin" (state 16)	[i = (i+1)]
19: proc 0 (:init:) line 36 "increment.spin" (state 17)	[((i>=2))]
20: proc 0 (:init:) line 32 "increment.spin" (state 21)	[break]
spin: line  38 "increment.spin", Error: assertion violated
spin: text of failed assertion: assert(((sum<2)||(counter==2)))
 21:	proc  0 (:init:) line  38 "increment.spin" (state 22)	[assert(((sum<2)||(counter==2)))]
spin: trail ends after 21 steps
#processes: 1
                counter = 1
                progress[0] = 1
                progress[1] = 1
21: proc 0 (:init:) line 40 "increment.spin" (state 24) <valid end state>
3 processes created
\end{verbatim}
}
\caption{Non-Atomic Increment Error Trail}
\label{fig:analysis:Non-Atomic Increment Error Trail}
\end{figure*}

This gives the output shown in
Figure~\ref{fig:analysis:Non-Atomic Increment Error Trail}.
As can be seen, the first portion of the init block created both
incrementer processes, both of which first fetched the counter,
then both incremented and stored it, losing a count.
The assertion then triggered, after which the global state is displayed.

\section{Promela Example: Atomic Increment}
\label{app:formal:Promela Example: Atomic Increment}

\begin{figure}[htbp]
{ \scriptsize
\begin{verbatim}
  1 proctype incrementer(byte me)
  2 {
  3   int temp;
  4
  5   atomic {
  6     temp = counter;
  7     counter = temp + 1;
  8   }
  9   progress[me] = 1;
 10 }
\end{verbatim}
}
\caption{Promela Code for Atomic Increment}
\label{fig:analysis:Promela Code for Atomic Increment}
\end{figure}

\begin{figure}[htbp]
{ \scriptsize
\begin{verbatim}
(Spin Version 4.2.5 -- 2 April 2005)
        + Partial Order Reduction

Full statespace search for:
        never claim             - (none specified)
        assertion violations    +
        cycle checks            - (disabled by -DSAFETY)
        invalid end states      +

State-vector 40 byte, depth reached 20, errors: 0
      52 states, stored
      21 states, matched
      73 transitions (= stored+matched)
      66 atomic steps
hash conflicts: 0 (resolved)

2.622   memory usage (Mbyte)

unreached in proctype incrementer
        (0 of 5 states)
unreached in proctype :init:
        (0 of 24 states)
\end{verbatim}
}
\caption{Atomic Increment spin Output}
\label{fig:analysis:Atomic Increment spin Output}
\end{figure}

It is easy to fix this example by placing the body of the incrementer
processes in an atomic blocks as shown in
Figure~\ref{fig:analysis:Promela Code for Atomic Increment}.
One could also have simply replaced the pair of statements with
{\tt counter = counter + 1}, because Promela statements are
atomic.
Either way, running this modified model gives us an error-free traversal
of the state space, as shown in
Figure~\ref{fig:analysis:Atomic Increment spin Output}.

\subsection{Combinatorial Explosion}
\label{app:formal:Combinatorial Explosion}

\begin{table}
\begin{center}
\begin{tabular}{c|r|r}
	\# incrementers & \# states &	megabytes \\
	\hline
	\hline
	1 &		        11 &          2.6 \\
	\hline
	2 &		        52 &          2.6 \\
	\hline
	3 &		       372 &          2.6 \\
	\hline
	4 &		     3,496 &          2.7 \\
	\hline
	5 &		    40,221 &          5.0 \\
	\hline
	6 &		   545,720 &         40.5 \\
	\hline
	7 &		 8,521,450 &        652.7 \\
\end{tabular}
\end{center}
\caption{Memory Usage of Increment Model}
\label{tab:advsync:Memory Usage of Increment Model}
\end{table}

Table~\ref{tab:advsync:Memory Usage of Increment Model}
shows the number of states and memory consumed
as a function of number of incrementers modeled
(by redefining {\tt NUMPROCS}):

Running unnecessarily large models is thus subtly discouraged, although
652MB is well within the limits of modern desktop and laptop machines.

With this example under our belt, let's take a closer look at the
commands used to analyze Promela models and then look at more
elaborate examples.

\section{How to Use Promela}
\label{app:formal:How to Use Promela}

Given a source file \url{qrcu.spin}, one can use the following commands:

\begin{description}
\item[{\tt spin -a qrcu.spin}]
	Create a file pan.c that fully searches the state machine.
\item[{\tt cc -DSAFETY -o pan pan.c}]
	Compile the generated state-machine search.  The \url{-DSAFETY}
	generates optimizations that are appropriate if you have only
	assertions (and perhaps \url{never} statements).  If you have
	liveness, fairness, or forward-progress checks, you may need
	to compile without \url{-DSAFETY}.  If you leave off \url{-DSAFETY}
	when you could have used it, the program will let you know.
	
	The optimizations produced by \url{-DSAFETY} greatly speed things
	up, so you should use it when you can.
	An example situation where you cannot use \url{-DSAFETY} is
	when checking for livelocks (AKA ``non-progress cycles'')
	via \url{-DNP}.
\item[{\tt ./pan}]
	This actually searches the state space.  The number of states
	can reach into the tens of millions with very small state
	machines, so you will need a machine with large memory.
	For example, qrcu.spin with 3 readers and 2 updaters required
	2.7GB of memory.

	If you aren't sure whether your machine has enough memory,
	run \url{top} in one window and \url{./pan} in another.  Keep the
	focus on the \url{./pan} window so that you can quickly kill
	execution if need be.  As soon as CPU time drops much below
	100\%, kill \url{./pan}.  If you have removed focus from the
	window running \url{./pan}, you may wait a long time for the
	windowing system to grab enough memory to do anything for
	you.

	Don't forget to capture the output, especially
	if you are working on a remote machine,

	If your model includes forward-progress checks, you will likely
	need to enable ``weak fairness'' via the \url{-f} command-line
	argument to \url{./pan}.
	If your forward-progress checks involve \url{accept} labels,
	you will also need the \url{-a} argument.
	% forward reference to model: formal.2009.02.19a in
	% /home/linux/git/userspace-rcu/formal-model.
\item[{\tt spin -t -p qrcu.spin}]
	Given \url{trail} file output by a run that encountered an
	error, output the sequence of steps leading to that error.
	The \url{-g} flag will also include the values of changed
	global variables, and the  \url{-l} flag will also include
	the values of changed local variables.
\end{description}

\subsection{Promela Peculiarities}
\label{app:formal:Promela Peculiarities}

Although all computer languages have underlying similarities,
Promela will provide some surprises to people used to coding in C,
C++, or Java.

\begin{enumerate}
\item	In C, ``\url{;}'' terminates statements.  In Promela it separates them.
	Fortunately, more recent versions of Spin have become
	much more forgiving of ``extra'' semicolons.
\item	Promela's looping construct, the \url{do} statement, takes
	conditions.
	This \url{do} statement closely resembles a looping if-then-else
	statement.
\item	In C's \url{switch} statement, if there is no matching case, the whole
	statement is skipped.  In Promela's equivalent, confusingly called
	\url{if}, if there is no matching guard expression, you get an error
	without a recognizable corresponding error message.
	So, if the error output indicates an innocent line of code,
	check to see if you left out a condition from an \url{if} or \url{do}
	statement.
\item	When creating stress tests in C, one usually races suspect operations
	against each other repeatedly.	In Promela, one instead sets up
	a single race, because Promela will search out all the possible
	outcomes from that single race.	Sometimes you do need to loop
	in Promela, for example, if multiple operations overlap, but
	doing so greatly increases the size of your state space.
\item	In C, the easiest thing to do is to maintain a loop counter to track
	progress and terminate the loop.  In Promela, loop counters
	must be avoided like the plague because they cause the state
	space to explode.  On the other hand, there is no penalty for
	infinite loops in Promela as long as the none of the variables
	monotonically increase or decrease -- Promela will figure out
	how many passes through the loop really matter, and automatically
	prune execution beyond that point.
\item	In C torture-test code, it is often wise to keep per-task control
	variables.  They are cheap to read, and greatly aid in debugging the
	test code.  In Promela, per-task control variables should be used
	only when there is no other alternative.  To see this, consider
	a 5-task verification with one bit each to indicate completion.
	This gives 32 states.  In contrast, a simple counter would have
	only six states, more than a five-fold reduction.  That factor
	of five might not seem like a problem, at least not until you
	are struggling with a verification program possessing more than
	150 million states consuming more than 10GB of memory!
\item	One of the most challenging things both in C torture-test code and
	in Promela is formulating good assertions.  Promela also allows
	\url{never} claims that act sort of like an assertion replicated
	between every line of code.
\item	Dividing and conquering is extremely helpful in Promela in keeping
	the state space under control.  Splitting a large model into two
	roughly equal halves will result in the state space of each
	half being roughly the square root of the whole.
	For example, a million-state combined model might reduce to a
	pair of thousand-state models.
	Not only will Promela handle the two smaller models much more
	quickly with much less memory, but the two smaller algorithms
	are easier for people to understand.
\end{enumerate}


\subsection{Promela Coding Tricks}
\label{app:formal:Promela Coding Tricks}

Promela was designed to analyze protocols, so using it on parallel programs
is a bit abusive.
The following tricks can help you to abuse Promela safely:

\begin{enumerate}
\item	Memory reordering.  Suppose you have a pair of statements
	copying globals x and y to locals r1 and r2, where ordering
	matters (e.g., unprotected by locks), but where you have
	no memory barriers.  This can be modeled in Promela as follows:

\vspace{5pt}
\begin{minipage}[t]{\columnwidth}
\begin{verbatim}
  1 if
  2 :: 1 -> r1 = x;
  3   r2 = y
  4 :: 1 -> r2 = y;
  5   r1 = x
  6 fi
\end{verbatim}
\end{minipage}
\vspace{5pt}

	The two branches of the \url{if} statement will be selected
	nondeterministically, since they both are available.
	Because the full state space is searched, \emph{both} choices
	will eventually be made in all cases.

	Of course, this trick will cause your state space to explode
	if used too heavily.
	In addition, it requires you to anticipate possible reorderings.

\begin{figure}[tbp]
{ % \scriptsize
\begin{verbatim}
  1 i = 0;
  2 sum = 0;
  3 do
  4 :: i < N_QRCU_READERS ->
  5   sum = sum + (readerstart[i] == 1 &&
  6     readerprogress[i] == 1);
  7   i++
  8 :: i >= N_QRCU_READERS ->
  9   assert(sum == 0);
 10   break
 11 od
\end{verbatim}
}
\caption{Complex Promela Assertion}
\label{fig:analysis:Complex Promela Assertion}
\end{figure}

\begin{figure}[tbp]
{ % \scriptsize
\begin{verbatim}
  1 atomic {
  2   i = 0;
  3   sum = 0;
  4   do
  5   :: i < N_QRCU_READERS ->
  6     sum = sum + (readerstart[i] == 1 &&
  7       readerprogress[i] == 1);
  8     i++
  9   :: i >= N_QRCU_READERS ->
 10     assert(sum == 0);
 11     break
 12   od
 13 }
\end{verbatim}
}
\caption{Atomic Block for Complex Promela Assertion}
\label{fig:analysis:Atomic Block for Complex Promela Assertion}
\end{figure}

\item	State reduction.  If you have complex assertions, evaluate
	them under \url{atomic}.  After all, they are not part of the
	algorithm.  One example of a complex assertion (to be discussed
	in more detail later) is as shown in
	Figure~\ref{fig:analysis:Complex Promela Assertion}.

	There is no reason to evaluate this assertion
	non-atomically, since it is not actually part of the algorithm.
	Because each statement contributes to state, we can reduce
	the number of useless states by enclosing it in an \url{atomic}
	block as shown in
	Figure~\ref{fig:analysis:Atomic Block for Complex Promela Assertion}

\item	Promela does not provide functions.
	You must instead use C preprocessor macros.
	However, you must use them carefully in order to avoid
	combinatorial explosion.
\end{enumerate}

Now we are ready for more complex examples.

\section{Promela Example: Locking}
\label{app:formal:Promela Example: Locking}

\begin{figure}[tbp]
{ % \scriptsize
\begin{verbatim}
  1 #define spin_lock(mutex) \
  2   do \
  3   :: 1 -> atomic { \
  4       if \
  5       :: mutex == 0 -> \
  6         mutex = 1; \
  7         break \
  8       :: else -> skip \
  9       fi \
 10     } \
 11   od
 12
 13 #define spin_unlock(mutex) \
 14   mutex = 0
\end{verbatim}
}
\caption{Promela Code for Spinlock}
\label{fig:analysis:Promela Code for Spinlock}
\end{figure}

Since locks are generally useful, \url{spin_lock()} and
\url{spin_unlock()}
macros are provided in {\tt lock.h}, which may be included from
multiple Promela models, as shown in
Figure~\ref{fig:analysis:Promela Code for Spinlock}.
The \url{spin_lock()} macro contains an infinite do-od loop
spanning lines 2-11,
courtesy of the single guard expression of ``1'' on line 3.
The body of this loop is a single atomic block that contains
an if-fi statement.
The if-fi construct is similar to the do-od construct, except
that it takes a single pass rather than looping.
If the lock is not held on line 5, then line 6 acquires it and
line 7 breaks out of the enclosing do-od loop (and also exits
the atomic block).
On the other hand, if the lock is already held on line 8,
we do nothing (\url{skip}), and fall out of the if-fi and the
atomic block so as to take another pass through the outer
loop, repeating until the lock is available.

The \url{spin_unlock()} macro simply marks the lock as no
longer held.

Note that memory barriers are not needed because Promela assumes
full ordering.
In any given Promela state, all processes agree on both the current
state and the order of state changes that caused us to arrive at
the current state.
This is analogous to the ``sequentially consistent'' memory model
used by a few computer systems (such as MIPS and PA-RISC).
As noted earlier, and as will be seen in a later example,
weak memory ordering must be explicitly coded.

\begin{figure}[htbp]
{ % \scriptsize
\begin{verbatim}
  1 #include "lock.h"
  2
  3 #define N_LOCKERS 3
  4
  5 bit mutex = 0;
  6 bit havelock[N_LOCKERS];
  7 int sum;
  8
  9 proctype locker(byte me)
 10 {
 11   do
 12   :: 1 ->
 13     spin_lock(mutex);
 14     havelock[me] = 1;
 15     havelock[me] = 0;
 16     spin_unlock(mutex)
 17   od
 18 }
 19
 20 init {
 21   int i = 0;
 22   int j;
 23
 24 end:  do
 25   :: i < N_LOCKERS ->
 26     havelock[i] = 0;
 27     run locker(i);
 28     i++
 29   :: i >= N_LOCKERS ->
 30     sum = 0;
 31     j = 0;
 32     atomic {
 33       do
 34       :: j < N_LOCKERS ->
 35         sum = sum + havelock[j];
 36         j = j + 1
 37       :: j >= N_LOCKERS ->
 38         break
 39       od
 40     }
 41     assert(sum <= 1);
 42     break
 43   od
 44 }
\end{verbatim}
}
\caption{Promela Code to Test Spinlocks}
\label{fig:analysis:Promela Code to Test Spinlocks}
\end{figure}

These macros are tested by the Promela code shown in
Figure~\ref{fig:analysis:Promela Code to Test Spinlocks}.
This code is similar to that used to test the increments,
with the number of locking processes defined by the \url{N_LOCKERS}
macro definition on line 3.
The mutex itself is defined on line 5, an array to track the lock owner
on line 6, and line 7 is used by assertion
code to verify that only one process holds the lock.

The locker process is on lines 9-18, and simply loops forever
acquiring the lock on line 13, claiming it on line 14,
unclaiming it on line 15, and releasing it on line 16.

The init block on lines 20-44 initializes the current locker's
havelock array entry on line 26, starts the current locker on
line 27, and advances to the next locker on line 28.
Once all locker processes are spawned, the do-od loop
moves to line 29, which checks the assertion.
Lines 30 and 31 initialize the control variables,
lines 32-40 atomically sum the havelock array entries,
line 41 is the assertion, and line 42 exits the loop.

We can run this model by placing the above two code fragments into
files named \url{lock.h} and \url{lock.spin}, respectively, and then running
the following commands:

\vspace{5pt}
\begin{minipage}[t]{\columnwidth}
\begin{verbatim}
spin -a lock.spin
cc -DSAFETY -o pan pan.c
./pan
\end{verbatim}
\end{minipage}
\vspace{5pt}

\begin{figure}[htbp]
{ \scriptsize
\begin{verbatim}
(Spin Version 4.2.5 -- 2 April 2005)
        + Partial Order Reduction

Full statespace search for:
        never claim             - (none specified)
        assertion violations    +
        cycle checks            - (disabled by -DSAFETY)
        invalid end states      +

State-vector 40 byte, depth reached 357, errors: 0
     564 states, stored
     929 states, matched
    1493 transitions (= stored+matched)
     368 atomic steps
hash conflicts: 0 (resolved)

2.622   memory usage (Mbyte)

unreached in proctype locker
        line 18, state 20, "-end-"
        (1 of 20 states)
unreached in proctype :init:
        (0 of 22 states)
\end{verbatim}
}
\caption{Output for Spinlock Test}
\label{fig:analysis:Output for Spinlock Test}
\end{figure}

The output will look something like that shown in
Figure~\ref{fig:analysis:Output for Spinlock Test}.
As expected, this run has no assertion failures (``errors: 0'').

\QuickQuiz{}
	Why is there an unreached statement in
	locker?  After all, isn't this a \emph{full} state-space
	search???
\QuickQuizAnswer{
	The locker process is an infinite loop, so control
	never reaches the end of this process.
	However, since there are no monotonically increasing variables,
	Promela is able to model this infinite loop with a small
	number of states.
} \QuickQuizEnd

\QuickQuiz{}
	What are some Promela code-style issues with this example?
\QuickQuizAnswer{
	There are several:
	\begin{enumerate}
	\item	The declaration of {\tt sum} should be moved to within
		the init block, since it is not used anywhere else.
	\item	The assertion code should be moved outside of the
		initialization loop.  The initialization loop can
		then be placed in an atomic block, greatly reducing
		the state space (by how much?).
	\item	The atomic block covering the assertion code should
		be extended to include the initialization of {\tt sum}
		and {\tt j}, and also to cover the assertion.
		This also reduces the state space (again, by how
		much?).
	\end{enumerate}
} \QuickQuizEnd


\section{Promela Example: QRCU}
\label{app:formal:Promela Example: QRCU}

This final example demonstrates a real-world use of Promela on Oleg
Nesterov's
QRCU~\cite{OlegNesterov2006QRCU,OlegNesterov2006aQRCU},
but modified to speed up the \url{synchronize_qrcu()}
fastpath.

But first, what is QRCU?

QRCU is a variant of SRCU~\cite{PaulEMcKenney2006c}
that trades somewhat higher read overhead
(atomic increment and decrement on a global variable) for extremely
low grace-period latencies.
If there are no readers, the grace period will be detected in less
than a microsecond, compared to the multi-millisecond grace-period
latencies of most other RCU implementations.

\begin{enumerate}
\item	There is a \url{qrcu_struct} that defines a QRCU domain.
	Like SRCU (and unlike other variants of RCU) QRCU's action
	is not global, but instead focused on the specified
	\url{qrcu_struct}.
\item	There are \url{qrcu_read_lock()} and \url{qrcu_read_unlock()}
	primitives that delimit QRCU read-side critical sections.
	The corresponding \url{qrcu_struct} must be passed into
	these primitives, and the return value from \url{rcu_read_lock()}
	must be passed to \url{rcu_read_unlock()}.

	For example:

\vspace{5pt}
\begin{minipage}[t]{\columnwidth}
\begin{verbatim}
idx = qrcu_read_lock(&my_qrcu_struct);
/* read-side critical section. */
qrcu_read_unlock(&my_qrcu_struct, idx);
\end{verbatim}
\end{minipage}
\vspace{5pt}

\item	There is a \url{synchronize_qrcu()} primitive that blocks until
	all pre-existing QRCU read-side critical sections complete,
	but, like SRCU's \url{synchronize_srcu()}, QRCU's
	\url{synchronize_qrcu()} need wait only for those read-side
	critical sections that are using the same \url{qrcu_struct}.
	
	For example, \url{synchronize_qrcu(&your_qrcu_struct)}
	would \emph{not} need to wait on the earlier QRCU read-side
	critical section.
	In contrast, \url{synchronize_qrcu(&my_qrcu_struct)}
	\emph{would} need to wait, since it shares the same
	\url{qrcu_struct}.
\end{enumerate}

A Linux-kernel patch for QRCU has been
produced~\cite{PaulMcKenney2007QRCUpatch},
but has not yet been included in the Linux kernel as of
April 2008.

\begin{figure}[htbp]
{ % \scriptsize
\begin{verbatim}
  1 #include "lock.h"
  2
  3 #define N_QRCU_READERS 2
  4 #define N_QRCU_UPDATERS 2
  5
  6 bit idx = 0;
  7 byte ctr[2];
  8 byte readerprogress[N_QRCU_READERS];
  9 bit mutex = 0;
\end{verbatim}
}
\caption{QRCU Global Variables}
\label{fig:analysis:QRCU Global Variables}
\end{figure}

Returning to the Promela code for QRCU, the global variables are as shown in
Figure~\ref{fig:analysis:QRCU Global Variables}.
This example uses locking, hence including \url{lock.h}.
Both the number of readers and writers can be varied using the
two \url{#define} statements, giving us not one but two ways to create
combinatorial explosion.
The \url{idx} variable controls which of the two elements of the \url{ctr}
array will be used by readers, and the \url{readerprogress} variable
allows to assertion to determine when all the readers are finished
(since a QRCU update cannot be permitted to complete until all
pre-existing readers have completed their QRCU read-side critical
sections).
The readerprogress array elements have values as follows,
indicating the state of the corresponding reader:

\begin{enumerate}
\item	0: not yet started.
\item	1: within QRCU read-side critical section.
\item	2: finished with QRCU read-side critical section.
\end{enumerate}

Finally, the \url{mutex} variable is used to serialize updaters' slowpaths.

\begin{figure}[htbp]
{ % \scriptsize
\begin{verbatim}
  1 proctype qrcu_reader(byte me)
  2 {
  3   int myidx;
  4
  5   do
  6   :: 1 ->
  7     myidx = idx;
  8     atomic {
  9       if
 10       :: ctr[myidx] > 0 ->
 11         ctr[myidx]++;
 12         break
 13       :: else -> skip
 14       fi
 15     }
 16   od;
 17   readerprogress[me] = 1;
 18   readerprogress[me] = 2;
 19   atomic { ctr[myidx]-- }
 20 }
\end{verbatim}
}
\caption{QRCU Reader Process}
\label{fig:analysis:QRCU Reader Process}
\end{figure}

QRCU readers are modeled by the \url{qrcu_reader()} process shown in
Figure~\ref{fig:analysis:QRCU Reader Process}.
A do-od loop spans lines 5-16, with a single guard of ``1''
on line 6 that makes it an infinite loop.
Line 7 captures the current value of the global index, and lines 8-15
atomically increment it (and break from the infinite loop)
if its value was non-zero (\url{atomic_inc_not_zero()}).
Line 17 marks entry into the RCU read-side critical section, and
line 18 marks exit from this critical section, both lines for the benefit of
the {\tt assert()} statement that we shall encounter later.
Line 19 atomically decrements the same counter that we incremented,
thereby exiting the RCU read-side critical section.

\begin{figure}[htbp]
{ % \scriptsize
\begin{verbatim}
  1 #define sum_unordered \
  2   atomic { \
  3     do \
  4     :: 1 -> \
  5       sum = ctr[0]; \
  6       i = 1; \
  7       break \
  8     :: 1 -> \
  9       sum = ctr[1]; \
 10       i = 0; \
 11       break \
 12     od; \
 13   } \
 14   sum = sum + ctr[i]
\end{verbatim}
}
\caption{QRCU Unordered Summation}
\label{fig:analysis:QRCU Unordered Summation}
\end{figure}

The C-preprocessor macro shown in
Figure~\ref{fig:analysis:QRCU Unordered Summation}
sums the pair of counters so as to emulate weak memory ordering.
Lines 2-13 fetch one of the counters, and line 14 fetches the other
of the pair and sums them.
The atomic block consists of a single do-od statement.
This do-od statement (spanning lines 3-12) is unusual in that
it contains two unconditional
branches with guards on lines 4 and 8, which causes Promela to
non-deterministically choose one of the two (but again, the full
state-space search causes Promela to eventually make all possible
choices in each applicable situation).
The first branch fetches the zero-th counter and sets \url{i} to 1 (so
that line 14 will fetch the first counter), while the second
branch does the opposite, fetching the first counter and setting \url{i}
to 0 (so that line 14 will fetch the second counter).

\QuickQuiz{}
	Is there a more straightforward way to code the do-od statement?
\QuickQuizAnswer{
	Yes.
	Replace it with {\tt if-fi} and remove the two {\tt break} statements.
} \QuickQuizEnd

\begin{figure}[htbp]
{ \scriptsize
\begin{verbatim}
  1 proctype qrcu_updater(byte me)
  2 {
  3   int i;
  4   byte readerstart[N_QRCU_READERS];
  5   int sum;
  6
  7   do
  8   :: 1 ->
  9
 10     /* Snapshot reader state. */
 11
 12     atomic {
 13       i = 0;
 14       do
 15       :: i < N_QRCU_READERS ->
 16         readerstart[i] = readerprogress[i];
 17         i++
 18       :: i >= N_QRCU_READERS ->
 19         break
 20       od
 21     }
 22
 23     sum_unordered;
 24     if
 25     :: sum <= 1 -> sum_unordered
 26     :: else -> skip
 27     fi;
 28     if
 29     :: sum > 1 ->
 30       spin_lock(mutex);
 31       atomic { ctr[!idx]++ }
 32       idx = !idx;
 33       atomic { ctr[!idx]-- }
 34       do
 35       :: ctr[!idx] > 0 -> skip
 36       :: ctr[!idx] == 0 -> break
 37       od;
 38       spin_unlock(mutex);
 39     :: else -> skip
 40     fi;
 41
 42     /* Verify reader progress. */
 43
 44     atomic {
 45       i = 0;
 46       sum = 0;
 47       do
 48       :: i < N_QRCU_READERS ->
 49         sum = sum + (readerstart[i] == 1 &&
 50                readerprogress[i] == 1);
 51         i++
 52       :: i >= N_QRCU_READERS ->
 53         assert(sum == 0);
 54         break
 55       od
 56     }
 57   od
 58 }
\end{verbatim}
}
\caption{QRCU Updater Process}
\label{fig:analysis:QRCU Updater Process}
\end{figure}

With the \url{sum_unordered} macro in place, we can now proceed
to the update-side process shown in
Figure.
The update-side process repeats indefinitely, with the corresponding
do-od loop ranging over lines 7-57.
Each pass through the loop first snapshots the global {\tt readerprogress}
array into the local {\tt readerstart} array on lines 12-21.
This snapshot will be used for the assertion on line 53.
Line 23 invokes \url{sum_unordered}, and then lines 24-27
re-invoke \url{sum_unordered} if the fastpath is potentially
usable.

Lines 28-40 execute the slowpath code if need be, with
lines 30 and 38 acquiring and releasing the update-side lock,
lines 31-33 flipping the index, and lines 34-37 waiting for
all pre-existing readers to complete.

Lines 44-56 then compare the current values in the {\tt readerprogress}
array to those collected in the {\tt readerstart} array,
forcing an assertion failure should any readers that started before
this update still be in progress.

\QuickQuiz{}
	Why are there atomic blocks at lines 12-21
	and lines 44-56, when the operations within those atomic
	blocks have no atomic implementation on any current
	production microprocessor?
\QuickQuizAnswer{
	Because those operations are for the benefit of the
	assertion only.  They are not part of the algorithm itself.
	There is therefore no harm in marking them atomic, and
	so marking them greatly reduces the state space that must
	be searched by the Promela model.
} \QuickQuizEnd

\QuickQuiz{}
	Is the re-summing of the counters on lines 24-27
	\emph{really} necessary???
\QuickQuizAnswer{
	Yes.  To see this, delete these lines and run the model.

	Alternatively, consider the following sequence of steps:

	\begin{enumerate}
	\item	One process is within its RCU read-side critical
		section, so that the value of {\tt ctr[0]} is zero and
		the value of {\tt ctr[1]} is two.
	\item	An updater starts executing, and sees that the sum of
		the counters is two so that the fastpath cannot be
		executed.  It therefore acquires the lock.
	\item	A second updater starts executing, and fetches the value
		of {\tt ctr[0]}, which is zero.
	\item	The first updater adds one to {\tt ctr[0]}, flips
		the index (which now becomes zero), then subtracts
		one from {\tt ctr[1]} (which now becomes one).
	\item	The second updater fetches the value of {\tt ctr[1]},
		which is now one.
	\item	The second updater now incorrectly concludes that it
		is safe to proceed on the fastpath, despite the fact
		that the original reader has not yet completed.
	\end{enumerate}
} \QuickQuizEnd

\begin{figure}[htbp]
{ % \scriptsize
\begin{verbatim}
  1 init {
  2   int i;
  3
  4   atomic {
  5     ctr[idx] = 1;
  6     ctr[!idx] = 0;
  7     i = 0;
  8     do
  9     :: i < N_QRCU_READERS ->
 10       readerprogress[i] = 0;
 11       run qrcu_reader(i);
 12       i++
 13     :: i >= N_QRCU_READERS -> break
 14     od;
 15     i = 0;
 16     do
 17     :: i < N_QRCU_UPDATERS ->
 18       run qrcu_updater(i);
 19       i++
 20     :: i >= N_QRCU_UPDATERS -> break
 21     od
 22   }
 23 }
\end{verbatim}
}
\caption{QRCU Initialization Process}
\label{fig:analysis:QRCU Initialization Process}
\end{figure}

All that remains is the initialization block shown in
Figure~\ref{fig:analysis:QRCU Initialization Process}.
This block simply initializes the counter pair on lines 5-6,
spawns the reader processes on lines 7-14, and spawns the updater
processes on lines 15-21.
This is all done within an atomic block to reduce state space.

\subsection{Running the QRCU Example}
\label{app:formal:Running the QRCU Example}

To run the QRCU example, combine the code fragments in the previous
section into a single file named \url{qrcu.spin}, and place the definitions
for \url{spin_lock()} and \url{spin_unlock()} into a file named
\url{lock.h}.
Then use the following commands to build and run the QRCU model:

\vspace{5pt}
\begin{minipage}[t]{\columnwidth}
\begin{verbatim}
spin -a qrcu.spin
cc -DSAFETY -o pan pan.c
./pan
\end{verbatim}
\end{minipage}
\vspace{5pt}

\begin{table}
\begin{center}
\begin{tabular}{c|r|r|r}
	updaters &
	    readers &
		   \# states & MB \\
	\hline
	1 & 1 &         376 &      2.6 \\
	\hline
	1 & 2 &       6,177 &      2.9 \\
	\hline
	1 & 3 &      82,127 &      7.5 \\
	\hline
	2 & 1 &      29,399 &      4.5 \\
	\hline
	2 & 2 &   1,071,180 &     75.4 \\
	\hline
	2 & 3 &  33,866,700 &  2,715.2 \\
	\hline
	3 & 1 &     258,605 &     22.3 \\
	\hline
	3 & 2 & 169,533,000 & 14,979.9 \\
\end{tabular}
\end{center}
\caption{Memory Usage of QRCU Model}
\label{tab:advsync:Memory Usage of QRCU Model}
\end{table}

The resulting output shows that this model passes all of the cases
shown in
Table~\ref{tab:advsync:Memory Usage of QRCU Model}.
Now, it would be nice to run the case with three readers and three
updaters, however, simple extrapolation indicates that this will
require on the order of a terabyte of memory best case.
So, what to do?
Here are some possible approaches:

\begin{enumerate}
\item	See whether a smaller number of readers and updaters suffice
	to prove the general case.
\item	Manually construct a proof of correctness.
\item	Use a more capable tool.
\item	Divide and conquer.
\end{enumerate}

The following sections discuss each of these approaches.

\subsection{How Many Readers and Updaters Are Really Needed?}
\label{app:formal:How Many Readers and Updaters Are Really Needed?}

One approach is to look carefully at the Promela code for
\url{qrcu_updater()} and notice that the only global state
change is happening under the lock.
Therefore, only one updater at a time can possibly be modifying
state visible to either readers or other updaters.
This means that any sequences of state changes can be carried
out serially by a single updater due to the fact that Promela does a full
state-space search.
Therefore, at most two updaters are required: one to change state
and a second to become confused.

The situation with the readers is less clear-cut, as each reader
does only a single read-side critical section then terminates.
It is possible to argue that the useful number of readers is limited,
due to the fact that the fastpath must see at most a zero and a one
in the counters.
This is a fruitful avenue of investigation, in fact, it leads to
the full proof of correctness described in the next section.

\subsection{Alternative Approach: Proof of Correctness}
\label{app:formal:Alternative Approach: Proof of Correctness}

An informal proof~\cite{PaulMcKenney2007QRCUpatch}
follows:

\begin{enumerate}
\item	For \url{synchronize_qrcu()} to exit too early, then
	by definition there must have been at least one reader
	present during \url{synchronize_qrcu()}'s full
	execution.
\item	The counter corresponding to this reader will have been
	at least 1 during this time interval.
\item	The \url{synchronize_qrcu()} code forces at least one
	of the counters to be at least 1 at all times.
\item	Therefore, at any given point in time, either one of the
	counters will be at least 2, or both of the counters will
	be at least one.
\item	However, the \url{synchronize_qrcu()} fastpath code
	can read only one of the counters at a given time.
	It is therefore possible for the fastpath code to fetch
	the first counter while zero, but to race with a counter
	flip so that the second counter is seen as one.
\item	There can be at most one reader persisting through such
	a race condition, as otherwise the sum would be two or
	greater, which would cause the updater to take the slowpath.
\item	But if the race occurs on the fastpath's first read of the
	counters, and then again on its second read, there have
	to have been two counter flips.
\item	Because a given updater flips the counter only once, and
	because the update-side lock prevents a pair of updaters
	from concurrently flipping the counters, the only way that
	the fastpath code can race with a flip twice is if the
	first updater completes.
\item	But the first updater will not complete until after all
	pre-existing readers have completed.
\item	Therefore, if the fastpath races with a counter flip
	twice in succession, all pre-existing readers must have
	completed, so that it is safe to take the fastpath.
\end{enumerate}

Of course, not all parallel algorithms have such simple proofs.
In such cases, it may be necessary to enlist more capable tools.

\subsection{Alternative Approach: More Capable Tools}
\label{app:formal:Alternative Approach: More Capable Tools}

Although Promela and Spin are quite useful,
much more capable tools are available, particularly for verifying
hardware.
This means that if it is possible to translate your algorithm
to the hardware-design VHDL language, as it often will be for
low-level parallel algorithms, then it is possible to apply these
tools to your code (for example, this was done for the first
realtime RCU algorithm).
However, such tools can be quite expensive.

Although the advent of commodity multiprocessing
might eventually result in powerful free-software model-checkers
featuring fancy state-space-reduction capabilities,
this does not help much in the here and now.

As an aside, there are Spin features that support approximate searches
that require fixed amounts of memory, however, I have never been able
to bring myself to trust approximations when verifying parallel
algorithms.

Another approach might be to divide and conquer.

\subsection{Alternative Approach: Divide and Conquer}
\label{app:formal:Alternative Approach: Divide and Conquer}

It is often possible to break down a larger parallel algorithm into
smaller pieces, which can then be proven separately.
For example, a 10-billion-state model might be broken into a pair
of 100,000-state models.
Taking this approach not only makes it easier for tools such as
Promela to verify your algorithms, it can also make your algorithms
easier to understand.

% appendix/formal/dyntickrcu.tex

\section{Promela Parable: dynticks and Preemptable RCU}
\label{app:formal:Promela Parable: dynticks and Preemptable RCU}

In early 2008, a preemptable variant of RCU was accepted into
mainline Linux in support of real-time workloads,
a variant similar to the RCU implementations in
the -rt patchset~\cite{IngoMolnar05a}
since August 2005.
Preemptable RCU is needed for real-time workloads because older
RCU implementations disable preemption across RCU read-side
critical sections, resulting in excessive real-time latencies.

However, one disadvantage of the older -rt implementation
(described in Appendix~\ref{app:rcuimpl:Preemptable RCU})
was that each grace period
requires work to be done on each CPU, even if that CPU is in a low-power
``dynticks-idle'' state,
and thus incapable of executing RCU read-side critical sections.
The idea behind the dynticks-idle state is that idle CPUs
should be physically powered down in order to conserve energy.
In short, preemptable RCU can disable a valuable energy-conservation
feature of recent Linux kernels.
Although Josh Triplett and Paul McKenney
had discussed some approaches for allowing
CPUs to remain in low-power state throughout an RCU grace period
(thus preserving the Linux kernel's ability to conserve energy), matters
did not come to a head until Steve Rostedt integrated a new dyntick
implementation with preemptable RCU in the -rt patchset.

This combination caused one of Steve's systems to hang on boot, so in
October, Paul coded up a dynticks-friendly modification to preemptable RCU's
grace-period processing.
Steve coded up \url{rcu_irq_enter()} and \url{rcu_irq_exit()}
interfaces called from the
\url{irq_enter()} and \url{irq_exit()} interrupt
entry/exit functions.
These \url{rcu_irq_enter()} and \url{rcu_irq_exit()}
functions are needed to allow RCU to reliably handle situations where
a dynticks-idle CPUs is momentarily powered up for an interrupt
handler containing RCU read-side critical sections.
With these changes in place, Steve's system booted reliably,
but Paul continued inspecting the code periodically on the assumption
that we could not possibly have gotten the code right on the first try.

Paul reviewed the code repeatedly from October 2007 to February 2008,
and almost always found at least one bug.
In one case, Paul even coded and tested a fix before realizing that the
bug was illusory, and in fact in all cases, the ``bug'' turned out to be
illusory.

Near the end of February, Paul grew tired of this game.
He therefore decided to enlist the aid of
Promela and spin~\cite{Holzmann03a}, as described in
Appendix~\ref{app:formal:Formal Verification}.
The following presents a series of seven increasingly realistic
Promela models, the last of which passes, consuming about
40GB of main memory for the state space.

More important, Promela and Spin did find a very subtle bug for me!!!

\QuickQuiz{}
	Yeah, that's great!!!
	Now, just what am I supposed to do if I don't happen to have a
	machine with 40GB of main memory???
\QuickQuizAnswer{
	Relax, there are a number of lawful answers to
	this question:
	\begin{enumerate}
	\item	Further optimize the model, reducing its memory consumption.
	\item	Work out a pencil-and-paper proof, perhaps starting with the
		comments in the code in the Linux kernel.
	\item	Devise careful torture tests, which, though they cannot prove
		the code correct, can find hidden bugs.
	\item	There is some movement towards tools that do model
		checking on clusters of smaller machines.
		However, please note that we have not actually used such
		tools myself, courtesy of some large machines that Paul has
		occasional access to.
	\end{enumerate}
} \QuickQuizEnd

Still better would be to come up with a simpler and faster algorithm
that has a smaller state space.
Even better would be an algorithm so simple that its correctness was
obvious to the casual observer!

Section~\ref{app:formal:Introduction to Preemptable RCU and dynticks}
gives an overview of preemptable RCU's dynticks interface,
Section~\ref{app:formal:Validating Preemptable RCU and dynticks},
and
Section~\ref{app:formal:Lessons (Re)Learned} lists
lessons (re)learned during this effort.

\subsection{Introduction to Preemptable RCU and dynticks}
\label{app:formal:Introduction to Preemptable RCU and dynticks}

The per-CPU \url{dynticks_progress_counter} variable is
central to the interface between dynticks and preemptable RCU.
This variable has an even value whenever the corresponding CPU
is in dynticks-idle mode, and an odd value otherwise.
A CPU exits dynticks-idle mode for the following three reasons:

\begin{enumerate}
\item	to start running a task,
\item	when entering the outermost of a possibly nested set of interrupt
	handlers, and
\item	when entering an NMI handler.
\end{enumerate}

Preemptable RCU's grace-period machinery samples the value of
the \url{dynticks_progress_counter} variable in order to
determine when a dynticks-idle CPU may safely be ignored.

The following three sections give an overview of the task
interface, the interrupt/NMI interface, and the use of
the \url{dynticks_progress_counter} variable by the
grace-period machinery.

\subsubsection{Task Interface}
\label{app:formal:Task Interface}

When a given CPU enters dynticks-idle mode because it has no more
tasks to run, it invokes \url{rcu_enter_nohz()}:

{ \scriptsize
\begin{verbatim}
  1 static inline void rcu_enter_nohz(void)
  2 {
  3   mb();
  4   __get_cpu_var(dynticks_progress_counter)++;
  5   WARN_ON(__get_cpu_var(dynticks_progress_counter) & 0x1);
  6 }
\end{verbatim}
}

This function simply increments \url{dynticks_progress_counter} and
checks that the result is even, but first executing a memory barrier
to ensure that any other CPU that sees the new value of
\url{dynticks_progress_counter} will also see the completion
of any prior RCU read-side critical sections.

Similarly, when a CPU that is in dynticks-idle mode prepares to
start executing a newly runnable task, it invokes
\url{rcu_exit_nohz}:

{ \scriptsize
\begin{verbatim}
  1 static inline void rcu_exit_nohz(void)
  2 {
  3   __get_cpu_var(dynticks_progress_counter)++;
  4   mb();
  5   WARN_ON(!(__get_cpu_var(dynticks_progress_counter) &
  6             0x1));
  7 }
\end{verbatim}
}

This function again increments \url{dynticks_progress_counter},
but follows it with a memory barrier to ensure that if any other CPU
sees the result of any subsequent RCU read-side critical section,
then that other CPU will also see the incremented value of
\url{dynticks_progress_counter}.
Finally, \url{rcu_exit_nohz()} checks that the result of the
increment is an odd value.

The \url{rcu_enter_nohz()} and \url{rcu_exit_nohz}
functions handle the case where a CPU enters and exits dynticks-idle
mode due to task execution, but does not handle interrupts, which are
covered in the following section.

\subsubsection{Interrupt Interface}
\label{app:formal:Interrupt Interface}

The \url{rcu_irq_enter()} and \url{rcu_irq_exit()}
functions handle interrupt/NMI entry and exit, respectively.
Of course, nested interrupts must also be properly accounted for.
The possibility of nested interrupts is handled by a second per-CPU
variable, \url{rcu_update_flag}, which is incremented upon
entry to an interrupt or NMI handler (in \url{rcu_irq_enter()})
and is decremented upon exit (in \url{rcu_irq_exit()}).
In addition, the pre-existing \url{in_interrupt()} primitive is
used to distinguish between an outermost or a nested interrupt/NMI.

Interrupt entry is handled by the \url{rcu_irq_enter}
shown below:

{ \scriptsize
\begin{verbatim}
  1 void rcu_irq_enter(void)
  2 {
  3   int cpu = smp_processor_id();
  4
  5   if (per_cpu(rcu_update_flag, cpu))
  6     per_cpu(rcu_update_flag, cpu)++;
  7   if (!in_interrupt() &&
  8       (per_cpu(dynticks_progress_counter,
  9                cpu) & 0x1) == 0) {
 10     per_cpu(dynticks_progress_counter, cpu)++;
 11     smp_mb();
 12     per_cpu(rcu_update_flag, cpu)++;
 13   }
 14 }
\end{verbatim}
}

Line~3 fetches the current CPU's number, while lines~5 and 6
increment the \url{rcu_update_flag} nesting counter if it
is already non-zero.
Lines~7-9 check to see whether we are the outermost level of
interrupt, and, if so, whether \url{dynticks_progress_counter}
needs to be incremented.
If so, line~10 increments \url{dynticks_progress_counter},
line~11 executes a memory barrier, and line~12 increments
\url{rcu_update_flag}.
As with \url{rcu_exit_nohz()}, the memory barrier ensures that
any other CPU that sees the effects of an RCU read-side critical section
in the interrupt handler (following the \url{rcu_irq_enter()}
invocation) will also see the increment of
\url{dynticks_progress_counter}.

\QuickQuiz{}
	Why not simply increment \url{rcu_update_flag}, and then only
	increment \url{dynticks_progress_counter} if the old value
	of \url{rcu_update_flag} was zero???
\QuickQuizAnswer{
	This fails in presence of NMIs.
	To see this, suppose an NMI was received just after
	\url{rcu_irq_enter()} incremented \url{rcu_update_flag},
	but before it incremented \url{dynticks_progress_counter}.
	The instance of \url{rcu_irq_enter()} invoked by the NMI
	would see that the original value of \url{rcu_update_flag}
	was non-zero, and would therefore refrain from incrementing
	\url{dynticks_progress_counter}.
	This would leave the RCU grace-period machinery no clue that the
	NMI handler was executing on this CPU, so that any RCU read-side
	critical sections in the NMI handler would lose their RCU protection.

	The possibility of NMI handlers, which, by definition cannot
	be masked, does complicate this code.
} \QuickQuizEnd

\QuickQuiz{}
	But if line~7 finds that we are the outermost interrupt,
	wouldn't we \emph{always} need to increment
	\url{dynticks_progress_counter}?
\QuickQuizAnswer{
	Not if we interrupted a running task!
	In that case, \url{dynticks_progress_counter} would
	have already been incremented by \url{rcu_exit_nohz()},
	and there would be no need to increment it again.
} \QuickQuizEnd

Interrupt exit is handled similarly by
\url{rcu_irq_exit()}:

{ \scriptsize
\begin{verbatim}
  1 void rcu_irq_exit(void)
  2 {
  3   int cpu = smp_processor_id();
  4
  5   if (per_cpu(rcu_update_flag, cpu)) {
  6     if (--per_cpu(rcu_update_flag, cpu))
  7       return;
  8     WARN_ON(in_interrupt());
  9     smp_mb();
 10     per_cpu(dynticks_progress_counter, cpu)++;
 11     WARN_ON(per_cpu(dynticks_progress_counter,
 12                     cpu) & 0x1);
 13   }
 14 }
\end{verbatim}
}

Line~3 fetches the current CPU's number, as before.
Line~5 checks to see if the \url{rcu_update_flag} is
non-zero, returning immediately (via falling off the end of the
function) if not.
Otherwise, lines~6 through 12 come into play.
Line~6 decrements \url{rcu_update_flag}, returning
if the result is not zero.
Line~8 verifies that we are indeed leaving the outermost
level of nested interrupts, line~9 executes a memory barrier,
line~10 increments \url{dynticks_progress_counter},
and lines~11 and 12 verify that this variable is now even.
As with \url{rcu_enter_nohz()}, the memory barrier ensures that
any other CPU that sees the increment of
\url{dynticks_progress_counter}
will also see the effects of an RCU read-side critical section
in the interrupt handler (preceding the \url{rcu_irq_exit()}
invocation).

These two sections have described how the
\url{dynticks_progress_counter} variable is maintained during
entry to and exit from dynticks-idle mode, both by tasks and by
interrupts and NMIs.
The following section describes how this variable is used by
preemptable RCU's grace-period machinery.

\subsubsection{Grace-Period Interface}
\label{app:formal:Grace-Period Interface}

Of the four preemptable RCU grace-period states shown in
Figure~\ref{app:rcuimpl:Preemptable RCU State Machine} on
page~\pageref{app:rcuimpl:Preemptable RCU State Machine} in
Appendix~\ref{app:rcuimpl:Preemptable RCU},
only the \url{rcu_try_flip_waitack_state()}
and \url{rcu_try_flip_waitmb_state()} states need to wait
for other CPUs to respond.

Of course, if a given CPU is in dynticks-idle state, we shouldn't
wait for it.
Therefore, just before entering one of these two states,
the preceding state takes a snapshot of each CPU's
\url{dynticks_progress_counter} variable, placing the
snapshot in another per-CPU variable,
\url{rcu_dyntick_snapshot}.
This is accomplished by invoking
\url{dyntick_save_progress_counter}, shown below:

{ \scriptsize
\begin{verbatim}
  1 static void dyntick_save_progress_counter(int cpu)
  2 {
  3   per_cpu(rcu_dyntick_snapshot, cpu) =
  4     per_cpu(dynticks_progress_counter, cpu);
  5 }
\end{verbatim}
}

The \url{rcu_try_flip_waitack_state()} state invokes
\url{rcu_try_flip_waitack_needed()}, shown below:

{ \scriptsize
\begin{verbatim}
  1 static inline int
  2 rcu_try_flip_waitack_needed(int cpu)
  3 {
  4   long curr;
  5   long snap;
  6
  7   curr = per_cpu(dynticks_progress_counter, cpu);
  8   snap = per_cpu(rcu_dyntick_snapshot, cpu);
  9   smp_mb();
 10   if ((curr == snap) && ((curr & 0x1) == 0))
 11     return 0;
 12   if ((curr - snap) > 2 || (snap & 0x1) == 0)
 13     return 0;
 14   return 1;
 15 }
\end{verbatim}
}

Lines~7 and 8 pick up current and snapshot versions of
\url{dynticks_progress_counter}, respectively.
The memory barrier on line~ensures that the counter checks
in the later \url{rcu_try_flip_waitzero_state} follow
the fetches of these counters.
Lines~10 and 11 return zero (meaning no communication with the
specified CPU is required) if that CPU has remained in dynticks-idle
state since the time that the snapshot was taken.
Similarly, lines~12 and 13 return zero if that CPU was initially
in dynticks-idle state or if it has completely passed through a
dynticks-idle state.
In both these cases, there is no way that that CPU could have retained
the old value of the grace-period counter.
If neither of these conditions hold, line~14 returns one, meaning
that the CPU needs to explicitly respond.

For its part, the \url{rcu_try_flip_waitmb_state} state
invokes \url{rcu_try_flip_waitmb_needed()}, shown below:

{ \scriptsize
\begin{verbatim}
  1 static inline int
  2 rcu_try_flip_waitmb_needed(int cpu)
  3 {
  4   long curr;
  5   long snap;
  6
  7   curr = per_cpu(dynticks_progress_counter, cpu);
  8   snap = per_cpu(rcu_dyntick_snapshot, cpu);
  9   smp_mb();
 10   if ((curr == snap) && ((curr & 0x1) == 0))
 11     return 0;
 12   if (curr != snap)
 13     return 0;
 14   return 1;
 15 }
\end{verbatim}
}

This is quite similar to \url{rcu_try_flip_waitack_needed},
the difference being in lines~12 and 13, because any transition
either to or from dynticks-idle state executes the memory barrier
needed by the \url{rcu_try_flip_waitmb_state()} state.

We now have seen all the code involved in the interface between
RCU and the dynticks-idle state.
The next section builds up the Promela model used to verify this
code.

\QuickQuiz{}
	Can you spot any bugs in any of the code in this section?
\QuickQuizAnswer{
	Read the next section to see if you were correct.
} \QuickQuizEnd

\subsection{Validating Preemptable RCU and dynticks}
\label{app:formal:Validating Preemptable RCU and dynticks}

This section develops a Promela model for the interface between
dynticks and RCU step by step, with each of the following sections
illustrating one step, starting with the process-level code,
adding assertions, interrupts, and finally NMIs.

\subsubsection{Basic Model}
\label{app:formal:Basic Model}

This section translates the process-level dynticks entry/exit
code and the grace-period processing into
Promela~\cite{Holzmann03a}.
We start with \url{rcu_exit_nohz()} and
\url{rcu_enter_nohz()}
from the 2.6.25-rc4 kernel, placing these in a single Promela
process that models exiting and entering dynticks-idle mode in
a loop as follows:

{ \scriptsize
\begin{verbatim}
  1 proctype dyntick_nohz()
  2 {
  3   byte tmp;
  4   byte i = 0;
  5
  6   do
  7   :: i >= MAX_DYNTICK_LOOP_NOHZ -> break;
  8   :: i < MAX_DYNTICK_LOOP_NOHZ ->
  9     tmp = dynticks_progress_counter;
 10     atomic {
 11       dynticks_progress_counter = tmp + 1;
 12       assert((dynticks_progress_counter & 1) == 1);
 13     }
 14     tmp = dynticks_progress_counter;
 15     atomic {
 16       dynticks_progress_counter = tmp + 1;
 17       assert((dynticks_progress_counter & 1) == 0);
 18     }
 19     i++;
 20   od;
 21 }
\end{verbatim}
}

Lines~6 and 20 define a loop.
Line~7 exits the loop once the loop counter \url{i}
has exceeded the limit \url{MAX_DYNTICK_LOOP_NOHZ}.
Line~8 tells the loop construct to execute lines~9-19
for each pass through the loop.
Because the conditionals on lines~7 and 8 are exclusive of
each other, the normal Promela random selection of true conditions
is disabled.
Lines~9 and 11 model \url{rcu_exit_nohz()}'s non-atomic
increment of \url{dynticks_progress_counter}, while
line 12 models the \url{WARN_ON()}.
The \url{atomic} construct simply reduces the Promela state space,
given that the \url{WARN_ON()} is not strictly speaking part
of the algorithm.
Lines~14-18 similarly models the increment and
\url{WARN_ON()} for \url{rcu_enter_nohz()}.
Finally, line~19 increments the loop counter.

Each pass through the loop therefore models a CPU exiting
dynticks-idle mode (for example, starting to execute a task), then
re-entering dynticks-idle mode (for example, that same task blocking).

\QuickQuiz{}
	Why isn't the memory barrier in \url{rcu_exit_nohz()}
	and \url{rcu_enter_nohz()} modeled in Promela?
\QuickQuizAnswer{
	Promela assumes sequential consistency, so
	it is not necessary to model memory barriers.
	In fact, one must instead explicitly model lack of memory barriers,
	for example, as shown in
	Figure~\ref{fig:analysis:QRCU Unordered Summation} on
	page~\pageref{fig:analysis:QRCU Unordered Summation}.
} \QuickQuizEnd

\QuickQuiz{}
	Isn't it a bit strange to model \url{rcu_exit_nohz()}
	followed by \url{rcu_enter_nohz()}?
	Wouldn't it be more natural to instead model entry before exit?
\QuickQuizAnswer{
	It probably would be more natural, but we will need
	this particular order for the liveness checks that we will add later.
} \QuickQuizEnd

The next step is to model the interface to RCU's grace-period
processing.
For this, we need to model
\url{dyntick_save_progress_counter()},
\url{rcu_try_flip_waitack_needed()},
\url{rcu_try_flip_waitmb_needed()},
as well as portions of
\url{rcu_try_flip_waitack()} and
\url{rcu_try_flip_waitmb()}, all from the 2.6.25-rc4 kernel.
The following \url{grace_period()} Promela process models
these functions as they would be invoked during a single pass
through preemptable RCU's grace-period processing.

{ \scriptsize
\begin{verbatim}
  1 proctype grace_period()
  2 {
  3   byte curr;
  4   byte snap;
  5
  6   atomic {
  7     printf("MDLN = %d\n", MAX_DYNTICK_LOOP_NOHZ);
  8     snap = dynticks_progress_counter;
  9   }
 10   do
 11   :: 1 ->
 12     atomic {
 13       curr = dynticks_progress_counter;
 14       if
 15       :: (curr == snap) && ((curr & 1) == 0) ->
 16         break;
 17       :: (curr - snap) > 2 || (snap & 1) == 0 ->
 18         break;
 19       :: 1 -> skip;
 20       fi;
 21     }
 22   od;
 23   snap = dynticks_progress_counter;
 24   do
 25   :: 1 ->
 26     atomic {
 27       curr = dynticks_progress_counter;
 28       if
 29       :: (curr == snap) && ((curr & 1) == 0) ->
 30         break;
 31       :: (curr != snap) ->
 32         break;
 33       :: 1 -> skip;
 34       fi;
 35     }
 36   od;
 37 }
\end{verbatim}
}

Lines~6-9 print out the loop limit (but only into the .trail file
in case of error) and models a line of code
from \url{rcu_try_flip_idle()} and its call to
\url{dyntick_save_progress_counter()}, which takes a
snapshot of the current CPU's \url{dynticks_progress_counter}
variable.
These two lines are executed atomically to reduce state space.

Lines~10-22 model the relevant code in
\url{rcu_try_flip_waitack()} and its call to
\url{rcu_try_flip_waitack_needed()}.
This loop is modeling the grace-period state machine waiting for
a counter-flip acknowledgement from each CPU, but only that part
that interacts with dynticks-idle CPUs.

Line~23 models a line from \url{rcu_try_flip_waitzero()}
and its call to \url{dyntick_save_progress_counter()}, again
taking a snapshot of the CPU's \url{dynticks_progress_counter}
variable.

Finally, lines~24-36 model the relevant code in
\url{rcu_try_flip_waitack()} and its call to
\url{rcu_try_flip_waitack_needed()}.
This loop is modeling the grace-period state-machine waiting for
each CPU to execute a memory barrier, but again only that part
that interacts with dynticks-idle CPUs.

\QuickQuiz{}
	Wait a minute!
	In the Linux kernel, both \url{dynticks_progress_counter} and
	\url{rcu_dyntick_snapshot} are per-CPU variables.
	So why are they instead being modeled as single global variables?
\QuickQuizAnswer{
	Because the grace-period code processes each
	CPU's \url{dynticks_progress_counter} and
	\url{rcu_dyntick_snapshot} variables separately,
	we can collapse the state onto a single CPU.
	If the grace-period code were instead to do something special
	given specific values on specific CPUs, then we would indeed need
	to model multiple CPUs.
	But fortunately, we can safely confine ourselves to two CPUs, the
	one running the grace-period processing and the one entering and
	leaving dynticks-idle mode.
} \QuickQuizEnd

The resulting model (\url{dyntickRCU-base.spin}),
when run with the
\url{runspin.sh} script,
generates 691 states and
passes without errors, which is not at all surprising given that
it completely lacks the assertions that could find failures.
The next section therefore adds safety assertions.

\subsubsection{Validating Safety}
\label{app:formal:Validating Safety}

A safe RCU implementation must never permit a grace period to
complete before the completion of any RCU readers that started
before the start of the grace period.
This is modeled by a \url{grace_period_state} variable that
can take on three states as follows:

\vspace{5pt}
\begin{minipage}[t]{\columnwidth}
\begin{verbatim}
  1 #define GP_IDLE    0
  2 #define GP_WAITING  1
  3 #define GP_DONE    2
  4 byte grace_period_state = GP_DONE;
\end{verbatim}
\end{minipage}
\vspace{5pt}

The \url{grace_period()} process sets this variable as it
progresses through the grace-period phases, as shown below:

{ \scriptsize
\begin{verbatim}
  1 proctype grace_period()
  2 {
  3   byte curr;
  4   byte snap;
  5
  6   grace_period_state = GP_IDLE;
  7   atomic {
  8     printf("MDLN = %d\n", MAX_DYNTICK_LOOP_NOHZ);
  9     snap = dynticks_progress_counter;
 10     grace_period_state = GP_WAITING;
 11   }
 12   do
 13   :: 1 ->
 14     atomic {
 15       curr = dynticks_progress_counter;
 16       if
 17       :: (curr == snap) && ((curr & 1) == 0) ->
 18         break;
 19       :: (curr - snap) > 2 || (snap & 1) == 0 ->
 20         break;
 21       :: 1 -> skip;
 22       fi;
 23     }
 24   od;
 25   grace_period_state = GP_DONE;
 26   grace_period_state = GP_IDLE;
 27   atomic {
 28     snap = dynticks_progress_counter;
 29     grace_period_state = GP_WAITING;
 30   }
 31   do
 32   :: 1 ->
 33     atomic {
 34       curr = dynticks_progress_counter;
 35       if
 36       :: (curr == snap) && ((curr & 1) == 0) ->
 37         break;
 38       :: (curr != snap) ->
 39         break;
 40       :: 1 -> skip;
 41       fi;
 42     }
 43   od;
 44   grace_period_state = GP_DONE;
 45 }
\end{verbatim}
}

Lines~6, 10, 25, 26, 29, and 44 update this variable (combining
atomically with algorithmic operations where feasible) to
allow the \url{dyntick_nohz()} process to verify the basic
RCU safety property.
The form of this verification is to assert that the value of the
\url{grace_period_state} variable cannot jump from
\url{GP_IDLE} to \url{GP_DONE} during a time period
over which RCU readers could plausibly persist.

\QuickQuiz{}
	Given there are a pair of back-to-back changes to
	\url{grace_period_state} on lines~25 and 26,
	how can we be sure that line~25's changes won't be lost?
\QuickQuizAnswer{
	Recall that Promela and spin trace out
	every possible sequence of state changes.
	Therefore, timing is irrelevant: Promela/spin will be quite
	happy to jam the entire rest of the model between those two
	statements unless some state variable specifically prohibits
	doing so.
} \QuickQuizEnd

The \url{dyntick_nohz()} Promela process implements
this verification as shown below:

{ \scriptsize
\begin{verbatim}
  1 proctype dyntick_nohz()
  2 {
  3   byte tmp;
  4   byte i = 0;
  5   bit old_gp_idle;
  6
  7   do
  8   :: i >= MAX_DYNTICK_LOOP_NOHZ -> break;
  9   :: i < MAX_DYNTICK_LOOP_NOHZ ->
 10     tmp = dynticks_progress_counter;
 11     atomic {
 12       dynticks_progress_counter = tmp + 1;
 13       old_gp_idle = (grace_period_state == GP_IDLE);
 14       assert((dynticks_progress_counter & 1) == 1);
 15     }
 16     atomic {
 17       tmp = dynticks_progress_counter;
 18       assert(!old_gp_idle ||
 19              grace_period_state != GP_DONE);
 20     }
 21     atomic {
 22       dynticks_progress_counter = tmp + 1;
 23       assert((dynticks_progress_counter & 1) == 0);
 24     }
 25     i++;
 26   od;
 27 }
\end{verbatim}
}

Line~13 sets a new \url{old_gp_idle} flag if the
value of the \url{grace_period_state} variable is
\url{GP_IDLE} at the beginning of task execution,
and the assertion at lines~18 and 19 fire if the \url{grace_period_state}
variable has advanced to \url{GP_DONE} during task execution,
which would be illegal given that a single RCU read-side critical
section could span the entire intervening time period.

The resulting
model (\url{dyntickRCU-base-s.spin}),
when run with the \url{runspin.sh} script,
generates 964 states and passes without errors, which is reassuring.
That said, although safety is critically important, it is also quite
important to avoid indefinitely stalling grace periods.
The next section therefore covers verifying liveness.

\subsubsection{Validating Liveness}
\label{app:formal:Validating Liveness}

Although liveness can be difficult to prove, there is a simple
trick that applies here.
The first step is to make \url{dyntick_nohz()} indicate that
it is done via a \url{dyntick_nohz_done} variable, as shown on
line~27 of the following:

{ \scriptsize
\begin{verbatim}
  1 proctype dyntick_nohz()
  2 {
  3   byte tmp;
  4   byte i = 0;
  5   bit old_gp_idle;
  6
  7   do
  8   :: i >= MAX_DYNTICK_LOOP_NOHZ -> break;
  9   :: i < MAX_DYNTICK_LOOP_NOHZ ->
 10     tmp = dynticks_progress_counter;
 11     atomic {
 12       dynticks_progress_counter = tmp + 1;
 13       old_gp_idle = (grace_period_state == GP_IDLE);
 14       assert((dynticks_progress_counter & 1) == 1);
 15     }
 16     atomic {
 17       tmp = dynticks_progress_counter;
 18       assert(!old_gp_idle ||
 19              grace_period_state != GP_DONE);
 20     }
 21     atomic {
 22       dynticks_progress_counter = tmp + 1;
 23       assert((dynticks_progress_counter & 1) == 0);
 24     }
 25     i++;
 26   od;
 27   dyntick_nohz_done = 1;
 28 }
\end{verbatim}
}

With this variable in place, we can add assertions to
\url{grace_period()} to check for unnecessary blockage
as follows:

{ \scriptsize
\begin{verbatim}
  1 proctype grace_period()
  2 {
  3   byte curr;
  4   byte snap;
  5   bit shouldexit;
  6
  7   grace_period_state = GP_IDLE;
  8   atomic {
  9     printf("MDLN = %d\n", MAX_DYNTICK_LOOP_NOHZ);
 10     shouldexit = 0;
 11     snap = dynticks_progress_counter;
 12     grace_period_state = GP_WAITING;
 13   }
 14   do
 15   :: 1 ->
 16     atomic {
 17       assert(!shouldexit);
 18       shouldexit = dyntick_nohz_done;
 19       curr = dynticks_progress_counter;
 20       if
 21       :: (curr == snap) && ((curr & 1) == 0) ->
 22         break;
 23       :: (curr - snap) > 2 || (snap & 1) == 0 ->
 24         break;
 25       :: else -> skip;
 26       fi;
 27     }
 28   od;
 29   grace_period_state = GP_DONE;
 30   grace_period_state = GP_IDLE;
 31   atomic {
 32     shouldexit = 0;
 33     snap = dynticks_progress_counter;
 34     grace_period_state = GP_WAITING;
 35   }
 36   do
 37   :: 1 ->
 38     atomic {
 39       assert(!shouldexit);
 40       shouldexit = dyntick_nohz_done;
 41       curr = dynticks_progress_counter;
 42       if
 43       :: (curr == snap) && ((curr & 1) == 0) ->
 44         break;
 45       :: (curr != snap) ->
 46         break;
 47       :: else -> skip;
 48       fi;
 49     }
 50   od;
 51   grace_period_state = GP_DONE;
 52 }
\end{verbatim}
}

We have added the \url{shouldexit} variable on line~5,
which we initialize to zero on line~10.
Line~17 asserts that \url{shouldexit} is not set, while
line~18 sets \url{shouldexit} to the \url{dyntick_nohz_done}
variable maintained by \url{dyntick_nohz()}.
This assertion will therefore trigger if we attempt to take more than
one pass through the wait-for-counter-flip-acknowledgement
loop after \url{dyntick_nohz()} has completed
execution.
After all, if \url{dyntick_nohz()} is done, then there cannot be
any more state changes to force us out of the loop, so going through twice
in this state means an infinite loop, which in turn means no end to the
grace period.

Lines~32, 39, and 40 operate in a similar manner for the
second (memory-barrier) loop.

However, running this
model (\url{dyntickRCU-base-sl-busted.spin})
results in failure, as line~23 is checking that the wrong variable
is even.
Upon failure, \url{spin} writes out a
``trail'' file
(\url{dyntickRCU-base-sl-busted.spin.trail})
file, which records the sequence of states that lead to the failure.
Use the {\tt spin -t -p -g -l dyntickRCU-base-sl-busted.spin}
command to cause \url{spin} to retrace this sequence of state,
printing the statements executed and the values of variables
(\url{dyntickRCU-base-sl-busted.spin.trail.txt}).
Note that the line numbers do not match the listing above due to
the fact that spin takes both functions in a single file.
However, the line numbers \emph{do} match the full
model (\url{dyntickRCU-base-sl-busted.spin}).

We see that the \url{dyntick_nohz()} process completed
at step 34 (search for ``34:''), but that the
\url{grace_period()} process nonetheless failed to exit the loop.
The value of \url{curr} is \url{6} (see step 35)
and that the value of \url{snap} is \url{5} (see step 17).
Therefore the first condition on line~21 above does not hold because
\url{curr != snap}, and the second condition on line~23
does not hold either because \url{snap} is odd and because
\url{curr} is only one greater than \url{snap}.

So one of these two conditions has to be incorrect.
Referring to the comment block in \url{rcu_try_flip_waitack_needed()}
for the first condition:

\begin{quote}
	If the CPU remained in dynticks mode for the entire time
	and didn't take any interrupts, NMIs, SMIs, or whatever,
	then it cannot be in the middle of an \url{rcu_read_lock()}, so
	the next \url{rcu_read_lock()} it executes must use the new value
	of the counter.  So we can safely pretend that this CPU
	already acknowledged the counter.
\end{quote}

The first condition does match this, because if \url{curr == snap}
and if \url{curr} is even, then the corresponding CPU has been
in dynticks-idle mode the entire time, as required.
So let's look at the comment block for the second condition:

\begin{quote}
	If the CPU passed through or entered a dynticks idle phase with
	no active irq handlers, then, as above, we can safely pretend
	that this CPU already acknowledged the counter.
\end{quote}

The first part of the condition is correct, because if \url{curr}
and \url{snap} differ by two, there will be at least one even
number in between, corresponding to having passed completely through
a dynticks-idle phase.
However, the second part of the condition corresponds to having
\emph{started} in dynticks-idle mode, not having \emph{finished}
in this mode.
We therefore need to be testing \url{curr} rather than
\url{snap} for being an even number.

The corrected C code is as follows:

{ \scriptsize
\begin{verbatim}
  1 static inline int
  2 rcu_try_flip_waitack_needed(int cpu)
  3 {
  4   long curr;
  5   long snap;
  6
  7   curr = per_cpu(dynticks_progress_counter, cpu);
  8   snap = per_cpu(rcu_dyntick_snapshot, cpu);
  9   smp_mb();
 10   if ((curr == snap) && ((curr & 0x1) == 0))
 11     return 0;
 12   if ((curr - snap) > 2 || (curr & 0x1) == 0)
 13     return 0;
 14   return 1;
 15 }
\end{verbatim}
}

Lines~10-13 can now be combined and simplified,
resulting in the following.
A similar simplification can be applied to
\url{rcu_try_flip_waitmb_needed}.

{ \scriptsize
\begin{verbatim}
  1 static inline int
  2 rcu_try_flip_waitack_needed(int cpu)
  3 {
  4   long curr;
  5   long snap;
  6
  7   curr = per_cpu(dynticks_progress_counter, cpu);
  8   snap = per_cpu(rcu_dyntick_snapshot, cpu);
  9   smp_mb();
 10   if ((curr - snap) >= 2 || (curr & 0x1) == 0)
 11     return 0;
 12   return 1;
 13 }
\end{verbatim}
}

Making the corresponding correction in the
model (\url{dyntickRCU-base-sl.spin})
results in a correct verification with 661 states that passes without
errors.
However, it is worth noting that the first version of the liveness
verification failed to catch this bug, due to a bug in the liveness
verification itself.
This liveness-verification bug was located by inserting an infinite
loop in the \url{grace_period()} process, and noting that
the liveness-verification code failed to detect this problem!

We have now successfully verified both safety and liveness
conditions, but only for processes running and blocking.
We also need to handle interrupts, a task taken up in the next section.

\subsubsection{Interrupts}
\label{app:formal:Interrupts}

There are a couple of ways to model interrupts in Promela:
\begin{enumerate}
\item	using C-preprocessor tricks to insert the interrupt handler
	between each and every statement of the \url{dynticks_nohz()}
	process, or
\item	modeling the interrupt handler with a separate process.
\end{enumerate}

A bit of thought indicated that the second approach would have a
smaller state space, though it requires that the interrupt handler
somehow run atomically with respect to the \url{dynticks_nohz()}
process, but not with respect to the \url{grace_period()}
process.

Fortunately, it turns out that Promela permits you to branch
out of atomic statements.
This trick allows us to have the interrupt handler set a flag, and
recode \url{dynticks_nohz()} to atomically check this flag
and execute only when the flag is not set.
This can be accomplished with a C-preprocessor macro that takes
a label and a Promela statement as follows:

{ \scriptsize
\begin{verbatim}
  1 #define EXECUTE_MAINLINE(label, stmt) \
  2 label: skip; \
  3     atomic { \
  4       if \
  5       :: in_dyntick_irq -> goto label; \
  6       :: else -> stmt; \
  7       fi; \
  8     } \
\end{verbatim}
}

One might use this macro as follows:

\vspace{5pt}
\begin{minipage}[t]{\columnwidth}
\scriptsize
\begin{verbatim}
EXECUTE_MAINLINE(stmt1,
                 tmp = dynticks_progress_counter)
\end{verbatim}
\end{minipage}
\vspace{5pt}

Line~2 of the macro creates the specified statement label.
Lines~3-8 are an atomic block that tests the \url{in_dyntick_irq}
variable, and if this variable is set (indicating that the interrupt
handler is active), branches out of the atomic block back to the
label.
Otherwise, line~6 executes the specified statement.
The overall effect is that mainline execution stalls any time an interrupt
is active, as required.

\subsubsection{Validating Interrupt Handlers}
\label{app:formal:Validating Interrupt Handlers}

The first step is to convert \url{dyntick_nohz()} to
\url{EXECUTE_MAINLINE()} form, as follows:

{ \scriptsize
\begin{verbatim}
  1 proctype dyntick_nohz()
  2 {
  3   byte tmp;
  4   byte i = 0;
  5   bit old_gp_idle;
  6
  7   do
  8   :: i >= MAX_DYNTICK_LOOP_NOHZ -> break;
  9   :: i < MAX_DYNTICK_LOOP_NOHZ ->
 10     EXECUTE_MAINLINE(stmt1,
 11       tmp = dynticks_progress_counter)
 12     EXECUTE_MAINLINE(stmt2,
 13       dynticks_progress_counter = tmp + 1;
 14       old_gp_idle = (grace_period_state == GP_IDLE);
 15       assert((dynticks_progress_counter & 1) == 1))
 16     EXECUTE_MAINLINE(stmt3,
 17       tmp = dynticks_progress_counter;
 18       assert(!old_gp_idle ||
 19              grace_period_state != GP_DONE))
 20     EXECUTE_MAINLINE(stmt4,
 21       dynticks_progress_counter = tmp + 1;
 22       assert((dynticks_progress_counter & 1) == 0))
 23     i++;
 24   od;
 25   dyntick_nohz_done = 1;
 26 }
\end{verbatim}
}

It is important to note that when a group of statements is passed
to \url{EXECUTE_MAINLINE()}, as in lines~11-14, all
statements in that group execute atomically.

\QuickQuiz{}
	But what would you do if you needed the statements in a single
	\url{EXECUTE_MAINLINE()} group to execute non-atomically?
\QuickQuizAnswer{
	The easiest thing to do would be to put
	each such statement in its own \url{EXECUTE_MAINLINE()}
	statement.
} \QuickQuizEnd

\QuickQuiz{}
	But what if the \url{dynticks_nohz()} process had
	``if'' or ``do'' statements with conditions,
	where the statement bodies of these constructs
	needed to execute non-atomically?
\QuickQuizAnswer{
	One approach, as we will see in a later section,
	is to use explicit labels and ``goto'' statements.
	For example, the construct:

	\vspace{5pt}
	\begin{minipage}[t]{\columnwidth}
	\scriptsize
	\begin{verbatim}
		if
		:: i == 0 -> a = -1;
		:: else -> a = -2;
		fi;
	\end{verbatim}
	\end{minipage}
	\vspace{5pt}

	could be modeled as something like:

	\vspace{5pt}
	\begin{minipage}[t]{\columnwidth}
	\scriptsize
	\begin{verbatim}
		EXECUTE_MAINLINE(stmt1,
				 if
				 :: i == 0 -> goto stmt1_then;
				 :: else -> goto stmt1_else;
				 fi)
		stmt1_then: skip;
		EXECUTE_MAINLINE(stmt1_then1, a = -1; goto stmt1_end)
		stmt1_else: skip;
		EXECUTE_MAINLINE(stmt1_then1, a = -2)
		stmt1_end: skip;
	\end{verbatim}
	\end{minipage}
	\vspace{5pt}

	However, it is not clear that the macro is helping much in the case
	of the ``if'' statement, so these sorts of situations will
	be open-coded in the following sections.
} \QuickQuizEnd

The next step is to write a \url{dyntick_irq()} process
to model an interrupt handler:

{ \scriptsize
\begin{verbatim}
  1 proctype dyntick_irq()
  2 {
  3   byte tmp;
  4   byte i = 0;
  5   bit old_gp_idle;
  6
  7   do
  8   :: i >= MAX_DYNTICK_LOOP_IRQ -> break;
  9   :: i < MAX_DYNTICK_LOOP_IRQ ->
 10     in_dyntick_irq = 1;
 11     if
 12     :: rcu_update_flag > 0 ->
 13        tmp = rcu_update_flag;
 14       rcu_update_flag = tmp + 1;
 15     :: else -> skip;
 16     fi;
 17     if
 18     :: !in_interrupt &&
 19       (dynticks_progress_counter & 1) == 0 ->
 20       tmp = dynticks_progress_counter;
 21       dynticks_progress_counter = tmp + 1;
 22       tmp = rcu_update_flag;
 23       rcu_update_flag = tmp + 1;
 24     :: else -> skip;
 25     fi;
 26     tmp = in_interrupt;
 27     in_interrupt = tmp + 1;
 28     old_gp_idle = (grace_period_state == GP_IDLE);
 29     assert(!old_gp_idle || grace_period_state != GP_DONE);
 30     tmp = in_interrupt;
 31     in_interrupt = tmp - 1;
 32     if
 33     :: rcu_update_flag != 0 ->
 34       tmp = rcu_update_flag;
 35       rcu_update_flag = tmp - 1;
 36       if
 37       :: rcu_update_flag == 0 ->
 38         tmp = dynticks_progress_counter;
 39         dynticks_progress_counter = tmp + 1;
 40       :: else -> skip;
 41       fi;
 42     :: else -> skip;
 43     fi;
 44     atomic {
 45       in_dyntick_irq = 0;
 46       i++;
 47     }
 48   od;
 49   dyntick_irq_done = 1;
 50 }
\end{verbatim}
}

The loop from line~7-48 models up to \url{MAX_DYNTICK_LOOP_IRQ}
interrupts, with lines~8 and 9 forming the loop condition and line~45
incrementing the control variable.
Line~10 tells \url{dyntick_nohz()} that an interrupt handler
is running, and line~45 tells \url{dyntick_nohz()} that this
handler has completed.
Line~49 is used for liveness verification, much as is the corresponding
line of \url{dyntick_nohz()}.

\QuickQuiz{}
	Why are lines~45 and 46 (the \url{in_dyntick_irq = 0;}
	and the \url{i++;}) executed atomically?
\QuickQuizAnswer{
	These lines of code pertain to controlling the
	model, not to the code being modeled, so there is no reason to
	model them non-atomically.
	The motivation for modeling them atomically is to reduce the size
	of the state space.
} \QuickQuizEnd

Lines~11-25 model \url{rcu_irq_enter()}, and
lines~26 and 27 model the relevant snippet of \url{__irq_enter()}.
Lines~28 and 29 verifies safety in much the same manner as do the
corresponding lines of \url{dynticks_nohz()}.
Lines~30 and 31 model the relevant snippet of \url{__irq_exit()},
and finally lines~32-43 model \url{rcu_irq_exit()}.

\QuickQuiz{}
	What property of interrupts is this \url{dynticks_irq()}
	process unable to model?
\QuickQuizAnswer{
	One such property is nested interrupts,
	which are handled in the following section.
} \QuickQuizEnd

The \url{grace_period} process then becomes as follows:

{ \scriptsize
\begin{verbatim}
  1 proctype grace_period()
  2 {
  3   byte curr;
  4   byte snap;
  5   bit shouldexit;
  6
  7   grace_period_state = GP_IDLE;
  8   atomic {
  9     printf("MDLN = %d\n", MAX_DYNTICK_LOOP_NOHZ);
 10     printf("MDLI = %d\n", MAX_DYNTICK_LOOP_IRQ);
 11     shouldexit = 0;
 12     snap = dynticks_progress_counter;
 13     grace_period_state = GP_WAITING;
 14   }
 15   do
 16   :: 1 ->
 17     atomic {
 18       assert(!shouldexit);
 19       shouldexit = dyntick_nohz_done && dyntick_irq_done;
 20       curr = dynticks_progress_counter;
 21       if
 22       :: (curr - snap) >= 2 || (curr & 1) == 0 ->
 23         break;
 24       :: else -> skip;
 25       fi;
 26     }
 27   od;
 28   grace_period_state = GP_DONE;
 29   grace_period_state = GP_IDLE;
 30   atomic {
 31     shouldexit = 0;
 32     snap = dynticks_progress_counter;
 33     grace_period_state = GP_WAITING;
 34   }
 35   do
 36   :: 1 ->
 37     atomic {
 38       assert(!shouldexit);
 39       shouldexit = dyntick_nohz_done && dyntick_irq_done;
 40       curr = dynticks_progress_counter;
 41       if
 42       :: (curr != snap) || ((curr & 1) == 0) ->
 43         break;
 44       :: else -> skip;
 45       fi;
 46     }
 47   od;
 48   grace_period_state = GP_DONE;
 49 }
\end{verbatim}
}

The implementation of \url{grace_period()} is very similar
to the earlier one.
The only changes are the addition of line~10 to add the new
interrupt-count parameter, changes to lines~19 and 39 to
add the new \url{dyntick_irq_done} variable to the liveness
checks, and of course the optimizations on lines~22 and 42.

This model (\url{dyntickRCU-irqnn-ssl.spin})
results in a correct verification with roughly half a million
states, passing without errors.
However, this version of the model does not handle nested
interrupts.
This topic is taken up in the nest section.

\subsubsection{Validating Nested Interrupt Handlers}
\label{app:formal:Validating Nested Interrupt Handlers}

Nested interrupt handlers may be modeled by splitting the body of
the loop in \url{dyntick_irq()} as follows:

{ \scriptsize
\begin{verbatim}
  1 proctype dyntick_irq()
  2 {
  3   byte tmp;
  4   byte i = 0;
  5   byte j = 0;
  6   bit old_gp_idle;
  7   bit outermost;
  8
  9   do
 10   :: i >= MAX_DYNTICK_LOOP_IRQ &&
 11      j >= MAX_DYNTICK_LOOP_IRQ -> break;
 12   :: i < MAX_DYNTICK_LOOP_IRQ ->
 13     atomic {
 14       outermost = (in_dyntick_irq == 0);
 15       in_dyntick_irq = 1;
 16     }
 17     if
 18     :: rcu_update_flag > 0 ->
 19       tmp = rcu_update_flag;
 20       rcu_update_flag = tmp + 1;
 21     :: else -> skip;
 22     fi;
 23     if
 24     :: !in_interrupt &&
 25        (dynticks_progress_counter & 1) == 0 ->
 26       tmp = dynticks_progress_counter;
 27       dynticks_progress_counter = tmp + 1;
 28       tmp = rcu_update_flag;
 29       rcu_update_flag = tmp + 1;
 30     :: else -> skip;
 31     fi;
 32     tmp = in_interrupt;
 33     in_interrupt = tmp + 1;
 34     atomic {
 35       if
 36       :: outermost ->
 37         old_gp_idle = (grace_period_state == GP_IDLE);
 38       :: else -> skip;
 39       fi;
 40     }
 41     i++;
 42   :: j < i ->
 43     atomic {
 44       if
 45       :: j + 1 == i ->
 46         assert(!old_gp_idle ||
 47                grace_period_state != GP_DONE);
 48       :: else -> skip;
 49       fi;
 50     }
 51     tmp = in_interrupt;
 52     in_interrupt = tmp - 1;
 53     if
 54     :: rcu_update_flag != 0 ->
 55       tmp = rcu_update_flag;
 56       rcu_update_flag = tmp - 1;
 57       if
 58       :: rcu_update_flag == 0 ->
 59         tmp = dynticks_progress_counter;
 60         dynticks_progress_counter = tmp + 1;
 61       :: else -> skip;
 62       fi;
 63     :: else -> skip;
 64     fi;
 65     atomic {
 66       j++;
 67       in_dyntick_irq = (i != j);
 68     }
 69   od;
 70   dyntick_irq_done = 1;
 71 }
\end{verbatim}
}

This is similar to the earlier \url{dynticks_irq()} process.
It adds a second counter variable \url{j} on line~5, so that
\url{i} counts entries to interrupt handlers and \url{j}
counts exits.
The \url{outermost} variable on line~7 helps determine
when the \url{grace_period_state} variable needs to be sampled
for the safety checks.
The loop-exit check on lines~10 and 11 is updated to require that the
specified number of interrupt handlers are exited as well as entered,
and the increment of \url{i} is moved to line~41, which is
the end of the interrupt-entry model.
Lines~13-16 set the \url{outermost} variable to indicate
whether this is the outermost of a set of nested interrupts and to
set the \url{in_dyntick_irq} variable that is used by the
\url{dyntick_nohz()} process.
Lines~34-40 capture the state of the \url{grace_period_state}
variable, but only when in the outermost interrupt handler.

Line~42 has the do-loop conditional for interrupt-exit modeling:
as long as we have exited fewer interrupts than we have entered, it is
legal to exit another interrupt.
Lines~43-50 check the safety criterion, but only if we are exiting
from the outermost interrupt level.
Finally, lines~65-68 increment the interrupt-exit count \url{j}
and, if this is the outermost interrupt level, clears
\url{in_dyntick_irq}.

This model (\url{dyntickRCU-irq-ssl.spin})
results in a correct verification with a bit more than half a million
states, passing without errors.
However, this version of the model does not handle NMIs,
which are taken up in the nest section.

\subsubsection{Validating NMI Handlers}
\label{app:formal:Validating NMI Handlers}

We take the same general approach for NMIs as we do for interrupts,
keeping in mind that NMIs do not nest.
This results in a \url{dyntick_nmi()} process as follows:

{ \scriptsize
\begin{verbatim}
  1 proctype dyntick_nmi()
  2 {
  3   byte tmp;
  4   byte i = 0;
  5   bit old_gp_idle;
  6
  7   do
  8   :: i >= MAX_DYNTICK_LOOP_NMI -> break;
  9   :: i < MAX_DYNTICK_LOOP_NMI ->
 10     in_dyntick_nmi = 1;
 11     if
 12     :: rcu_update_flag > 0 ->
 13       tmp = rcu_update_flag;
 14       rcu_update_flag = tmp + 1;
 15     :: else -> skip;
 16     fi;
 17     if
 18     :: !in_interrupt &&
 19        (dynticks_progress_counter & 1) == 0 ->
 20       tmp = dynticks_progress_counter;
 21       dynticks_progress_counter = tmp + 1;
 22       tmp = rcu_update_flag;
 23       rcu_update_flag = tmp + 1;
 24     :: else -> skip;
 25     fi;
 26     tmp = in_interrupt;
 27     in_interrupt = tmp + 1;
 28     old_gp_idle = (grace_period_state == GP_IDLE);
 29     assert(!old_gp_idle || grace_period_state != GP_DONE);
 30     tmp = in_interrupt;
 31     in_interrupt = tmp - 1;
 32     if
 33     :: rcu_update_flag != 0 ->
 34       tmp = rcu_update_flag;
 35       rcu_update_flag = tmp - 1;
 36       if
 37       :: rcu_update_flag == 0 ->
 38         tmp = dynticks_progress_counter;
 39         dynticks_progress_counter = tmp + 1;
 40       :: else -> skip;
 41       fi;
 42     :: else -> skip;
 43     fi;
 44     atomic {
 45       i++;
 46       in_dyntick_nmi = 0;
 47     }
 48   od;
 49   dyntick_nmi_done = 1;
 50 }
\end{verbatim}
}

Of course, the fact that we have NMIs requires adjustments in
the other components.
For example, the \url{EXECUTE_MAINLINE()} macro now needs to
pay attention to the NMI handler (\url{in_dyntick_nmi}) as well
as the interrupt handler (\url{in_dyntick_irq}) by checking
the \url{dyntick_nmi_done} variable as follows:

{ \scriptsize
\begin{verbatim}
  1 #define EXECUTE_MAINLINE(label, stmt) \
  2 label: skip; \
  3     atomic { \
  4       if \
  5       :: in_dyntick_irq || \
  6          in_dyntick_nmi -> goto label; \
  7       :: else -> stmt; \
  8       fi; \
  9     } \
\end{verbatim}
}

We will also need to introduce an \url{EXECUTE_IRQ()}
macro that checks \url{in_dyntick_nmi} in order to allow
\url{dyntick_irq()} to exclude \url{dyntick_nmi()}:

{ \scriptsize
\begin{verbatim}
  1 #define EXECUTE_IRQ(label, stmt) \
  2 label: skip; \
  3     atomic { \
  4       if \
  5       :: in_dyntick_nmi -> goto label; \
  6       :: else -> stmt; \
  7       fi; \
  8     } \
\end{verbatim}
}

It is further necessary to convert \url{dyntick_irq()}
to \url{EXECUTE_IRQ()} as follows:

{ \scriptsize
\begin{verbatim}
  1 proctype dyntick_irq()
  2 {
  3   byte tmp;
  4   byte i = 0;
  5   byte j = 0;
  6   bit old_gp_idle;
  7   bit outermost;
  8
  9   do
 10   :: i >= MAX_DYNTICK_LOOP_IRQ &&
 11      j >= MAX_DYNTICK_LOOP_IRQ -> break;
 12   :: i < MAX_DYNTICK_LOOP_IRQ ->
 13     atomic {
 14       outermost = (in_dyntick_irq == 0);
 15       in_dyntick_irq = 1;
 16     }
 17 stmt1: skip;
 18     atomic {
 19       if
 20       :: in_dyntick_nmi -> goto stmt1;
 21       :: !in_dyntick_nmi && rcu_update_flag ->
 22         goto stmt1_then;
 23       :: else -> goto stmt1_else;
 24       fi;
 25     }
 26 stmt1_then: skip;
 27     EXECUTE_IRQ(stmt1_1, tmp = rcu_update_flag)
 28     EXECUTE_IRQ(stmt1_2, rcu_update_flag = tmp + 1)
 29 stmt1_else: skip;
 30 stmt2: skip;  atomic {
 31       if
 32       :: in_dyntick_nmi -> goto stmt2;
 33       :: !in_dyntick_nmi &&
 34          !in_interrupt &&
 35          (dynticks_progress_counter & 1) == 0 ->
 36            goto stmt2_then;
 37       :: else -> goto stmt2_else;
 38       fi;
 39     }
 40 stmt2_then: skip;
 41     EXECUTE_IRQ(stmt2_1, tmp = dynticks_progress_counter)
 42     EXECUTE_IRQ(stmt2_2,
 43       dynticks_progress_counter = tmp + 1)
 44     EXECUTE_IRQ(stmt2_3, tmp = rcu_update_flag)
 45     EXECUTE_IRQ(stmt2_4, rcu_update_flag = tmp + 1)
 46 stmt2_else: skip;
 47     EXECUTE_IRQ(stmt3, tmp = in_interrupt)
 48     EXECUTE_IRQ(stmt4, in_interrupt = tmp + 1)
 49 stmt5: skip;
 50     atomic {
 51       if
 52       :: in_dyntick_nmi -> goto stmt4;
 53       :: !in_dyntick_nmi && outermost ->
 54         old_gp_idle = (grace_period_state == GP_IDLE);
 55       :: else -> skip;
 56       fi;
 57     }
 58     i++;
 59   :: j < i ->
 60 stmt6: skip;
 61     atomic {
 62       if
 63       :: in_dyntick_nmi -> goto stmt6;
 64       :: !in_dyntick_nmi && j + 1 == i ->
 65         assert(!old_gp_idle ||
 66                grace_period_state != GP_DONE);
 67       :: else -> skip;
 68       fi;
 69     }
 70     EXECUTE_IRQ(stmt7, tmp = in_interrupt);
 71     EXECUTE_IRQ(stmt8, in_interrupt = tmp - 1);
 72
 73 stmt9: skip;
 74     atomic {
 75       if
 76       :: in_dyntick_nmi -> goto stmt9;
 77       :: !in_dyntick_nmi && rcu_update_flag != 0 ->
 78         goto stmt9_then;
 79       :: else -> goto stmt9_else;
 80       fi;
 81     }
 82 stmt9_then: skip;
 83     EXECUTE_IRQ(stmt9_1, tmp = rcu_update_flag)
 84     EXECUTE_IRQ(stmt9_2, rcu_update_flag = tmp - 1)
 85 stmt9_3: skip;
 86     atomic {
 87       if
 88       :: in_dyntick_nmi -> goto stmt9_3;
 89       :: !in_dyntick_nmi && rcu_update_flag == 0 ->
 90         goto stmt9_3_then;
 91       :: else -> goto stmt9_3_else;
 92       fi;
 93     }
 94 stmt9_3_then: skip;
 95     EXECUTE_IRQ(stmt9_3_1,
 96       tmp = dynticks_progress_counter)
 97     EXECUTE_IRQ(stmt9_3_2,
 98       dynticks_progress_counter = tmp + 1)
 99 stmt9_3_else:
100 stmt9_else: skip;
101     atomic {
102       j++;
103       in_dyntick_irq = (i != j);
104     }
105   od;
106   dyntick_irq_done = 1;
107 }
\end{verbatim}
}

Note that we have open-coded the ``if'' statements
(for example, lines~17-29).
In addition, statements that process strictly local state
(such as line~58) need not exclude \url{dyntick_nmi()}.

Finally, \url{grace_period()} requires only a few changes:

{ \scriptsize
\begin{verbatim}
  1 proctype grace_period()
  2 {
  3   byte curr;
  4   byte snap;
  5   bit shouldexit;
  6
  7   grace_period_state = GP_IDLE;
  8   atomic {
  9     printf("MDLN = %d\n", MAX_DYNTICK_LOOP_NOHZ);
 10     printf("MDLI = %d\n", MAX_DYNTICK_LOOP_IRQ);
 11     printf("MDLN = %d\n", MAX_DYNTICK_LOOP_NMI);
 12     shouldexit = 0;
 13     snap = dynticks_progress_counter;
 14     grace_period_state = GP_WAITING;
 15   }
 16   do
 17   :: 1 ->
 18     atomic {
 19       assert(!shouldexit);
 20       shouldexit = dyntick_nohz_done &&
 21              dyntick_irq_done &&
 22              dyntick_nmi_done;
 23       curr = dynticks_progress_counter;
 24       if
 25       :: (curr - snap) >= 2 || (curr & 1) == 0 ->
 26         break;
 27       :: else -> skip;
 28       fi;
 29     }
 30   od;
 31   grace_period_state = GP_DONE;
 32   grace_period_state = GP_IDLE;
 33   atomic {
 34     shouldexit = 0;
 35     snap = dynticks_progress_counter;
 36     grace_period_state = GP_WAITING;
 37   }
 38   do
 39   :: 1 ->
 40     atomic {
 41       assert(!shouldexit);
 42       shouldexit = dyntick_nohz_done &&
 43              dyntick_irq_done &&
 44              dyntick_nmi_done;
 45       curr = dynticks_progress_counter;
 46       if
 47       :: (curr != snap) || ((curr & 1) == 0) ->
 48         break;
 49       :: else -> skip;
 50       fi;
 51     }
 52   od;
 53   grace_period_state = GP_DONE;
 54 }
\end{verbatim}
}

We have added the \url{printf()} for the new
\url{MAX_DYNTICK_LOOP_NMI} parameter on line~11 and
added \url{dyntick_nmi_done} to the \url{shouldexit}
assignments on lines~22 and 44.

The model (\url{dyntickRCU-irq-nmi-ssl.spin})
results in a correct verification with several hundred million
states, passing without errors.

\QuickQuiz{}
	Does Paul always write his code in this painfully incremental
	manner???
\QuickQuizAnswer{
	Not always, but more and more frequently.
	In this case, Paul started with the smallest slice of code that
	included an interrupt handler, because he was not sure how best
	to model interrupts in Promela.
	Once he got that working, he added other features.
	(But if he was doing it again, he would start with a ``toy'' handler.
	For example, he might have the handler increment a variable twice and
	have the mainline code verify that the value was always even.)

	Why the incremental approach?
	Consider the following, attributed to Brian W. Kernighan:

	\begin{quote}
		Debugging is twice as hard as writing the code in the first
		place. Therefore, if you write the code as cleverly as possible,
		you are, by definition, not smart enough to debug it.
	\end{quote}

	This means that any attempt to optimize the production of code should
	place at least 66\% of its emphasis on optimizing the debugging process,
	even at the expense of increasing the time and effort spent coding.
	Incremental coding and testing is one way to optimize the debugging
	process, at the expense of some increase in coding effort.
	Paul uses this approach because he rarely has the luxury of
	devoting full days (let alone weeks) to coding and debugging.
} \QuickQuizEnd

\subsection{Lessons (Re)Learned}
\label{app:formal:Lessons (Re)Learned}

\begin{figure}[tbp]
{ \scriptsize
\begin{verbatim}
 static inline void rcu_enter_nohz(void)
 {
+       mb();
        __get_cpu_var(dynticks_progress_counter)++;
-       mb();
 }

 static inline void rcu_exit_nohz(void)
 {
-       mb();
        __get_cpu_var(dynticks_progress_counter)++;
+       mb();
 }
\end{verbatim}
}
\caption{Memory-Barrier Fix Patch}
\label{fig:app:formal:Memory-Barrier Fix Patch}
\end{figure}

\begin{figure}[tbp]
{ \scriptsize
\begin{verbatim}
-       if ((curr - snap) > 2 || (snap & 0x1) == 0)
+       if ((curr - snap) > 2 || (curr & 0x1) == 0)
\end{verbatim}
}
\caption{Variable-Name-Typo Fix Patch}
\label{fig:app:formal:Variable-Name-Typo Fix Patch}
\end{figure}

This effort provided some lessons (re)learned:

\begin{enumerate}
\item	{\bf Promela and spin can verify interrupt/NMI-handler
	interactions}.
\item	{\bf Documenting code can help locate bugs}.
	In this case, the documentation effort located
	a misplaced memory barrier in
	\url{rcu_enter_nohz()} and \url{rcu_exit_nohz()},
	as shown by the patch in
	Figure~\ref{fig:app:formal:Memory-Barrier Fix Patch}.
\item	{\bf Validate your code early, often, and up to the point
	of destruction.}
	This effort located one subtle bug in
	\url{rcu_try_flip_waitack_needed()}
	that would have been quite difficult to test or debug, as
	shown by the patch in
	Figure~\ref{fig:app:formal:Variable-Name-Typo Fix Patch}.
\item	{\bf Always verify your verification code.}
	The usual way to do this is to insert a deliberate bug
	and verify that the verification code catches it.  Of course,
	if the verification code fails to catch this bug, you may also
	need to verify the bug itself, and so on, recursing infinitely.
	However, if you find yourself in this position,
	getting a good night's sleep
	can be an extremely effective debugging technique.
\item	{\bf Use of atomic instructions can simplify verification.}
	Unfortunately, use of the \url{cmpxchg} atomic instruction
	would also slow down the critical irq fastpath, so they
	are not appropriate in this case.
\item	{\bf The need for complex formal verification often indicates
	a need to re-think your design.}
	In fact the design verified in this section turns out to have
	a much simpler solution, which is presented in the next section.
\end{enumerate}

\section{Simplicity Avoids Formal Verification}
\label{app:formal:Simplicity Avoids Formal Verification}

The complexity of the dynticks interface for preemptable RCU is primarily
due to the fact that both irqs and NMIs use the same code path and the
same state variables.
This leads to the notion of providing separate code paths and variables
for irqs and NMIs, as has been done for
hierarchical RCU~\cite{PaulEMcKenney2008HierarchicalRCU}
as indirectly suggested by
Manfred Spraul~\cite{ManfredSpraul2008StateMachineRCU}.

\subsection{State Variables for Simplified Dynticks Interface}
\label{app:formal:State Variables for Simplified Dynticks Interface}

\begin{figure}[tbp]
{ \scriptsize
\begin{verbatim}
  1 struct rcu_dynticks {
  2   int dynticks_nesting;
  3   int dynticks;
  4   int dynticks_nmi;
  5 };
  6
  7 struct rcu_data {
  8   ...
  9   int dynticks_snap;
 10   int dynticks_nmi_snap;
 11   ...
 12 };
\end{verbatim}
}
\caption{Variables for Simple Dynticks Interface}
\label{fig:app:formal:Variables for Simple Dynticks Interface}
\end{figure}

Figure~\ref{fig:app:formal:Variables for Simple Dynticks Interface}
shows the new per-CPU state variables.
These variables are grouped into structs to allow multiple independent
RCU implementations (e.g., \url{rcu} and \url{rcu_bh}) to conveniently
and efficiently share dynticks state.
In what follows, they can be thought of as independent per-CPU variables.

The \url{dynticks_nesting}, \url{dynticks}, and \url{dynticks_snap} variables
are for the irq code paths, and the \url{dynticks_nmi} and
\url{dynticks_nmi_snap} variables are for the NMI code paths, although
the NMI code path will also reference (but not modify) the
\url{dynticks_nesting} variable.
These variables are used as follows:

\begin{description}
\item[\url{dynticks_nesting}:]
	This counts the number of reasons that the corresponding
	CPU should be monitored for RCU read-side critical sections.
	If the CPU is in dynticks-idle mode, then this counts the
	irq nesting level, otherwise it is one greater than the
	irq nesting level.
\item[\url{dynticks}:]
	This counter's value is even if the corresponding CPU is
	in dynticks-idle mode and there are no irq handlers currently
	running on that CPU, otherwise the counter's value is odd.
	In other words, if this counter's value is odd, then the
	corresponding CPU might be in an RCU read-side critical section.
\item[\url{dynticks_nmi}:]
	This counter's value is odd if the corresponding CPU is
	in an NMI handler, but only if the NMI arrived while this
	CPU was in dyntick-idle mode with no irq handlers running.
	Otherwise, the counter's value will be even.
\item[\url{dynticks_snap}:]
	This will be a snapshot of the \url{dynticks} counter, but
	only if the current RCU grace period has extended for too
	long a duration.
\item[\url{dynticks_nmi_snap}:]
	This will be a snapshot of the \url{dynticks_nmi} counter, but
	again only if the current RCU grace period has extended for too
	long a duration.
\end{description}

If both \url{dynticks} and \url{dynticks_nmi} have taken on an even
value during a given time interval, then the corresponding CPU has
passed through a quiescent state during that interval.

\QuickQuiz{}
	But what happens if an NMI handler starts running before
	an irq handler completes, and if that NMI handler continues
	running until a second irq handler starts?
\QuickQuizAnswer{
	This cannot happen within the confines of a single CPU.
	The first irq handler cannot complete until the NMI handler
	returns.
	Therefore, if each of the \url{dynticks} and \url{dynticks_nmi}
	variables have taken on an even value during a given time
	interval, the corresponding CPU really was in a quiescent
	state at some time during that interval.
} \QuickQuizEnd

\subsection{Entering and Leaving Dynticks-Idle Mode}
\label{app:formal:Entering and Leaving Dynticks-Idle Mode}

\begin{figure}[tbp]
{ \scriptsize
\begin{verbatim}
  1 void rcu_enter_nohz(void)
  2 {
  3   unsigned long flags;
  4   struct rcu_dynticks *rdtp;
  5
  6   smp_mb();
  7   local_irq_save(flags);
  8   rdtp = &__get_cpu_var(rcu_dynticks);
  9   rdtp->dynticks++;
 10   rdtp->dynticks_nesting--;
 11   WARN_ON_RATELIMIT(rdtp->dynticks & 0x1, &rcu_rs);
 12   local_irq_restore(flags);
 13 }
 14
 15 void rcu_exit_nohz(void)
 16 {
 17   unsigned long flags;
 18   struct rcu_dynticks *rdtp;
 19
 20   local_irq_save(flags);
 21   rdtp = &__get_cpu_var(rcu_dynticks);
 22   rdtp->dynticks++;
 23   rdtp->dynticks_nesting++;
 24   WARN_ON_RATELIMIT(!(rdtp->dynticks & 0x1), &rcu_rs);
 25   local_irq_restore(flags);
 26   smp_mb();
 27 }
\end{verbatim}
}
\caption{Entering and Exiting Dynticks-Idle Mode}
\label{fig:app:formal:Entering and Exiting Dynticks-Idle Mode}
\end{figure}

Figure~\ref{fig:app:formal:Entering and Exiting Dynticks-Idle Mode}
shows the \url{rcu_enter_nohz()} and \url{rcu_exit_nohz()},
which enter and exit dynticks-idle mode, also known as ``nohz'' mode.
These two functions are invoked from process context.

Line~6 ensures that any prior memory accesses (which might
include accesses from RCU read-side critical sections) are seen
by other CPUs before those marking entry to dynticks-idle mode.
Lines~7 and 12 disable and reenable irqs.
Line~8 acquires a pointer to the current CPU's \url{rcu_dynticks}
structure, and
line~9 increments the current CPU's \url{dynticks} counter, which
should now be even, given that we are entering dynticks-idle mode
in process context.
Finally, line~10 decrements \url{dynticks_nesting}, which should now be zero.

The \url{rcu_exit_nohz()} function is quite similar, but increments
\url{dynticks_nesting} rather than decrementing it and checks for
the opposite \url{dynticks} polarity.

\subsection{NMIs From Dynticks-Idle Mode}
\label{app:formal:NMIs From Dynticks-Idle Mode}

\begin{figure}[tbp]
{ \scriptsize
\begin{verbatim}
  1 void rcu_nmi_enter(void)
  2 {
  3   struct rcu_dynticks *rdtp;
  4
  5   rdtp = &__get_cpu_var(rcu_dynticks);
  6   if (rdtp->dynticks & 0x1)
  7     return;
  8   rdtp->dynticks_nmi++;
  9   WARN_ON_RATELIMIT(!(rdtp->dynticks_nmi & 0x1),
 10                     &rcu_rs);
 11   smp_mb();
 12 }
 13
 14 void rcu_nmi_exit(void)
 15 {
 16   struct rcu_dynticks *rdtp;
 17
 18   rdtp = &__get_cpu_var(rcu_dynticks);
 19   if (rdtp->dynticks & 0x1)
 20     return;
 21   smp_mb();
 22   rdtp->dynticks_nmi++;
 23   WARN_ON_RATELIMIT(rdtp->dynticks_nmi & 0x1, &rcu_rs);
 24 }
\end{verbatim}
}
\caption{NMIs From Dynticks-Idle Mode}
\label{fig:app:formal:NMIs From Dynticks-Idle Mode}
\end{figure}

Figure~\ref{fig:app:formal:NMIs From Dynticks-Idle Mode}
show the \url{rcu_nmi_enter()} and \url{rcu_nmi_exit()} functions,
which inform RCU of NMI entry and exit, respectively, from dynticks-idle
mode.
However, if the NMI arrives during an irq handler, then RCU will already
be on the lookout for RCU read-side critical sections from this CPU,
so lines~6 and 7 of \url{rcu_nmi_enter} and lines~19 and 20
of \url{rcu_nmi_exit} silently return if \url{dynticks} is odd.
Otherwise, the two functions increment \url{dynticks_nmi}, with
\url{rcu_nmi_enter()} leaving it with an odd value and \url{rcu_nmi_exit()}
leaving it with an even value.
Both functions execute memory barriers between this increment
and possible RCU read-side critical sections on lines~11 and 21,
respectively.

\subsection{Interrupts From Dynticks-Idle Mode}
\label{app:formal:Interrupts From Dynticks-Idle Mode}

\begin{figure}[tbp]
{ \scriptsize
\begin{verbatim}
  1 void rcu_irq_enter(void)
  2 {
  3   struct rcu_dynticks *rdtp;
  4
  5   rdtp = &__get_cpu_var(rcu_dynticks);
  6   if (rdtp->dynticks_nesting++)
  7     return;
  8   rdtp->dynticks++;
  9   WARN_ON_RATELIMIT(!(rdtp->dynticks & 0x1), &rcu_rs);
 10   smp_mb();
 11 }
 12
 13 void rcu_irq_exit(void)
 14 {
 15   struct rcu_dynticks *rdtp;
 16
 17   rdtp = &__get_cpu_var(rcu_dynticks);
 18   if (--rdtp->dynticks_nesting)
 19     return;
 20   smp_mb();
 21   rdtp->dynticks++;
 22   WARN_ON_RATELIMIT(rdtp->dynticks & 0x1, &rcu_rs);
 23   if (__get_cpu_var(rcu_data).nxtlist ||
 24       __get_cpu_var(rcu_bh_data).nxtlist)
 25     set_need_resched();
 26 }
\end{verbatim}
}
\caption{Interrupts From Dynticks-Idle Mode}
\label{fig:app:formal:Interrupts From Dynticks-Idle Mode}
\end{figure}

Figure~\ref{fig:app:formal:Interrupts From Dynticks-Idle Mode}
shows \url{rcu_irq_enter()} and \url{rcu_irq_exit()}, which
inform RCU of entry to and exit from, respectively, irq context.
Line~6 of \url{rcu_irq_enter()} increments \url{dynticks_nesting},
and if this variable was already non-zero, line~7 silently returns.
Otherwise, line~8 increments \url{dynticks}, which will then have
an odd value, consistent with the fact that this CPU can now
execute RCU read-side critical sections.
Line~10 therefore executes a memory barrier to ensure that
the increment of \url{dynticks} is seen before any
RCU read-side critical sections that the subsequent irq handler
might execute.

Line~18 of \url{rcu_irq_exit} decrements \url{dynticks_nesting}, and
if the result is non-zero, line~19 silently returns.
Otherwise, line~20 executes a memory barrier to ensure that the
increment of \url{dynticks} on line~21 is seen after any RCU
read-side critical sections that the prior irq handler might have executed.
Line~22 verifies that \url{dynticks} is now even, consistent with
the fact that no RCU read-side critical sections may appear in
dynticks-idle mode.
Lines~23-25 check to see if the prior irq handlers enqueued any
RCU callbacks, forcing this CPU out of dynticks-idle mode via
an reschedule IPI if so.

\subsection{Checking For Dynticks Quiescent States}
\label{app:formal:Checking For Dynticks Quiescent States}

\begin{figure}[tbp]
{ \scriptsize
\begin{verbatim}
  1 static int
  2 dyntick_save_progress_counter(struct rcu_data *rdp)
  3 {
  4   int ret;
  5   int snap;
  6   int snap_nmi;
  7
  8   snap = rdp->dynticks->dynticks;
  9   snap_nmi = rdp->dynticks->dynticks_nmi;
 10   smp_mb();
 11   rdp->dynticks_snap = snap;
 12   rdp->dynticks_nmi_snap = snap_nmi;
 13   ret = ((snap & 0x1) == 0) && ((snap_nmi & 0x1) == 0);
 14   if (ret)
 15     rdp->dynticks_fqs++;
 16   return ret;
 17 }
\end{verbatim}
}
\caption{Saving Dyntick Progress Counters}
\label{fig:app:formal:Saving Dyntick Progress Counters}
\end{figure}

Figure~\ref{fig:app:formal:Saving Dyntick Progress Counters}
shows \url{dyntick_save_progress_counter()}, which takes a snapshot
of the specified CPU's \url{dynticks} and \url{dynticks_nmi}
counters.
Lines~8 and 9 snapshot these two variables to locals, line~10
executes a memory barrier to pair with the memory barriers in
the functions in
Figures~\ref{fig:app:formal:Entering and Exiting Dynticks-Idle Mode},
\ref{fig:app:formal:NMIs From Dynticks-Idle Mode}, and
\ref{fig:app:formal:Interrupts From Dynticks-Idle Mode}.
Lines~11 and 12 record the snapshots for later calls to
\url{rcu_implicit_dynticks_qs},
and 13 checks to see if the CPU is in dynticks-idle mode with
neither irqs nor NMIs in progress (in other words, both snapshots
have even values), hence in an extended quiescent state.
If so, lines~14 and 15 count this event, and line~16 returns
true if the CPU was in a quiescent state.

\begin{figure}[tbp]
{ \scriptsize
\begin{verbatim}
  1 static int
  2 rcu_implicit_dynticks_qs(struct rcu_data *rdp)
  3 {
  4   long curr;
  5   long curr_nmi;
  6   long snap;
  7   long snap_nmi;
  8
  9   curr = rdp->dynticks->dynticks;
 10   snap = rdp->dynticks_snap;
 11   curr_nmi = rdp->dynticks->dynticks_nmi;
 12   snap_nmi = rdp->dynticks_nmi_snap;
 13   smp_mb();
 14   if ((curr != snap || (curr & 0x1) == 0) &&
 15       (curr_nmi != snap_nmi || (curr_nmi & 0x1) == 0)) {
 16     rdp->dynticks_fqs++;
 17     return 1;
 18   }
 19   return rcu_implicit_offline_qs(rdp);
 20 }
\end{verbatim}
}
\caption{Checking Dyntick Progress Counters}
\label{fig:app:formal:Checking Dyntick Progress Counters}
\end{figure}

Figure~\ref{fig:app:formal:Checking Dyntick Progress Counters}
shows \url{dyntick_save_progress_counter}, which is called to check
whether a CPU has entered dyntick-idle mode subsequent to a call
to \url{dynticks_save_progress_counter()}.
Lines~9 and 11 take new snapshots of the corresponding CPU's
\url{dynticks} and \url{dynticks_nmi} variables, while lines~10 and 12
retrieve the snapshots saved earlier by
\url{dynticks_save_progress_counter()}.
Line~13 then
executes a memory barrier to pair with the memory barriers in
the functions in
Figures~\ref{fig:app:formal:Entering and Exiting Dynticks-Idle Mode},
\ref{fig:app:formal:NMIs From Dynticks-Idle Mode}, and
\ref{fig:app:formal:Interrupts From Dynticks-Idle Mode}.
Lines~14 and 15 then check to see if the CPU is either currently in
a quiescent state (\url{curr} and \url{curr_nmi} having even values) or
has passed through a quiescent state since the last call to
\url{dynticks_save_progress_counter()} (the values of
\url{dynticks} and \url{dynticks_nmi} having changed).
If these checks confirm that the CPU has passed through a dyntick-idle
quiescent state, then line~16 counts that fact and line~16 returns
an indication of this fact.
Either way, line~19 checks for race conditions that can result in RCU
waiting for a CPU that is offline.

\QuickQuiz{}
	This is still pretty complicated.
	Why not just have a \url{cpumask_t} that has a bit set for
	each CPU that is in dyntick-idle mode, clearing the bit
	when entering an irq or NMI handler, and setting it upon
	exit?
\QuickQuizAnswer{
	Although this approach would be functionally correct, it
	would result in excessive irq entry/exit overhead on
	large machines.
	In contrast, the approach laid out in this section allows
	each CPU to touch only per-CPU data on irq and NMI entry/exit,
	resulting in much lower irq entry/exit overhead, especially
	on large machines.
} \QuickQuizEnd

\subsection{Discussion}
\label{app:formal:Discussion}

A slight shift in viewpoint resulted in a substantial simplification
of the dynticks interface for RCU.
The key change leading to this simplification was minimizing of
sharing between irq and NMI contexts.
The only sharing in this simplified interface is references from NMI
context to irq variables (the \url{dynticks} variable).
This type of sharing is benign, because the NMI functions never update
this variable, so that its value remains constant through the lifetime
of the NMI handler.
This limitation of sharing allows the individual functions to be
understood one at a time, in happy contrast to the situation
described in
Section~\ref{app:formal:Promela Parable: dynticks and Preemptable RCU},
where an NMI might change shared state at any point during execution of
the irq functions.

Verification can be a good thing, but simplicity is even better.


\section{Summary}
\label{app:formal:Summary}

Promela is a very powerful tool for validating small parallel algorithms.
It is a useful tool in the parallel kernel hacker's toolbox, but
it should not be the only tool.
The QRCU experience is a case in point: given the Promela validation,
the proof of correctness, and several
rcutorture
% @@@ <A HREF="http://git.kernel.org/?p=linux/kernel/git/torvalds/linux-2.6.git;a=blob;hb=HEAD;f=Documentation/RCU/torture.txt">rcutorture</A>
runs, I now feel
reasonably confident in the QRCU algorithm and its implementation.
But I would certainly not feel so confident given only one of the three!

Nevertheless, if your code is so complex that you find yourself
relying too heavily on validation
tools, you should carefully rethink your design.
For example, a complex implementation of the dynticks interface for
preemptable RCU turned out to
have a much simpler alternative implementation, as discussed in
Section~\ref{app:formal:Simplicity Avoids Formal Validation}.
All else being equal, a simpler implementation is much better than
a mechanical proof for a complex implementation!
